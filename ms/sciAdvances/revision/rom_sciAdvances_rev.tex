% Use only LaTeX2e, calling the article.cls class and 12-point type.

\documentclass[12pt]{article}

\usepackage{scicite}
\let\citep=\cite

% \usepackage{times}


\usepackage{graphicx}
\usepackage{authblk}
\usepackage{lineno}
\usepackage{paralist}
\usepackage{amsmath}

%% for citing stuff in the supp
% \usepackage{xr}
% \externaldocument{rom_sciAdvances_supp}

% The following parameters seem to provide a reasonable page setup.

\topmargin 0.0cm
\oddsidemargin 0.2cm
\textwidth 16cm 
\textheight 21cm
\footskip 1.0cm


%The next command sets up an environment for the abstract to your paper.

\newenvironment{sciabstract} 
{\bfseries}
{}



% Include your paper's title here

\title{Non-equilibrium evolution of volatility in origination and extinction explains fat-tailed
  fluctuations in Phanerozoic biodiversity}

\author[1, {*}]{Andrew J. Rominger}
\author[1, 2, 3]{Miguel A. Fuentes}
\author[1, 4, 5, 6, 7]{Pablo A. Marquet}

\affil[1]{\normalsize{Santa Fe Institute, 1399 Hyde Park Road, Santa Fe, New
Mexico 87501, US}}
%
\affil[2]{\normalsize{Instituto de Investigaciones Filos\'oficas, SADAF, CONICET,
Bulnes 642, 1428 Buenos Aires, Argentin}}
%
\affil[3]{\normalsize{Facultad de Ingenier\'ia y Tecnolog\'ia, Universidad San
Sebasti\'an, Lota 2465, Santiago 7510157, Chile}}
%
\affil[4]{\normalsize{Departamento de Ecolog\'ia, Facultad de Ciencias
Biol\'ogicas, Pontificia Universidad de Chile, Alameda 340, Santiago,
Chile}}
%
\affil[5]{\normalsize{Instituto de Ecolog\'ia y Biodiversidad, Casilla 653,
Santiago, Chile}}
%
\affil[6]{\normalsize{Laboratorio Internacional de Cambio Global (LINCGlobal),
Pontificia Universidad Católica de Chile, Alameda 340, Santiago,
Chile}}
%
\affil[7]{\normalsize{Centro Cambio Global UC, Av.~Vicu\~na Mackenna 4860, Campus
San Vicu\~na, Santiago, Chile}}
%
\affil[{*}]{\normalsize{To whom correspondence should be addressed; E-mail: rominger@santafe.edu}}

\date{}



%%%%%%%%%%%%%%%%% END OF PREAMBLE %%%%%%%%%%%%%%%%



\begin{document} 

% Double-space the manuscript.

\baselineskip24pt

% Make the title.

\maketitle 
\clearpage
\linenumbers

\begin{sciabstract}
Fluctuations in biodiversity, both large and small, are pervasive
through the fossil record, yet we do not understand the processes
generating them.
% 
Here we use a novel extension of theory from non-equilibrium
statistical physics to describe the previously unaccounted for
fat-tailed form of fluctuations in marine invertabrate richness
through the Phanerozoic.
%
Using this theory, known as super-statistics, we show that orders and
the families they subsume, are largely autonomous evolutionary units,
each likely experiencing its own unique and conserved region of an
adaptive landscape.  The separation of timescales between background
origination and extinction compared to the origin of major ecological
and evolutionary innovations between orders and families allows
within-clade dynamics to reach equilibrium, while between-clade
diversification is non-equilibrial.
%
This between clade non-equilibrium accounts for the fat-tailed nature
of the system as a whole.
%
The distribution of shifts in diversification dynamics across orders
and families is consistent with niche conservatism and pulsed
exploration of adaptive landscapes by higher taxa.
%
Compared to other approaches that have used simple birth-death
processes, equilibrial dynamics, or non-linear theories from
complexity science, super-statistics is superior in its ability to
account for both small and extreme fluctuations in the richess of
fossil taxa.
% 
Its success opens up new research directions to better understand the
evolutionary processes leading to the stasis of order- and
family-level occupancy in an adaptive landscape interrupted by
innovations that lead to new orders and families.
\end{sciabstract}

\section*{Introduction}

%% Diversity changes
Biodiversity has not remained constant nor followed a simple
trajectory through geologic time \citep{raup1982, sepkoski1984,
  gilinsky1994, liow2007, alroy08}.  Instead, it has been marked by
fluctuations in the richness of taxa, both positive in the case of net
origination, or negative in the case of net extinction. Major events,
such as adaptive radiations and mass extinctions have received special
attention \citep{benton1995, Erwin1998}, but fluctuations of all sizes
are ubiquitous \citep{sepkoski1984, alroy08, quental2013} and follow a
fat-tailed distribution, i.e. where large events are more probable
compared to models such as the Gaussian distribution. Understanding
the fat-tailed nature of these fluctuations continues to elude
paleobiologists and biodiversity theoreticians.

%% What drives diversity dynamics?
The fat-tailed distribution of fluctuations in taxon richness inspired earlier
researchers to invoke ideas from complex systems with similar
distributions. Such ideas include the hypotheses that biological
systems self-organize to the brink of critical phase-transitions
\citep{bak1993, sole1997}, and that environmental perterbations are
highly non-linear \citep{newman1995}. New data and analyses have not,
however, supported these hypotheses at the scale of the entire
Phanerozoic marine invertebrate fauna \citep{kirchner1998, alroy08}.
Other studies have modeled the mean trend in taxon richness as tracking a
potentially evolving equilibrium \citep{sepkoski1984, alroy2010,
  rabosky2009ecolLett, marshall2016} and yet ignore the potential role
of stochasticity and non-equilibrium dynamics in producing observed
patterns \citep{erwin2012, liow2007, quental2013, harmon2015,
  jordan2016}. Individual, population, and local ecosystem scale
processes that could produce complex dynamics, such as escalatory
co-evolutionary interactions \citep{vermeij1987}, have not been
documented to scale to global patterns \citep{madin2006} and indeed
should not be expected to scale as such \citep{vermeij2008}.  Thus, we
still lack a new hypothesis to describe the striking fat-tailed nature
of fluctuations throught the Panerozoic.


Despite the heterogeneity of explanations of Phanerozoic biodiversity,
consensus has emerged on one property of macroevolution: clades
experience different rates of morphological evolution, speciation and
extinction \citep{simpson1953, sepkoski1984, holman1989, gilinsky1994,
  stadler2011, rabosky2014}. Here we show that the simple fact of
conserved rates within clades and variable rates across clades is
sufficient to describe pervasive, fat-tailed fluctuations in taxonomic
richness throughout the marine Phanerozoic.  This biological mechanism
has a precise correspondence to non-equilibrial theory, known as
``superstatistics'' derived in statistical mechanics \citep{beck2003}
and applied across the physical and social sciences \citep{beck2004,
  fuentes2009}. We leverage this correspondence to derive a robust
prediction of the distribution of fluctuations in the standing
richness of marine invertebrates preserved in the Phanerozoic fossil
record. We further show that the specific mathematical derivation of
this superstatistical mechanism is consistent with niche conservatism
\citep{roy2009range, hopkins2014} and pulsed exploration on an
adaptive landscape by higher taxa \citep{eldredgeGould1972,
  newman1985adaptive, hopkins2014}.


\subsection*{Superstatistics of fossil biodiversity}




Superstatistics \citep{beck2003} proposes that non-equilibrial systems
can be decomposed into many local sub-systems, each of which attains a
unique dynamic equilibrium. The evolution of these dynamic equilibria
across sub-systems evolves more slowly. This separation in time scale
allows local systems to reach equilibrium while the system as a whole
is far from equilibrium \citep{beck2003}.  In the context of
macroevolution we propose that a clade with conserved evolutionary
rates and life history characteristics corresponds to a sub-system in
dynamic equilibrial.

In statistical mechanics, local sub-systems can be defined by a simple
statistical parameter $\beta$ often corresponding to inverse
temperature. In macroevolutionary ``mechanics'' we define the
$\beta_k$ of clade $k$ as the inverse variance of fluctuations $x_k$
in the number of genera within that clade, i.e. fluctuations in the
genus richness.  The $\beta_k$ thus represent the inverse variances of
the stationary distribution of a homogeneous origination-extinction
processes of genera. Fluctuations from this stationary process will be
approximately Gaussian if the clades' diversification dynamics are
independent and in local equilibrium (\citep{keilson1970,
  grassmann1987} see Supplemental Section
\ref{sec:suppLimitDist}). The variance of these within-clade Gaussian
fluctuation distributions is precisely what we mean by volatility.

We make the hypothesis of dynamic equilibrium within a clade following
MacArthur and Wilson \citep{macWilson} in recognition that while the
identity and exact number of taxa will fluctuate stochastically from
random origination and extinction (taking the place of local
immigration and extinction in \citep{macWilson}), the overall process
determining the number of taxa, and by extension, fluctuations in that
number, is in equilibrium. Indeed, the different regions of adaptive
space occupied by different clades can be conceptualized as different
islands with different dynamic equilibria, albeit with
macroevolutionary processes determining the colonization of adaptive
peaks, as opposed to short timescale biogeographic processes.

Variation in the volatility of richness fluctuations across these
islands of adaptive space will correspond to the life histories,
ecologies, and evolutionary histories that characterize each
region. We do not attempt to disagnose which characteristics of
different regions of adaptive space account for volatility
differences, but others have found relationships between larval type
\citep{jablonski2008}, body plan \citep{erwin2012}, body size
\citep{harnik2011}, range size \citep{harnik2011, foote2008paleobiol},
and substrate preference \citep{hopkins2014} on rates of origination
and extinction. Not all of these traits would be considered dimensions
of an ecological niche or characteristics of a guild \citep{bambach},
but they all point to different ecological strategies that may be more
or less favorable macroevlutionarily. These characteristics result
from interactions between heritable traits and environments, which may
also be viewed as semi-heritable \citep{nicheCons}. Thus different
regions of adaptive space, and the clades occupying them, will
experience different magnitudes of stochastic fluctuations in
taxonomic richness. Indeed, there is evidence that extinction rate as
a trait is phylogenetically conserved
\citep{rabosky2009heritability}. As clades occasionally split to fill
new regions of adaptive space their pulsed diversification determines
the non-equilibrium nature of the entire biota.


\subsection*{Real paleontological data to test superstatistics}

To uncover the superstatistical nature of the marine invertebrate
Phanerozoic fauna we analyze the distribution of fluctuations in the
number of genera (the lowest reliably recorded taxonomic resolution)
using the Paleobiology Database (PBDB; {\tt paleobiodb.org}). We
correct these raw data for incomplete sampling and bias using a new
approach described in the methods section. Occurrences from the PBDB
were matched to 49 standard timebins all of approximately 11MY in
duration following previous publicatoins \citep{alroy08,
  alroy2010}. Fluctuations in genus richness were calculated as the
simple difference between bias-corrected richnesses in adjacent
timebins. 

To focus attention on the variance of fluctuations we zero-centered
each clade's fluctuation distribution. In this way we focus on
fluctuations about any possible trend toward net diversificaiton or
extinction. Because ``equilibrium'' in the statistical mechanical
sense means a system undergoes coherent, concerted responses to
perturbation, the mean trend line (positive or negative) is of less
interest than deviations from it. We also note that the distributions
of fluctuations for most clades are already very close to a mean of 0
(mean of fluctuation distributions at the family level:
$0.038 \pm 0.176 \text{ SD}$), and so centering has little influence
on clade-specific fluctuation distributions.

We define potentially equilibrial sub-systems based on taxonomic
hierarchies as a full phylogenetic hypothesis for all marine
invertebrates is lacking.  Taxa ideally represent groups of organisms
that descend from a common ancestor and share similar ecologically and
evolutionary relevant morphological traits \citep{mayr1965systZool,
  erwin2007, ezard2016}. Thus our model assumes that at a given higher
taxonomic level (above genus level), within-taxon fluctuations in
richness are driven by equillibrial processes characterized by
Gaussian distributions. We further assume that new higher taxa arrise
due to the emergence of sufficiently novel traits (be they ecological
such as substrate preference, morphological such as ontology, or
macroecological such as range size) so that those new taxa occupy a
new region of adaptive space. We lastly assume that different regions
of adaptive space are characterized by different volatilities in
origination and extinction.

To evaluate the optimal taxonomic level for sub-system designation, we
test our superstatistical theory using taxonomic levels from family to
phylum. Additionally, we compare our results to randomized taxonomies
and confirm that the observed fit of superstatistical theory is not an
artifact of arbitrary classification but instead represents real,
biologically relevant diversification processes within and between
clades. We find that families and orders conform to the assumptions of
our superstatistical model while classes and orders do not.

\section*{Results}

 Three
exemplar family-level dynamics are shown in Figure \ref{fig:pk_f}, and
indeed all richness fluctuations within families are well
characterized by a Gaussian distribution (Fig. \ref{fig:pk_f}).

We compare the Gaussian fit to these family-level distributions with a
fat-tailed distribution (specifically the super-statistical
distribution we derive in equation \ref{eq:Px}) and find that the
non-fat-tailed Gaussian better explains these data (see Supplemental
Section \ref{sec:suppGaussian}.


\begin{figure}[!h]
  \centering
  \includegraphics[scale=0.8]{../../fig_pkx-fbeta.pdf}
  \caption[Variability in trajectories of within-order fluctuations in
  genus richness]{The distributions of within-order fluctuations in
    genus richness shown for the trajectories of three exemplar
    orders (A) and shown as an empirical cumulative density aggregated
    across all orders (B). To display all orders simultaneously we
    simply collapse their fluctuation distributions by dividing by
    their standard deviations. If orders conform to the Gaussian
    hypothesis their scaled fluctuations should fall along the
    cumulative density line of a normal N(0, 1) distribution, as shown
    in (B). In (C) the distribution of inverse variances $\beta_k$
    across all orders matches very closely to a Gamma distribution
    (black line); exemplar orders are again highlighted.}
  \label{fig:pk_f}
\end{figure}


To predict the super-statistical behavior of the entire marine
invertebrate Phanerozoic fauna we must integrate over all possible
local equilibria that each clade could experience. The stationary
distribution of $\beta_k$ values describes these possible equilibria,
specifying the probability that a given clade, chosen at random, will
occupy a region of adaptive space characterized by $\beta_k$.

%%% move elsewhere if not said already
% The form of
% this stationary distribution could shed interesting light on the
% biological processes that lead different clades to explore different
% regions of adaptive landscapes, and thus different equilibria, as
% discussed below.
%%%

We estimate the distribution of $\beta_k$'s simply as the maximum
likelihood distribution describing the set of inverse variances for
all orders. Phanerozoic marine invertebrate families clearly follow a
Gamma distribution in their $\beta_k$ values (Fig. \ref{fig:pk_f}).

Using the observation of within family statistical equilibrium and
Gamma-distributed $\beta_k$ parameters we can calculate, without
further adjusting free parameters, the distributions of family-level
fluctuations for the entire marine Phanerozoic, $P(x)$, as
\begin{equation}
  P(x) = \int_0^\infty p_k(x \mid \beta) f(\beta) d\beta \label{eq:PxInt}
\end{equation}
where
$p_k(x \mid \beta) = \sqrt{\frac{\beta}{2\pi}} e^{-\frac{\beta
    x^2}{2}}$ is the distribution of fluctuations within an order and
$f(\beta) = \frac{1}{\Gamma(b_1/2)}
\left(\frac{b_1}{2b_0}\right)^{b_1/2} \beta^{(b_1/2) - 1}
\text{exp}\left(-\frac{b_1 \beta}{2 b_0}\right)$ is the stationary
distribution of inverse variances in the magnitude of order-level
fluctuations in richness. The integral in (\ref{eq:PxInt}) leads to
\begin{equation}
  \label{eq:Px}
  P(x) = \frac{\Gamma\left(\frac{b_1 +
        1}{2}\right)}{\Gamma\left(\frac{b_1}{2}\right)}
  \sqrt{\frac{b_0}{\pi b_1}} \left(1 + \frac{b_0
      x^2}{b_1}\right)^{-\frac{b_1 + 1}{2}}
\end{equation}
This corresponds to a non-Gaussian, fat-tailed prediction for $P(x)$
which closely matches family-level fluctuations in the bias-corrected
PBDB (Fig. \ref{fig:Px}).

\begin{figure}[!h]
  \centering
  \includegraphics[scale=1]{../../fig_Px.pdf} 
  \caption[Order-level distribution of richness
  fluctuations]{Distribution of fluctuations in genus richness within
    orders of marine invertebrates in the Paleobiology Database
    \citep{alroy08} after sampling correction. The distribution is
    fat-tailed as compared to the maximum likelihood estimate of the
    normal distribution (blue line).  At the order level the empirical
    distribution of fluctuations are well described by our
    super-statistical approach, both when computed from integrating
    over the distribution of observed variances (red line) and when
    fit via maximum likelihood (95\% confidence interval; red
    shading).}
  \label{fig:Px}
\end{figure}

To quantitatively evaluate how well the super-statistical prediction
matches the family-level data we constructed a 95\% confidence
envelope from bootstrapped maximum likelihood estimates for
$P(x)$. Observed fluctuations fall within this 95\% confidence
envelope (Fig. \ref{fig:Px}), indicating that the data do not reject
the super-statistical prediction. For further comparison, we fit a
Gaussian distribution to the observed fluctuations, which corresponds
to the equilibrium hypothesis that all orders conform to the same
dynamic. Using Akaike Information Criterion (AIC) we find that
observed fluctuations are considerably better explained by the
super-statistical prediction than by the Gaussian hypothesis 
%%%%%%%%%% UPDATE!!!!!
({\small $\Delta$}AIC = 11285.18). Thus, as expected under the
%%%%%%%%%%%%%%%%%%%
superstatistical hypothesis, the fat-tailed distribution of
fluctuations arise from the superposition of independent Gaussian
statistics of fluctuations within families.

%% calculating sstat at different taxonomic levels (D-stat fig)
Computing the distribution of fluctuations using classes instead of
orders leads to a substantially poorer fit to the observed data
(Fig. \ref{fig:supp_PBDB_Px_cls}). We quantify this shift with the
Kolmogorov-Smirnov statistic, which changes from 0.041 for orders to
0.062 for classes (Fig. \ref{fig:dStat}). However, if
super-statistical theory explains the data, this worsening fit with
increasing taxonomic scale is expected as the different classes should
not represent dynamically equilibrial sub-systems in their fluctuation
dynamics. Instead, classes aggregate increasingly disparate groups of
organisms, and thus effectively mix their associated Gaussian
fluctuations, meaning that one statistic should no longer be
sufficient to describe class-level dynamics. 

Our analysis indicates that orders are evolutionarily equilibrial and
independent entities, with all subsumed taxa sharing key ecological
and evolutionary attributes allowing them to reach steady state
diversification independently from other orders. Both the good fit at
the order level and worsening fit at higher taxonomic levels is
confirmed in Sepkoski's compendium.

\begin{figure}[!h]
  \centering
  \includegraphics[scale=1]{../../fig_dStat.pdf}
  \caption[Goodness of super-statistical theory fit]{Distribution of
    Kolmogorov-Smirnov (KS) statistics from randomly permuting genera
    within orders (gray shading represents 95\% confidence
    interval). Solid black line is observed KS statistic at the order
    level, while the dashed black line shows the observed KS statistic
    at the class level.}
  \label{fig:dStat}
\end{figure}

To further test the evolutionary coherence of orders we conducted a
permutation experiment in which genera were randomly reassigned to
orders while maintaining the number of genera in each order. For each
permutation, we calculated the super-statistical prediction and the
Kolmogorov-Smirnov statistic. The permutation simulates a null model
in which common evolutionary history is stripped away (genera are
placed in random orders) but the total number of observed genera per
order is held constant.  Repeating this permutation 500 times yields a
null distribution of Kolmogorov-Smirnov statistics that is far
separated from the observed value (Fig. \ref{fig:dStat}) suggesting
the good fit at the order level is not merely a statistical artifact
of classification but carries important biological information.

\section*{Discussion}

%% why orders
Our analysis makes no assumption that orders should correspond to
super-statistical subsystems, but identifies them as the appropriate
level for marine invertebrates. As we show, orders differ only in the
variances of their richness fluctuations (Fig. \ref{fig:pk_f}).

Our study is the first to demonstrate that complex patterns in the
fluctuation of taxon richness resulting from the sequence of origination
and extinction events in the fossil record are the result of a simple
underlying process analogous to the statistical mechanisms by which
complexity emerges in large, non-equilibrium physical \citep{beck2004}
and social systems \citep{fuentes2009}.  We do so by identifying the
biological scale at which clades conform to locally independent
dynamic equilibria in fluctuations. This scale is determined by the
process of niche conservatism \citep{roy2009range, hopkins2014} within
orders.  Equilibrium could result from many processes, including
neutrality \citep{macWilson, hubbell2001}, diversity-dependence
\citep{gavrilets2005, rabosky2009ecolLett} and processes that
dampen---rather than exacerbate---fluctuations in complex ecological
networks \citep{berlow2009}. 

%%%% this could go here or elsewhere
Independent and dynamically equilibrial dynamics suggested by these
Gaussian fluctuations in genus richness could result from neutral-like
processes \citep{hubbell2001}, where the dynamics of one taxon are
unaffected by those of another, or from dampening mechanisms that
stabilize complex networks of interacting taxa \citep{brose2005}. This
is in direct contrast to the instability hypothesis underlying the
self-organized criticality theory of paleo-biodiversity
\citep{bak1993, sole1997}.
%%%%%



We then show that punctuated shifts to different equilibria between
order, a consequence of punctuated exploration of niche space by newly
evolving clades \citep{eldredgeGould1972, newman1985adaptive,
  hopkins2014}, leads to a characteristically non-equilibrial
distribution of richness fluctuations when the marine Phanerozoic
fauna is viewed as a whole macro-system.

The distribution describing this process of evolution in equilibria
between orders is clearly Gamma (Fig. \ref{fig:pk_f}.  A Gamma
distribution, while consistent with multiple processes (e.g.,
\citep{cir1985}), could result from evolution of diversification rates
across an adaptive landscape that promotes niche conservatism and
punctuated exploration of niche space.  Specifically, if $\beta_k$
values are associated with a clade's physiological and life history
traits, and those traits evolve via Ornstein-Uhlenbeck-like
exploration of an adaptive landscape, the resulting stationary
distribution of $\beta_k$ will be Gamma \citep{cir1985, butler2004}.
For macroevolutionary rates to vary across an adaptive landscape, this
landscape cannot be flat, and thus niche conservatism punctuated by
adaptive exploration is inevitable \citep{newman1985adaptive}. The
specifics of how this adaptive landscape is shaped and is traversed by
evolving clades will likely determine the specific distribution
(e.g. Gamma versus Chi-squared, etc.) describing punctuated evolution
of clades' equilibria.  Our work thus motivates study of the trait
spaces and evolutionary shifts consistent with Gamma-distributed
equilibria in richness fluctuations.

%%% how this motivates future work in figuring out the details of
%%% biotic evolution that are important for driving stasis and
%%% punctuation
Our work highlights the importance of both niche conservatism and
punctuated adaptive radiation in producing the statistical behavior of
the Phanerozoic; our theory thus provides new motivation for
identifying the eco-evolutionary causes of innovations between
lineages and how those innovations are eventually conserved within
lineages. Armed with an understanding of the statistical behavior of
diversification we can go on to examine mechanisms underlying
additional patterns in the mean trend of biodiversity through the
Phanerozoic. In particular, clades have been shown to wax and wane
systematically through time \citep{liow2007,
  quental2013}, a pattern that we cannot explain with super-statistics
alone.

%% note on how sstat could be applied to other questions in eco-evo.
Superstatistics could also be applied to other areas of evolution and
macroecology.  For example new phylogenetic models already consider
heterogeneous rates of diversification (e.g.,
\citep{rabosky2006laser}). The superstatistics of clades in adaptive
landscapes could provide a means to build efficient models that
jointly predict morphological change and diversification. This
framework could also provide a new paradigm in modeling the
distributions of richness, abundance and resource use in non-neutral
communities. Non-neutral models in ecology are criticized for their
over-parameterization \citep{rosindell2011}, yet a persistent counter
argument to neutral theory \citep{hubbell2001} is the unrealistic
assumption of ecological equivalency \citep{chave2004neutral} and poor
prediction of real dynamics \citep{ricklefs2006neutral}. If ecosystems
are viewed as the super-position of many individualistically evolving
clades, each exploiting the environment differently and thus obeying a
different set of non-equivalent statistics, then diversity dynamics
could be parsimoniously predicted with superstatistics while
incorporating real biological information on ecological differences
between taxa.

Superstatistics is a powerful tool to derive macro-scale predictions
from locally fluctuating sub-systems whose evolution is driven by
interesting, but complex and difficult to model, biological
mechanisms. As such, applications of superstatistics from islands to
populations to clades are ripe for exploration.


\section*{Methods and Materials}

\subsection*{Paleobiology Database data download and filtering}
Data were downloaded from the Paleobiology Database (PBDB; {\tt
  https://paleobiodb.org}) on 15 November 2018 via the database's API
(data retrival and processing script in the supplement). Collections
were filtered using the same approach as Alroy \citep{alroy08} to
insure that only well preserved marine invertebrate occurrences were
used in subsequent analyses. This filtering resulted in 221202 genus
occurrences. These were further filtered to exclude those occurrences
without family-level taxonomy and those collections with age estimate
resolutions outside the 11MY timebins proposed by Alroy
\citep{alroy08} resulting in 189516 occurrences. These timebins were
compiled from {\tt http://fossilworks.org} with a custum script
reproduced in the supplement. The first and last of these timebins,
corresponding to the earliest Cambrian and the latest Cenozoic, were
excluded from analysis because their sampling completeness (see below)
could not be assessed.


%%%%%%% FIX THE ABOVE!!!!!

\subsection*{Correcting for imperfect and potentially biased sampling} 
\label{sec:3TP}
We use a new and flexible method to correct for known sampling
incompleteness and biases in publication-based specimen databases
\citep{alroy08, alroy2010}. Incompleteness is inherent in all
biodiversity samples, the fossil record being no exception
\citep{miller1996, alroy2010, foote2016, starrfelt2016,
  close2018}. This incompleteness is only an issue if it varies across
the fossil record, as indeed it does \citep{alroy2010, foote2016,
  starrfelt2016, close2018}.  In addition to variable incompleteness,
bias may result from preferential publication of novel taxa
\citep{alroy2010} which exacerbates the difference between
poorly-sampled and well-sampled time periods. We therefore develop a
simple two-step method: we first correct for incomplete sampling using
the ``three-timer'' correction \citep{alroy08} and then further
correct this three-timer estimate by accounting for any correlation
between the number of genera and the number of publications in a time
period.

The three-timer correction estimates the probability of failure to
observe a genus in a given time period $p_t$ as the number of times
any genus is recorded before and after that period but not during,
divided by the number of genera whoes occurrence histories span the
period in question.  To calculate the sampling-corrected richness
$\hat{D}_{kt}$ of a clade $k$ in the time period in question, the observed
genera within that clade and time period are divided by $1 - p_t$ and
their occurrences summed:
\begin{equation}
  \hat{D}_{kt} = \sum_{j \in k} \frac{I_{jt}}{1 - p_t}
\end{equation}
where $j \in k$ designates genera in clade $k$ and $I_{jt}$ is an
indicator equal to 1 if the genus $j$ occurs in time period $t$.

The estimator $\hat{D}_{kt}$ is the maximum likelihood estimator of
richness in a simple occupancy through time type model assuming
binomial sampling \citep{}, and in that way mimics other proposed
methods for the fossil record \citep{foote2016, starrfelt2016}. We
avoid parametrically modeling the sampling process through time by
instead taking a sliding window of timebins from the Cambrian to the
Cenozoic. It should be noted that the three-timer correction compares
favorably to other similar methods to account for imperfect detection
\citep{alroy2014}

To eliminate further bias due to preferential publication of novel
taxa \citep{alroy2010} we divide the three-timer-corrected number of
genera per family per time period by the expected number of genera
given publications in that time period.  The expected number is
calculated by regressing the log-transformed three-timer-corrected
number of genera on log-transformed number of publications. There is
only a weak trend toward higher richness with more publications
(Fig. \ref{fig:divByPub}) meaning that the most important correction
comes from the three timer correction.

Our new method re-scales each genus occurrence from 0 or 1 (absent or
present) to a weighted number continuously ranging between 0 and
1. Because these weighted numbers represent sampling and
bias-corrected {\it occurrences} we can add them arbitrarilly,
corresponding to the membership of a genera in any given taxonomic
group from the level of family and above.  We must, however, choose a
taxonomic level at which to evaluate the relationship between richness
and publications; we choose the level of family because this is the
most finely resolved option.

This method achieves similar results at the global scale across all
clades to more computationally intensive subsampling procedures
\citep{miller1996, alroy2010, kocsis2018}. These subsampling would not
even be advisable for clades with few genera. We directly compare our
predicted time series of global fluctuations in genus richness with
results derived from rarifaction and shareholder quarum subsampling
(SQS; \citep{kocsis2018}) in Figure \ref{fig:supp_3TPub}.  Our method
shows very minor differences with these subsampling-based predictions
and any discrepancies do not impact the statistical distribution of
fluctuations (Fig. \ref{fig:supp_3TPub}).


\subsection*{Super-statistical methods} \label{sec:numMeth}



We first derive the super-statistical distribution $P(x)$ by fitting
Gaussian distributions to clade-level distributions of fluctuations
$p_k(x)$, extracting the inverse variances $\beta_k$ of those
$p_k(x)$, testing the best function to describe the distribution of
$\beta_k$, and then integrating
$P(x) = \int_{\beta}p_k(x | \beta) f(\beta)$. This process allows to
free parameters to hone the fit of $P(x)$ to the data.  However, each
inverse variance must of course be estimated for each clade.  To do so
we use least squares instead of maximum likelihood because the
asymmetric fluctuation distributions of small clades were more
reliably fit with curve fitting than with the maximum likelihood
estimator.

We then estimated $P(x)$ directly from the raw data using maximum
likelihood to compare the fit of our super-statistical prediction and
that of a simple Gaussian distribution using AIC. To calculate a
likelihood-based confidence interval on our prediction we bootstrapped
the data, subsampling fluctuations with replacement from all orders
combined, and calculating AIC of the superstatistical and Gaussian
models on these bootstrapped datasets.

\bibliographystyle{Science}
\bibliography{../../superStat}


\section*{Acknowledgments}
\begin{itemize}
\item[{\bf General:}] We thank John Harte, Rosemary Gillespie, Linden
  Schneider, and Jun Ying Lim for helpful discussion. We thank the
  many contributors to the Paleobiology Database for making data
  available.
\item[{\bf Funding:}] AJR thanks funding from Fulbright Chile, the
  National Science Foundation Graduate Research Fellowship Program and
  the Omidyar Program at the Santa Fe Institute; MAF thanks FONDECYT
  1140278; PM thanks CONICYT PFB-023, ICM-P05-002 and FONDECYT
  1161023.
\item[{\bf Author contributions:}] AJR, MAF and PAM designed the
  study; AJR and MAF preformed the analyses; AJR, MAF and PAM
  interpreted the results and wrote the manuscript.
\item[{\bf Competing interests:}] none.
\item[{\bf Data and materials availability:}] Data are available
  through the Paleobiology Database ({\tt paleobiodb.org}) and all
  code needed to interface with the {\tt paleobiodb.org} API, process,
  clean, and ultimately analyze the data are available online at {\tt
    github.com/ajrominger/paleo\_supStat}. This github repository also
  hosts the exact download from {\tt paleobiodb.org} used in this
  analysis. Scripts are also availible and explained in the
  Supplement.
\end{itemize}


\clearpage

\newcommand{\beginsupplement}{%
  \setcounter{table}{0}
  \renewcommand{\thetable}{S\arabic{table}}%
  \setcounter{figure}{0}
  \renewcommand{\thefigure}{S\arabic{figure}}%
  \setcounter{section}{0}
  \renewcommand{\thesection}{S\arabic{section}}%
}

\beginsupplement

\begin{center}
{\LARGE \bf Supplementary materials}
\end{center}
\vspace{2em}

\section{Limit distribution of a time-averaged homogeneous
  origination-extinction process}
\label{sec:suppLimitDist}
Fossil taxa gain and lose taxa according to an origination-extinction
process. We assume that most fossil occurrences of a taxon come from
the period of its history when it is dominant and in steady state. In
a time slice of duration $\tau$ during such a period of steady state
the latent per capita rates of origination and extinction would be
equal (i.e. $\lambda = \mu \equiv \rho$) and the number of origination
or extinctions events (call such events $Y$) each follow an
inhomogeneous Poisson process with rate $\rho N_t$ where $N_t$ is the
number of species or genera in the taxon of interest at time
$t$. Allowing $N_t$ to vary smoothly with time, and recognizing that
the sum of Poisson random variables remains Poisson, we arrive at the
number $Y$ of extinction \emph{or} origination events in $\tau$ being
distributed
\begin{equation}
  \label{eq:eventPois1}
  Y \sim \text{Pois}(\rho \int_{t=0}^\tau N(t) dt).
\end{equation}
Under the steady state assumption we can approximate $N(t)$ by
$\bar{N}$, the steady state richness, leading to
\begin{equation}
  \label{eq:eventPois2}
  Y \sim \text{Pois}(\rho \bar{N} \tau).
\end{equation}

Assuming the $\tau$ of each time period in the Paleobiology Database
or Sepkoski's compendium to be approximately equal (i.e. equal
durations of major asymptotic units) then the distribution of
fluctuations within taxa will be asymptotically Gaussian.

The Gaussian asymptotics of time-averaged birth-death processes have
been proven and explored elsewhere as well \citep{keilson1970,
  grassmann1987}.

\section{Additional super-statistical analyses}
To evaluate the sensitivity of our super-statistical analysis on the
particular data used and we tested our predictions on different data
sets (see below). The fact that it works in all different applications
indicates that it is robust to vagaries of different recording
strategies and bias corrections in paleobiology. This could mean that
much of the raw signal in massive fossil datasets, at least signals
regarding fluctuations, are not artifacts of sampling, as has been
proposed before \citep{hannisdal2011}.

\subsection{Raw PBDB data} \label{sec:rawPBDB}
We calculated the super-statistical prediction at the order level from
raw genus richness recorded in the PBDB without correcting for
taphonomic or sampling bias (Fig. \ref{fig:supp_rawPBDB_Px}). The
super-statistical calculation also closely fits the raw data as in the
case of sampling and publication bias-corrected data.

\subsection{Different taxonomic ranks in PBDB data}
As noted in the main text, the super-statistical prediction
predictably breaks down at higher taxonomic scales. In Figure
\ref{fig:supp_PBDB_Px_cls} we present this worsening fit graphically
using class level data with three-timer and publication corrected PBDB
data

\subsection{Sepkoski's compendium} \label{sec:suppSepk}
Sepkoski's compendium \citep{sepkoski1992} provided the first
hypothesis of Phanerozoic diversification.  As such, it has served as
a benchmark for further investigation into large-scale paleobiological
patterns \citep{alroy08}.  We conducted the same super-statistical
analysis as in the main text and find comparable results.
Specifically, the super-statistical prediction far out preforms the
null Gaussian model (Fig. \ref{fig:supp_sepkPx}) and worsens with
increasing taxonomic scale (Fig. \ref{fig:supp_sepkPx}), again
implying the uniqueness of orders.

\section*{Supplemental Figures}

\begin{figure}[!hp]
  \centering
  \includegraphics[width=0.4\textwidth]{../../figSupp_divByPub.pdf}
  \caption[Relationship between number of publications and genus
  richness]{Relationship between number of publications and genus
    richness as recorded by the PBDB.}
  \label{fig:divByPub}
\end{figure}

\begin{figure}[!hp]
  \centering
  \includegraphics[width=0.7\textwidth]{../../figSupp_divEstComp.pdf}
  \caption[Comparison of SQS method with the raw data and three-timer
  bias correction method]{Comparison of SQS method \citep{alroy2010}
    (solid black line) with the raw data (dashed black) and our
    three-timer and publication bias correction method (red). The
    time-series of all marine invertebrate genera shows general
    agreement with the only major deviations toward the modern
    (A). Despite these differences the distribution of fluctuations in
    genus richness across all marine invertebrates show good
    agreement (B).}
  \label{fig:supp_3TPub}
\end{figure}


\end{document}

