\documentclass[12pt]{article}
\usepackage{authblk}
\usepackage[margin=1in]{geometry}
\geometry{letterpaper}
\usepackage{graphicx}
\usepackage{amssymb}
\usepackage{amsmath}

\usepackage[style=nature, natbib=true, autocite = superscript]{biblatex}
\addbibresource{../../superStat.bib}
\let\citep=\autocite

\begin{document}

\noindent
28 July 2017

\vspace{2em}

\noindent
Dear editors,
\vspace{2em}

Please find attached our manuscript ``Punctuated non-equilibrium and
niche conservatism explain biodiversity fluctuations through the
Phanerozoic'' that we wish to submit for publication as an Article in
{\it Nature Ecology and Evolution}.

Our study is the first to demonstrate that complex, previously
unexplained patterns in the sequence of origination and extinction
events in the fossil record are the result of a simple underlying
process emerging from non-equilibrium evolution on an adaptive
landscape \citep{eldredgeGould1972, newman1985adaptive}. Our theory
provides a novel explanation for deep time diversity dynamics invoking
emergence of lineage-level traits as the drivers of complexity via the
same mechanisms by which complexity emerges in large physical
\citep{beck2004} and social systems \citep{fuentes2009}. In the context
of fossil diversity we show how this complexity arises naturally from
the uniquely biological mechanisms of punctuated adaptive radiation
\citep{eldredgeGould1972, newman1985adaptive, hopkins2014} followed by
long durations of niche conservatism \citep{ackerly2003, roy2009range,
  hopkins2014} and thus identify these mechanisms as sufficient and
necessary to produce observed patterns in the fossil record.

Using two seminal fossil datasets \citep{sepkoski1992, alroy08} we show
that fluctuations in marine biodiversity over the past 550 million
years results from the superposition of many independently fluctuating
subsystems whose fluctuations are Gaussian but give rise to
non-Gaussian patterns when combined.  These independent subsystems
correspond to lineages of closely related animal taxa, implying that
diversification within lineages is driven by random additive
interactions with the environment. Our findings thus challenge the
idea that changes in origination and extinction through deep geologic
time are the result of complicated evolutionary interactions among
organisms and between organisms and their environment \citep{bak1993,
  sole1997, newman1995}. However, we demonstrate that the evolutionary
process responsible for generating new lineages varies slowly through
time, possibly driven by non-random evolutionary innovations in the
physiology and demography of new lineages. This slow change between
lineages produces patterns of apparent complexity earlier ascribed to
unnecessarily complicated mechanisms. We further show, using
permutational null models, that our findings are not an artifact of how
fossils are taxonomically classified but rather capture true
underlying biological processes.

Our work will interest a lay audience because biodiversity, or the
fact that life takes many forms not just one, is a striking feature of
our world. More striking still is the fact that over the past 550
million years biodiversity has fluctuated between periods of rapid
diversification, such as the Cambrian explosion, and devastating
extinctions, such as the end Permian extinction.  We shed light on how
these seemingly complicated biodiversity patterns share common
mechanisms with other striking features of our world such as weather,
a non-equilibrium physical system, and vagaries of the stock market.

This work has not been published or accepted for publication, and is
not under consideration for publication elsewhere. Thank you very much
for your consideration.
\vspace{2em}

\noindent
Sincerely,
\vspace{2em}

\noindent
Andy Rominger

\noindent
Santa Fe Institute \\
1399 Hyde Park Road \\
Santa Fe, New Mexico 87501, USA \\
{\tt rominger@santafe.edu}

\printbibliography

\end{document}