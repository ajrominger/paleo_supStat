\documentclass[]{article}
\usepackage{lmodern}
\usepackage{amssymb,amsmath}
\usepackage{ifxetex,ifluatex}
\usepackage{fixltx2e} % provides \textsubscript
\ifnum 0\ifxetex 1\fi\ifluatex 1\fi=0 % if pdftex
  \usepackage[T1]{fontenc}
  \usepackage[utf8]{inputenc}
\else % if luatex or xelatex
  \ifxetex
    \usepackage{mathspec}
  \else
    \usepackage{fontspec}
  \fi
  \defaultfontfeatures{Ligatures=TeX,Scale=MatchLowercase}
\fi
% use upquote if available, for straight quotes in verbatim environments
\IfFileExists{upquote.sty}{\usepackage{upquote}}{}
% use microtype if available
\IfFileExists{microtype.sty}{%
\usepackage{microtype}
\UseMicrotypeSet[protrusion]{basicmath} % disable protrusion for tt fonts
}{}
\usepackage[margin=1in]{geometry}
\usepackage{hyperref}
\hypersetup{unicode=true,
            pdftitle={Appendix A: Code to reproduce the study},
            pdfborder={0 0 0},
            breaklinks=true}
\urlstyle{same}  % don't use monospace font for urls
\usepackage{color}
\usepackage{fancyvrb}
\newcommand{\VerbBar}{|}
\newcommand{\VERB}{\Verb[commandchars=\\\{\}]}
\DefineVerbatimEnvironment{Highlighting}{Verbatim}{commandchars=\\\{\}}
% Add ',fontsize=\small' for more characters per line
\usepackage{framed}
\definecolor{shadecolor}{RGB}{248,248,248}
\newenvironment{Shaded}{\begin{snugshade}}{\end{snugshade}}
\newcommand{\KeywordTok}[1]{\textcolor[rgb]{0.13,0.29,0.53}{\textbf{#1}}}
\newcommand{\DataTypeTok}[1]{\textcolor[rgb]{0.13,0.29,0.53}{#1}}
\newcommand{\DecValTok}[1]{\textcolor[rgb]{0.00,0.00,0.81}{#1}}
\newcommand{\BaseNTok}[1]{\textcolor[rgb]{0.00,0.00,0.81}{#1}}
\newcommand{\FloatTok}[1]{\textcolor[rgb]{0.00,0.00,0.81}{#1}}
\newcommand{\ConstantTok}[1]{\textcolor[rgb]{0.00,0.00,0.00}{#1}}
\newcommand{\CharTok}[1]{\textcolor[rgb]{0.31,0.60,0.02}{#1}}
\newcommand{\SpecialCharTok}[1]{\textcolor[rgb]{0.00,0.00,0.00}{#1}}
\newcommand{\StringTok}[1]{\textcolor[rgb]{0.31,0.60,0.02}{#1}}
\newcommand{\VerbatimStringTok}[1]{\textcolor[rgb]{0.31,0.60,0.02}{#1}}
\newcommand{\SpecialStringTok}[1]{\textcolor[rgb]{0.31,0.60,0.02}{#1}}
\newcommand{\ImportTok}[1]{#1}
\newcommand{\CommentTok}[1]{\textcolor[rgb]{0.56,0.35,0.01}{\textit{#1}}}
\newcommand{\DocumentationTok}[1]{\textcolor[rgb]{0.56,0.35,0.01}{\textbf{\textit{#1}}}}
\newcommand{\AnnotationTok}[1]{\textcolor[rgb]{0.56,0.35,0.01}{\textbf{\textit{#1}}}}
\newcommand{\CommentVarTok}[1]{\textcolor[rgb]{0.56,0.35,0.01}{\textbf{\textit{#1}}}}
\newcommand{\OtherTok}[1]{\textcolor[rgb]{0.56,0.35,0.01}{#1}}
\newcommand{\FunctionTok}[1]{\textcolor[rgb]{0.00,0.00,0.00}{#1}}
\newcommand{\VariableTok}[1]{\textcolor[rgb]{0.00,0.00,0.00}{#1}}
\newcommand{\ControlFlowTok}[1]{\textcolor[rgb]{0.13,0.29,0.53}{\textbf{#1}}}
\newcommand{\OperatorTok}[1]{\textcolor[rgb]{0.81,0.36,0.00}{\textbf{#1}}}
\newcommand{\BuiltInTok}[1]{#1}
\newcommand{\ExtensionTok}[1]{#1}
\newcommand{\PreprocessorTok}[1]{\textcolor[rgb]{0.56,0.35,0.01}{\textit{#1}}}
\newcommand{\AttributeTok}[1]{\textcolor[rgb]{0.77,0.63,0.00}{#1}}
\newcommand{\RegionMarkerTok}[1]{#1}
\newcommand{\InformationTok}[1]{\textcolor[rgb]{0.56,0.35,0.01}{\textbf{\textit{#1}}}}
\newcommand{\WarningTok}[1]{\textcolor[rgb]{0.56,0.35,0.01}{\textbf{\textit{#1}}}}
\newcommand{\AlertTok}[1]{\textcolor[rgb]{0.94,0.16,0.16}{#1}}
\newcommand{\ErrorTok}[1]{\textcolor[rgb]{0.64,0.00,0.00}{\textbf{#1}}}
\newcommand{\NormalTok}[1]{#1}
\usepackage{graphicx,grffile}
\makeatletter
\def\maxwidth{\ifdim\Gin@nat@width>\linewidth\linewidth\else\Gin@nat@width\fi}
\def\maxheight{\ifdim\Gin@nat@height>\textheight\textheight\else\Gin@nat@height\fi}
\makeatother
% Scale images if necessary, so that they will not overflow the page
% margins by default, and it is still possible to overwrite the defaults
% using explicit options in \includegraphics[width, height, ...]{}
\setkeys{Gin}{width=\maxwidth,height=\maxheight,keepaspectratio}
\IfFileExists{parskip.sty}{%
\usepackage{parskip}
}{% else
\setlength{\parindent}{0pt}
\setlength{\parskip}{6pt plus 2pt minus 1pt}
}
\setlength{\emergencystretch}{3em}  % prevent overfull lines
\providecommand{\tightlist}{%
  \setlength{\itemsep}{0pt}\setlength{\parskip}{0pt}}
\setcounter{secnumdepth}{5}
% Redefines (sub)paragraphs to behave more like sections
\ifx\paragraph\undefined\else
\let\oldparagraph\paragraph
\renewcommand{\paragraph}[1]{\oldparagraph{#1}\mbox{}}
\fi
\ifx\subparagraph\undefined\else
\let\oldsubparagraph\subparagraph
\renewcommand{\subparagraph}[1]{\oldsubparagraph{#1}\mbox{}}
\fi

%%% Use protect on footnotes to avoid problems with footnotes in titles
\let\rmarkdownfootnote\footnote%
\def\footnote{\protect\rmarkdownfootnote}

%%% Change title format to be more compact
\usepackage{titling}

% Create subtitle command for use in maketitle
\newcommand{\subtitle}[1]{
  \posttitle{
    \begin{center}\large#1\end{center}
    }
}

\setlength{\droptitle}{-2em}

  \title{Appendix A: Code to reproduce the study}
    \pretitle{\vspace{\droptitle}\centering\huge}
  \posttitle{\par}
    \author{}
    \preauthor{}\postauthor{}
    \date{}
    \predate{}\postdate{}
  

\begin{document}
\maketitle

\newcommand{\beginappendix}{%
  \setcounter{table}{0}
  \renewcommand{\thetable}{A\arabic{table}}%
  \setcounter{figure}{0}
  \renewcommand{\thefigure}{A\arabic{figure}}%
  \setcounter{section}{0}
  \renewcommand{\thesection}{A\arabic{section}}%
}

\beginappendix

The easiest way to reproduce this study is to download, clone, or fork
the GitHub repository at \texttt{github.com/ajrominger/paleo\_supStat}.
All scripts can then be run on new downloads of the PBDB and
modifications to the analysese can be made. The repository is organized
into directories \texttt{data} containing data and data-cleaning
scripts; \texttt{R} containing R functions for general use; and
\texttt{analysis} containing analysis scripts that use the data and R
functions. The GitHub repository also contains a manuscript directory
(\texttt{ms}), this document (\texttt{sstat\_notebook.Rmd}), and an R
script (\texttt{sstat\_make.R}) that calls each data cleaning and
analysis script in sequence to automatically reproduce the entire study.
The \texttt{data}, \texttt{R}, and \texttt{analysis} directories can
also be recreated from the scripts reproduced below.

The accompanying explainations below (organized by the flow of data
aquisition, cleaning, and then analysis) will help the user understand
the purpose of each script/function such that they can reproduce the
results, or modify the routine.

Lastly, this study depends on the contributed packages \emph{divDyn},
\emph{parallel}, and \emph{R.utils} which should be downloaded from
CRAN, and on a custom package \emph{socorro} which must be downloaded
from GitHub:

\begin{Shaded}
\begin{Highlighting}[]
\NormalTok{devtools}\OperatorTok{::}\KeywordTok{install_github}\NormalTok{(}\StringTok{'ajrominger/socorro'}\NormalTok{)}
\end{Highlighting}
\end{Shaded}

\section{Getting the data}\label{getting-the-data}

\subsection{PBDB API}\label{pbdb-api}

To obtain the PBDB data we make use of the API in script
\texttt{data/pbdb\_data\_get.R}, which accesses the API and cleans the
data by:

\begin{itemize}
\tightlist
\item
  removing poorly lithified specimens
\item
  removing collections at the basin scale
\item
  including only fine-scale stratigraphy (below the ``group'' level)
\item
  resolving taxonomy to the genus or subgenus level where availible
  (storing genus or subgenus as \texttt{otu})
\item
  combining multiple records of the same OTU per collection
\item
  importing standardized timebins from \url{fossilworks.org} (timebins
  are scraped with script \texttt{data/fossilworks\_tbins\_intervals.R})
\end{itemize}

The data gathering script \texttt{data/pbdb\_data\_get.R} is shown
below:

\begin{Shaded}
\begin{Highlighting}[]
\CommentTok{# **script to interface with PBDB API and clean resulting data**}

\CommentTok{# call to the API}
\NormalTok{show <-}\StringTok{ }\KeywordTok{paste0}\NormalTok{(}\KeywordTok{c}\NormalTok{(}\StringTok{'ident'}\NormalTok{, }\StringTok{'phylo'}\NormalTok{, }\StringTok{'lith'}\NormalTok{, }\StringTok{'loc'}\NormalTok{, }\StringTok{'time'}\NormalTok{, }\StringTok{'geo'}\NormalTok{, }\StringTok{'stratext'}\NormalTok{,}
                 \StringTok{'ecospace'}\NormalTok{),}
               \DataTypeTok{collapse =} \StringTok{','}\NormalTok{)}
\NormalTok{version <-}\StringTok{ '1.2'}
\NormalTok{base_name <-}\StringTok{ 'Animalia^Craniata'}
\NormalTok{min_ma <-}\StringTok{ }\DecValTok{0}
\NormalTok{max_ma <-}\StringTok{ }\DecValTok{560}
\NormalTok{timerule <-}\StringTok{ 'contain'}
\NormalTok{envtype <-}\StringTok{ 'marine'}

\CommentTok{# break-up backbone URI just so it can be nicely displayed}
\NormalTok{bbURI <-}\StringTok{ }\KeywordTok{paste0}\NormalTok{(}\StringTok{'https://paleobiodb.org/data%s/occs/list.csv?'}\NormalTok{,}
                \StringTok{'base_name=%s&show=%s&limit=all&min_ma=%s&max_ma=%s&'}\NormalTok{,}
                \StringTok{'timerule=%s&envtype=%s'}\NormalTok{)}

\CommentTok{# the actual call to the URI}
\NormalTok{uri <-}\StringTok{ }\KeywordTok{sprintf}\NormalTok{(bbURI,}
\NormalTok{               version,}
\NormalTok{               base_name,}
\NormalTok{               show,}
\NormalTok{               min_ma,}
\NormalTok{               max_ma,}
\NormalTok{               timerule,}
\NormalTok{               envtype)}

\CommentTok{# get pbdb occurences}
\NormalTok{x <-}\StringTok{ }\KeywordTok{read.csv}\NormalTok{(uri, }\DataTypeTok{as.is =} \OtherTok{TRUE}\NormalTok{)}

\CommentTok{# write out raw data}
\KeywordTok{write.csv}\NormalTok{(x, }\StringTok{'data/pbdb_data_raw.csv'}\NormalTok{, }\DataTypeTok{row.names =} \OtherTok{FALSE}\NormalTok{)}


\CommentTok{# clean up}

\CommentTok{# remove unnecceary columns}
\NormalTok{c2rm <-}\StringTok{ }\KeywordTok{c}\NormalTok{(}\StringTok{'record_type'}\NormalTok{, }\StringTok{'reid_no'}\NormalTok{, }\StringTok{'flags'}\NormalTok{, }\StringTok{'identified_name'}\NormalTok{,}
          \StringTok{'identified_rank'}\NormalTok{, }\StringTok{'identified_no'}\NormalTok{, }\StringTok{'difference'}\NormalTok{, }\StringTok{'species_name'}\NormalTok{,}
          \StringTok{'species_reso'}\NormalTok{, }\StringTok{'lithdescript'}\NormalTok{, }\StringTok{'lithology1'}\NormalTok{, }\StringTok{'minor_lithology1'}\NormalTok{,}
          \StringTok{'lithology2'}\NormalTok{, }\StringTok{'lithification2'}\NormalTok{, }\StringTok{'minor_lithology2'}\NormalTok{, }\StringTok{'cc'}\NormalTok{, }\StringTok{'state'}\NormalTok{,}
          \StringTok{'county'}\NormalTok{, }\StringTok{'latlng_basis'}\NormalTok{, }\StringTok{'geogcomments'}\NormalTok{, }\StringTok{'geology_comments'}\NormalTok{,}
          \StringTok{'zone'}\NormalTok{, }\StringTok{'localsection'}\NormalTok{, }\StringTok{'localbed'}\NormalTok{, }\StringTok{'localorder'}\NormalTok{,}
          \StringTok{'regionalsection'}\NormalTok{, }\StringTok{'regionalbed'}\NormalTok{, }\StringTok{'regionalorder'}\NormalTok{,}
          \StringTok{'stratcomments'}\NormalTok{)}
\NormalTok{x <-}\StringTok{ }\NormalTok{x[, }\OperatorTok{!}\NormalTok{(}\KeywordTok{names}\NormalTok{(x) }\OperatorTok\StringTok{ }\NormalTok{c2rm)]}

\CommentTok{# only well lithified specimens}
\NormalTok{x <-}\StringTok{ }\NormalTok{x[x}\OperatorTok{$}\NormalTok{lithification1 }\OperatorTok\StringTok{ }\KeywordTok{c}\NormalTok{(}\StringTok{''}\NormalTok{, }\StringTok{'lithified'}\NormalTok{), ]}


\CommentTok{# no basin-scale collections}
\NormalTok{x <-}\StringTok{ }\NormalTok{x[x}\OperatorTok{$}\NormalTok{geogscale }\OperatorTok{!=}\StringTok{ 'basin'}\NormalTok{, ]}

\CommentTok{# fine scale stratigraphy only}
\NormalTok{x <-}\StringTok{ }\NormalTok{x[}\OperatorTok{!}\NormalTok{(x}\OperatorTok{$}\NormalTok{stratscale }\OperatorTok\StringTok{ }\KeywordTok{c}\NormalTok{(}\StringTok{'group'}\NormalTok{, }\StringTok{'supergroup'}\NormalTok{)), ]}

\CommentTok{# resolve taxonomy to genus or subgenus where availible}
\NormalTok{otu <-}\StringTok{ }\NormalTok{x}\OperatorTok{$}\NormalTok{genus}
\NormalTok{otu[x}\OperatorTok{$}\NormalTok{subgenus_name }\OperatorTok{!=}\StringTok{ ''}\NormalTok{] <-}\StringTok{ }\KeywordTok{ifelse}\NormalTok{(x}\OperatorTok{$}\NormalTok{subgenus_reso[x}\OperatorTok{$}\NormalTok{subgenus_name }\OperatorTok{!=}\StringTok{ ''}\NormalTok{] }\OperatorTok{==}\StringTok{ ''}\NormalTok{,}
\NormalTok{                                     x}\OperatorTok{$}\NormalTok{subgenus_name[x}\OperatorTok{$}\NormalTok{subgenus_name }\OperatorTok{!=}\StringTok{ ''}\NormalTok{],}
\NormalTok{                                     otu[x}\OperatorTok{$}\NormalTok{subgenus_name }\OperatorTok{!=}\StringTok{ ''}\NormalTok{])}
\NormalTok{otu[x}\OperatorTok{$}\NormalTok{primary_reso }\OperatorTok{!=}\StringTok{ ''}\NormalTok{] <-}\StringTok{ ''}
\NormalTok{x}\OperatorTok{$}\NormalTok{otu <-}\StringTok{ }\NormalTok{otu}
\NormalTok{x <-}\StringTok{ }\NormalTok{x[x}\OperatorTok{$}\NormalTok{otu }\OperatorTok{!=}\StringTok{ ''}\NormalTok{, ]}


\CommentTok{# combine multiple records of same otu per collection}
\NormalTok{x <-}\StringTok{ }\NormalTok{x[}\OperatorTok{!}\KeywordTok{duplicated}\NormalTok{(x[, }\KeywordTok{c}\NormalTok{(}\StringTok{'collection_no'}\NormalTok{, }\StringTok{'otu'}\NormalTok{)]), ]}


\CommentTok{# standard time bins}
\NormalTok{stages <-}\StringTok{ }\KeywordTok{read.csv}\NormalTok{(}\StringTok{'data/tbins_stages.csv'}\NormalTok{, }\DataTypeTok{as.is =} \OtherTok{TRUE}\NormalTok{)}
\NormalTok{earlyTbin <-}\StringTok{ }\NormalTok{stages}\OperatorTok{$}\NormalTok{tbin[}\KeywordTok{match}\NormalTok{(x}\OperatorTok{$}\NormalTok{early_interval, stages}\OperatorTok{$}\NormalTok{name)]}
\NormalTok{lateTbin <-}\StringTok{ }\NormalTok{stages}\OperatorTok{$}\NormalTok{tbin[}\KeywordTok{match}\NormalTok{(x}\OperatorTok{$}\NormalTok{late_interval, stages}\OperatorTok{$}\NormalTok{name)]}
\NormalTok{lateTbin[}\KeywordTok{is.na}\NormalTok{(lateTbin)] <-}\StringTok{ }\NormalTok{earlyTbin[}\KeywordTok{is.na}\NormalTok{(lateTbin)]}
\NormalTok{earlyTbin[earlyTbin }\OperatorTok{!=}\StringTok{ }\NormalTok{lateTbin] <-}\StringTok{ }\OtherTok{NA}

\NormalTok{x}\OperatorTok{$}\NormalTok{tbin <-}\StringTok{ }\NormalTok{earlyTbin}
\NormalTok{x <-}\StringTok{ }\NormalTok{x[}\OperatorTok{!}\KeywordTok{is.na}\NormalTok{(x}\OperatorTok{$}\NormalTok{tbin), ]}


\CommentTok{# write out fully processed data}
\KeywordTok{write.csv}\NormalTok{(x, }\StringTok{'data/pbdb_data.csv'}\NormalTok{, }\DataTypeTok{row.names =} \OtherTok{FALSE}\NormalTok{)}
\end{Highlighting}
\end{Shaded}

\subsection{Scraping fossilworks}\label{scraping-fossilworks}

The script to pull Alory's time bins
(\texttt{data/fossilworks\_tbins\_intervals.R}) is below:

\begin{Shaded}
\begin{Highlighting}[]
\CommentTok{# **script to scrape time bins from fossilworks.org**}

\KeywordTok{options}\NormalTok{(}\DataTypeTok{stringsAsFactors =} \OtherTok{FALSE}\NormalTok{)}

\CommentTok{# loop through interval info on fossilworks.org}
\NormalTok{coreURI <-}\StringTok{ 'http://fossilworks.org/bridge.pl?a=displayInterval&interval_no='}

\NormalTok{tbinInfo <-}\StringTok{ }\KeywordTok{lapply}\NormalTok{(}\DecValTok{1}\OperatorTok{:}\DecValTok{1108}\NormalTok{, }\ControlFlowTok{function}\NormalTok{(i) \{}
    \KeywordTok{print}\NormalTok{(i)}
\NormalTok{    linfo <-}\StringTok{ }\KeywordTok{try}\NormalTok{(}\KeywordTok{readLines}\NormalTok{(}\KeywordTok{paste0}\NormalTok{(coreURI, i), }\DataTypeTok{n =} \DecValTok{150}\NormalTok{))}
    
    \ControlFlowTok{if}\NormalTok{(}\StringTok{'try-error'} \OperatorTok\StringTok{ }\KeywordTok{class}\NormalTok{(linfo)) }
\NormalTok{        linfo <-}\StringTok{ }\KeywordTok{try}\NormalTok{(}\KeywordTok{readLines}\NormalTok{(}\KeywordTok{paste0}\NormalTok{(coreURI, i), }\DataTypeTok{n =} \DecValTok{150}\NormalTok{))}
    
    \ControlFlowTok{if}\NormalTok{(}\StringTok{'try-error'} \OperatorTok\StringTok{ }\KeywordTok{class}\NormalTok{(linfo)) \{}
\NormalTok{        thisTbin <-}\StringTok{ }\NormalTok{thisMax <-}\StringTok{ }\NormalTok{thisMin <-}\StringTok{ }\NormalTok{thisName <-}\StringTok{ }\OtherTok{NA}
\NormalTok{    \} }\ControlFlowTok{else}\NormalTok{ \{}
\NormalTok{        thisTbin <-}\StringTok{ }\KeywordTok{gsub}\NormalTok{(}\StringTok{'^.*10 million year bin: |<br>.*$'}\NormalTok{, }\StringTok{''}\NormalTok{, }
\NormalTok{                         linfo[}\KeywordTok{grep}\NormalTok{(}\StringTok{'10 million year bin'}\NormalTok{, linfo)])}
\NormalTok{        thisMax <-}\StringTok{ }\KeywordTok{as.numeric}\NormalTok{(}\KeywordTok{gsub}\NormalTok{(}\StringTok{'^.*Lower boundary: equal to | Ma.*$|[^0-9}\CharTok{\textbackslash{}\textbackslash{}}\StringTok{.]'}\NormalTok{, }\StringTok{''}\NormalTok{,}
\NormalTok{                                   linfo[}\KeywordTok{grep}\NormalTok{(}\StringTok{'Lower boundary: equal to'}\NormalTok{, linfo)]))}
\NormalTok{        thisMin <-}\StringTok{ }\KeywordTok{as.numeric}\NormalTok{(}\KeywordTok{gsub}\NormalTok{(}\StringTok{'^.*Upper boundary: equal to | Ma.*$|[^0-9}\CharTok{\textbackslash{}\textbackslash{}}\StringTok{.]'}\NormalTok{, }\StringTok{''}\NormalTok{,}
\NormalTok{                                   linfo[}\KeywordTok{grep}\NormalTok{(}\StringTok{'Upper boundary: equal to'}\NormalTok{, linfo)]))}
\NormalTok{        thisName <-}\StringTok{ }\KeywordTok{gsub}\NormalTok{(}\StringTok{'^.*<p class="pageTitle">|</p>.*$'}\NormalTok{, }\StringTok{''}\NormalTok{, }
\NormalTok{                         linfo[}\KeywordTok{grep}\NormalTok{(}\StringTok{'class="pageTitle"'}\NormalTok{, linfo)])}
\NormalTok{    \}}
    
    \KeywordTok{return}\NormalTok{(}\KeywordTok{data.frame}\NormalTok{(}\DataTypeTok{name =} \KeywordTok{ifelse}\NormalTok{(}\KeywordTok{length}\NormalTok{(thisName) }\OperatorTok{==}\StringTok{ }\DecValTok{0}\NormalTok{, }\OtherTok{NA}\NormalTok{, thisName),}
                      \DataTypeTok{tbin =} \KeywordTok{ifelse}\NormalTok{(}\KeywordTok{length}\NormalTok{(thisTbin) }\OperatorTok{==}\StringTok{ }\DecValTok{0}\NormalTok{, }\OtherTok{NA}\NormalTok{, thisTbin), }
                      \DataTypeTok{ma_min =} \KeywordTok{ifelse}\NormalTok{(}\KeywordTok{length}\NormalTok{(thisMin) }\OperatorTok{==}\StringTok{ }\DecValTok{0}\NormalTok{, }\OtherTok{NA}\NormalTok{, thisMin),}
                      \DataTypeTok{ma_max =} \KeywordTok{ifelse}\NormalTok{(}\KeywordTok{length}\NormalTok{(thisMax) }\OperatorTok{==}\StringTok{ }\DecValTok{0}\NormalTok{, }\OtherTok{NA}\NormalTok{, thisMax)))}
\NormalTok{\})}

\NormalTok{tbinInfo <-}\StringTok{ }\KeywordTok{do.call}\NormalTok{(rbind, tbinInfo)}


\CommentTok{# clean up}
\CommentTok{# --------}

\NormalTok{tbinInfo <-}\StringTok{ }\NormalTok{tbinInfo[}\OperatorTok{!}\KeywordTok{is.na}\NormalTok{(tbinInfo}\OperatorTok{$}\NormalTok{name), ]}

\CommentTok{# remove 'stage' and equivilant from name}
\NormalTok{tbinInfo}\OperatorTok{$}\NormalTok{name <-}\StringTok{ }\KeywordTok{gsub}\NormalTok{(}\StringTok{' [[:lower:]].*'}\NormalTok{, }\StringTok{''}\NormalTok{, tbinInfo}\OperatorTok{$}\NormalTok{name)}

\CommentTok{# split up stages with a '/' into both names}
\NormalTok{temp <-}\StringTok{ }\NormalTok{tbinInfo[}\KeywordTok{grep}\NormalTok{(}\StringTok{'/'}\NormalTok{, tbinInfo}\OperatorTok{$}\NormalTok{name), ]}
\NormalTok{tbinInfo}\OperatorTok{$}\NormalTok{name <-}\StringTok{ }\KeywordTok{gsub}\NormalTok{(}\StringTok{'.*/'}\NormalTok{, }\StringTok{''}\NormalTok{, tbinInfo}\OperatorTok{$}\NormalTok{name)}
\NormalTok{temp}\OperatorTok{$}\NormalTok{name <-}\StringTok{ }\KeywordTok{gsub}\NormalTok{(}\StringTok{'/.* '}\NormalTok{, }\StringTok{' '}\NormalTok{, temp}\OperatorTok{$}\NormalTok{name)}
\NormalTok{tbinInfo <-}\StringTok{ }\KeywordTok{rbind}\NormalTok{(tbinInfo, temp)}

\CommentTok{# fix random typo}
\NormalTok{tbinInfo}\OperatorTok{$}\NormalTok{name[tbinInfo}\OperatorTok{$}\NormalTok{name }\OperatorTok{==}\StringTok{ 'Cazenovia'}\NormalTok{] <-}\StringTok{ 'Cazenovian'}


\CommentTok{# write out}
\KeywordTok{write.csv}\NormalTok{(tbinInfo, }\StringTok{'data/tbins_stages.csv'}\NormalTok{, }\DataTypeTok{row.names =} \OtherTok{FALSE}\NormalTok{)}


\CommentTok{# also write out summary of each time bin, most importantly (for plottin) }
\CommentTok{# its midpoint}

\NormalTok{tbinmid <-}\StringTok{ }\KeywordTok{sapply}\NormalTok{(}\KeywordTok{unique}\NormalTok{(tbinInfo}\OperatorTok{$}\NormalTok{tbin[}\OperatorTok{!}\KeywordTok{is.na}\NormalTok{(tbinInfo}\OperatorTok{$}\NormalTok{tbin)]), }\ControlFlowTok{function}\NormalTok{(tbin) \{}
\NormalTok{    tt <-}\StringTok{ }\KeywordTok{unlist}\NormalTok{(tbinInfo[tbinInfo}\OperatorTok{$}\NormalTok{tbin }\OperatorTok{==}\StringTok{ }\NormalTok{tbin, }\KeywordTok{c}\NormalTok{(}\StringTok{'ma_min'}\NormalTok{, }\StringTok{'ma_max'}\NormalTok{)])}
    \KeywordTok{return}\NormalTok{(}\KeywordTok{mean}\NormalTok{(}\KeywordTok{range}\NormalTok{(tt, }\DataTypeTok{na.rm =} \OtherTok{TRUE}\NormalTok{)))}
\NormalTok{\})}

\NormalTok{tbinmid <-}\StringTok{ }\KeywordTok{sort}\NormalTok{(tbinmid, }\DataTypeTok{decreasing =} \OtherTok{TRUE}\NormalTok{)}

\KeywordTok{write.csv}\NormalTok{(}\KeywordTok{data.frame}\NormalTok{(}\DataTypeTok{tbin =} \KeywordTok{names}\NormalTok{(tbinmid), }\DataTypeTok{ma_mid =} \KeywordTok{as.numeric}\NormalTok{(tbinmid)), }
          \StringTok{'data/tbinsMid.csv'}\NormalTok{, }\DataTypeTok{row.names =} \OtherTok{FALSE}\NormalTok{)}


\CommentTok{# lastly confirm the durations of tbins}
\KeywordTok{length}\NormalTok{(tbinmid)}

\NormalTok{tbinrange <-}\StringTok{ }\KeywordTok{sapply}\NormalTok{(}\KeywordTok{unique}\NormalTok{(tbinInfo}\OperatorTok{$}\NormalTok{tbin[}\OperatorTok{!}\KeywordTok{is.na}\NormalTok{(tbinInfo}\OperatorTok{$}\NormalTok{tbin)]), }\ControlFlowTok{function}\NormalTok{(tbin) \{}
\NormalTok{    tt <-}\StringTok{ }\KeywordTok{unlist}\NormalTok{(tbinInfo[tbinInfo}\OperatorTok{$}\NormalTok{tbin }\OperatorTok{==}\StringTok{ }\NormalTok{tbin, }\KeywordTok{c}\NormalTok{(}\StringTok{'ma_min'}\NormalTok{, }\StringTok{'ma_max'}\NormalTok{)])}
    \KeywordTok{return}\NormalTok{(}\KeywordTok{diff}\NormalTok{(}\KeywordTok{range}\NormalTok{(tt, }\DataTypeTok{na.rm =} \OtherTok{TRUE}\NormalTok{)))}
\NormalTok{\})}

\KeywordTok{mean}\NormalTok{(tbinrange)}
\end{Highlighting}
\end{Shaded}

\section{Three timer and publication bias
correction}\label{three-timer-and-publication-bias-correction}

Once the data have been downloaded and cleaned, we correct for
incomplete and biased sampling with the script
\texttt{data/pbdb\_3TPub\_make.R} which sources the function
\texttt{R/make3TPub.R} to generate the main output: a matrix with time
bins as rows, taxa (families in this case) as columns and bias-corrected
richness as cells.

\begin{Shaded}
\begin{Highlighting}[]
\CommentTok{# **this script produces diversity estimates per family per time bin from PBDB data}
\CommentTok{# corrected by the '3 timers method' and for possible publication bias**}

\CommentTok{# source function to produce a matrix of time by taxon with cells}
\CommentTok{# of corrected diversity}
\KeywordTok{source}\NormalTok{(}\StringTok{'R/make3TPub.R'}\NormalTok{)}

\CommentTok{# load other needed funcitons}

\CommentTok{# source('code/sstat_comp.R')}
\CommentTok{# source('code/sstat_methods.R')}
\CommentTok{# source('code/Px_gam.R')}

\CommentTok{# load and prepare data}
\CommentTok{# ---------------------}

\NormalTok{pbdbDat <-}\StringTok{ }\KeywordTok{read.csv}\NormalTok{(}\StringTok{'data/pbdb_data.csv'}\NormalTok{, }\DataTypeTok{as.is =} \OtherTok{TRUE}\NormalTok{)}

\CommentTok{# make column for midpoint ma}
\NormalTok{pbdbDat}\OperatorTok{$}\NormalTok{ma_mid <-}\StringTok{ }\NormalTok{(pbdbDat}\OperatorTok{$}\NormalTok{max_ma }\OperatorTok{+}\StringTok{ }\NormalTok{pbdbDat}\OperatorTok{$}\NormalTok{min_ma) }\OperatorTok{/}\StringTok{ }\DecValTok{2}

\CommentTok{# get rid of poor temporal resolution}
\NormalTok{pbdbDat <-}\StringTok{ }\NormalTok{pbdbDat[pbdbDat}\OperatorTok{$}\NormalTok{tbin }\OperatorTok{!=}\StringTok{ ''}\NormalTok{, ]}

\CommentTok{# get rid of bad taxonomy}
\NormalTok{pbdbDat <-}\StringTok{ }\NormalTok{pbdbDat[pbdbDat}\OperatorTok{$}\NormalTok{family }\OperatorTok{!=}\StringTok{ ''}\NormalTok{, ]}
\NormalTok{pbdbDat <-}\StringTok{ }\NormalTok{pbdbDat[pbdbDat}\OperatorTok{$}\NormalTok{otu }\OperatorTok{!=}\StringTok{ ''}\NormalTok{, ]}

\CommentTok{# get bin times}
\NormalTok{pbdbDat}\OperatorTok{$}\NormalTok{mid_ma <-}\StringTok{ }\NormalTok{(pbdbDat}\OperatorTok{$}\NormalTok{min_ma }\OperatorTok{+}\StringTok{ }\NormalTok{pbdbDat}\OperatorTok{$}\NormalTok{max_ma) }\OperatorTok{/}\StringTok{ }\DecValTok{2}
\NormalTok{pbdbTime <-}\StringTok{ }\KeywordTok{sort}\NormalTok{(}\KeywordTok{tapply}\NormalTok{(pbdbDat}\OperatorTok{$}\NormalTok{mid_ma, pbdbDat}\OperatorTok{$}\NormalTok{tbin, mean))}
\NormalTok{pbdbDat}\OperatorTok{$}\NormalTok{tbin <-}\StringTok{ }\KeywordTok{factor}\NormalTok{(pbdbDat}\OperatorTok{$}\NormalTok{tbin, }\DataTypeTok{levels =} \KeywordTok{names}\NormalTok{(pbdbTime))}


\CommentTok{# data.frame to hold publication, diversity and 3T stat}
\NormalTok{famTbinBias <-}\StringTok{ }\KeywordTok{aggregate}\NormalTok{(}\KeywordTok{list}\NormalTok{(}\DataTypeTok{div =}\NormalTok{ pbdbDat}\OperatorTok{$}\NormalTok{otu), }\KeywordTok{list}\NormalTok{(}\DataTypeTok{fam =}\NormalTok{ pbdbDat}\OperatorTok{$}\NormalTok{family,}
                                                       \DataTypeTok{tbin =}\NormalTok{ pbdbDat}\OperatorTok{$}\NormalTok{tbin),}
                         \ControlFlowTok{function}\NormalTok{(x) }\KeywordTok{length}\NormalTok{(}\KeywordTok{unique}\NormalTok{(x)))}


\CommentTok{# three timer stat and publication bias}
\CommentTok{# -------------------------------------}

\CommentTok{# matrix to determine three timers and part timers (sensu alroy 2008)}
\NormalTok{mt <-}\StringTok{ }\KeywordTok{matrix}\NormalTok{(}\DecValTok{0}\NormalTok{, }\DataTypeTok{nrow =} \KeywordTok{nlevels}\NormalTok{(pbdbDat}\OperatorTok{$}\NormalTok{tbin), }
             \DataTypeTok{ncol =} \KeywordTok{nlevels}\NormalTok{(pbdbDat}\OperatorTok{$}\NormalTok{tbin))}
\KeywordTok{diag}\NormalTok{(mt) <-}\StringTok{ }\OperatorTok{-}\DecValTok{10}
\NormalTok{mt[}\KeywordTok{abs}\NormalTok{(}\KeywordTok{row}\NormalTok{(mt) }\OperatorTok{-}\StringTok{ }\KeywordTok{col}\NormalTok{(mt)) }\OperatorTok{==}\StringTok{ }\DecValTok{1}\NormalTok{] <-}\StringTok{ }\DecValTok{1}

\CommentTok{# loop through and compute three timers and part timers}
\NormalTok{timers <-}\StringTok{ }\KeywordTok{lapply}\NormalTok{(}\KeywordTok{split}\NormalTok{(pbdbDat}\OperatorTok{$}\NormalTok{tbin, pbdbDat}\OperatorTok{$}\NormalTok{otu), }
                 \ControlFlowTok{function}\NormalTok{(x) \{}
                     \CommentTok{# browser()}
\NormalTok{                     tbins <-}\StringTok{ }\KeywordTok{integer}\NormalTok{(}\KeywordTok{nlevels}\NormalTok{(x))}
\NormalTok{                     tbins[}\KeywordTok{as.integer}\NormalTok{(}\KeywordTok{unique}\NormalTok{(x))] <-}\StringTok{ }\DecValTok{1}
\NormalTok{                     t3 <-}\StringTok{ }\KeywordTok{as.integer}\NormalTok{(mt }\OperatorTok\StringTok{ }\NormalTok{tbins }\OperatorTok{==}\StringTok{ }\DecValTok{2}\NormalTok{)}
\NormalTok{                     tp <-}\StringTok{ }\KeywordTok{as.integer}\NormalTok{(mt }\OperatorTok\StringTok{ }\NormalTok{tbins }\OperatorTok{==}\StringTok{ }\OperatorTok{-}\DecValTok{8}\NormalTok{)}
                     
                     \KeywordTok{return}\NormalTok{(}\KeywordTok{cbind}\NormalTok{(t3, tp))}
\NormalTok{                 \})}

\CommentTok{# compute 3 timer stat from 3 timers and part timers}
\NormalTok{timers <-}\StringTok{ }\KeywordTok{array}\NormalTok{(}\KeywordTok{unlist}\NormalTok{(timers), }\DataTypeTok{dim =} \KeywordTok{c}\NormalTok{(}\KeywordTok{nrow}\NormalTok{(timers[[}\DecValTok{1}\NormalTok{]]), }\DecValTok{2}\NormalTok{, }\KeywordTok{length}\NormalTok{(timers)))}
\NormalTok{t3stat <-}\StringTok{ }\DecValTok{1} \OperatorTok{-}\StringTok{ }\KeywordTok{rowSums}\NormalTok{(timers[, }\DecValTok{1}\NormalTok{, ]) }\OperatorTok{/}\StringTok{ }\NormalTok{(}\KeywordTok{rowSums}\NormalTok{(timers[, }\DecValTok{1}\NormalTok{, ]) }\OperatorTok{+}\StringTok{ }\KeywordTok{rowSums}\NormalTok{(timers[, }\DecValTok{2}\NormalTok{, ]))}

\CommentTok{# add to data.frame holding all info to be saved}
\NormalTok{famTbinBias}\OperatorTok{$}\NormalTok{T3Stat <-}\StringTok{ }\NormalTok{t3stat[}\KeywordTok{match}\NormalTok{(famTbinBias}\OperatorTok{$}\NormalTok{tbin, }
                                      \KeywordTok{levels}\NormalTok{(pbdbDat}\OperatorTok{$}\NormalTok{tbin))]}
\NormalTok{famTbinBias}\OperatorTok{$}\NormalTok{T3Div <-}\StringTok{ }\NormalTok{famTbinBias}\OperatorTok{$}\NormalTok{div }\OperatorTok{/}\StringTok{ }\NormalTok{famTbinBias}\OperatorTok{$}\NormalTok{T3Stat}

\CommentTok{# record pubs per tbin}
\NormalTok{tbinPub <-}\StringTok{ }\KeywordTok{tapply}\NormalTok{(pbdbDat}\OperatorTok{$}\NormalTok{reference_no, pbdbDat}\OperatorTok{$}\NormalTok{tbin,}
                   \ControlFlowTok{function}\NormalTok{(x) }\KeywordTok{length}\NormalTok{(}\KeywordTok{unique}\NormalTok{(x)))}
\NormalTok{famTbinBias}\OperatorTok{$}\NormalTok{tbinPub <-}\StringTok{ }\NormalTok{tbinPub[famTbinBias}\OperatorTok{$}\NormalTok{tbin]}

\CommentTok{# calculate corrected diversity}
\KeywordTok{pdf}\NormalTok{(}\StringTok{'ms/figSupp_divByPub_foo.pdf'}\NormalTok{, }\DataTypeTok{width =} \DecValTok{4}\NormalTok{, }\DataTypeTok{height =} \DecValTok{4}\NormalTok{)}
\NormalTok{pbdbFamDiv <-}\StringTok{ }\KeywordTok{with}\NormalTok{(famTbinBias,}
                   \KeywordTok{make3TPub}\NormalTok{(div, T3Stat, tbinPub, fam, tbin, pbdbTime, }
                             \DataTypeTok{minPub =} \DecValTok{10}\NormalTok{, }\DataTypeTok{plotit =} \OtherTok{TRUE}\NormalTok{))}
\KeywordTok{dev.off}\NormalTok{()}

\CommentTok{# write out corrected diversity}
\KeywordTok{write.csv}\NormalTok{(pbdbFamDiv, }\StringTok{'data/pbdb_3TPub-corrected.csv'}\NormalTok{)}


\CommentTok{# for permutational d-stat tests we need diversity at the genus level;}
\CommentTok{# make that here}

\CommentTok{# a data.frame holding only one record per genus per family per time bin}
\NormalTok{pbdbOcc <-}\StringTok{ }\NormalTok{pbdbDat[}\OperatorTok{!}\KeywordTok{duplicated}\NormalTok{(pbdbDat[, }\KeywordTok{c}\NormalTok{(}\StringTok{'tbin'}\NormalTok{, }\StringTok{'family'}\NormalTok{, }\StringTok{'otu'}\NormalTok{)]), ]}

\NormalTok{genTbinBias <-}\StringTok{ }\NormalTok{parallel}\OperatorTok{::}\KeywordTok{mclapply}\NormalTok{(}\KeywordTok{which}\NormalTok{(}\OperatorTok{!}\KeywordTok{is.nan}\NormalTok{(famTbinBias}\OperatorTok{$}\NormalTok{T3Stat)), }\DataTypeTok{mc.cores =} \DecValTok{3}\NormalTok{, }
                                  \DataTypeTok{FUN =} \ControlFlowTok{function}\NormalTok{(i) \{}
\NormalTok{                                      dat <-}\StringTok{ }\NormalTok{pbdbOcc[pbdbOcc}\OperatorTok{$}\NormalTok{family }\OperatorTok{==}\StringTok{ }\NormalTok{famTbinBias}\OperatorTok{$}\NormalTok{fam[i] }\OperatorTok{&}\StringTok{ }
\StringTok{                                                         }\NormalTok{pbdbOcc}\OperatorTok{$}\NormalTok{tbin }\OperatorTok{==}\StringTok{ }\NormalTok{famTbinBias}\OperatorTok{$}\NormalTok{tbin[i], }
                                                     \KeywordTok{c}\NormalTok{(}\StringTok{'tbin'}\NormalTok{, }\StringTok{'family'}\NormalTok{, }\StringTok{'otu'}\NormalTok{)]}
\NormalTok{                                      dat}\OperatorTok{$}\NormalTok{T3Occ <-}\StringTok{ }\DecValTok{1} \OperatorTok{/}\StringTok{ }\NormalTok{famTbinBias}\OperatorTok{$}\NormalTok{T3Stat[i]}
\NormalTok{                                      dat}\OperatorTok{$}\NormalTok{tbinPub <-}\StringTok{ }\NormalTok{famTbinBias}\OperatorTok{$}\NormalTok{tbinPub[i]}
                                      
                                      \KeywordTok{return}\NormalTok{(dat)}
\NormalTok{                                  \}}
\NormalTok{)}

\NormalTok{genTbinBias <-}\StringTok{ }\KeywordTok{do.call}\NormalTok{(rbind, genTbinBias)}
\NormalTok{pbdbGenDiv <-}\StringTok{ }\KeywordTok{data.frame}\NormalTok{(genTbinBias[, }\KeywordTok{c}\NormalTok{(}\StringTok{'tbin'}\NormalTok{, }\StringTok{'family'}\NormalTok{, }\StringTok{'otu'}\NormalTok{)], }
                         \DataTypeTok{T3PubDiv =}\NormalTok{ genTbinBias}\OperatorTok{$}\NormalTok{T3Occ }\OperatorTok{/}\StringTok{ }
\StringTok{                             }\KeywordTok{exp}\NormalTok{(}\KeywordTok{predict}\NormalTok{(pbdbPubLM, }
                                         \DataTypeTok{newdata =} \KeywordTok{data.frame}\NormalTok{(}
                                             \DataTypeTok{logPub =} \KeywordTok{log}\NormalTok{(genTbinBias}\OperatorTok{$}\NormalTok{tbinPub)))))}

\CommentTok{# write it out as a tidy data frame (not turned into a matrix) this will be easier}
\CommentTok{# for permuting}
\KeywordTok{write.csv}\NormalTok{(pbdbGenDiv, }\DataTypeTok{file =} \StringTok{'data/pbdb_3TPub_genera.csv'}\NormalTok{, }\DataTypeTok{row.names =} \OtherTok{FALSE}\NormalTok{)}
\end{Highlighting}
\end{Shaded}

Here is the guts of the \texttt{make3TPub} function

\begin{Shaded}
\begin{Highlighting}[]
\CommentTok{#' @description function to produce a matrix of time by taxa with cells of corrected diversity}
\CommentTok{#' @param rawDiv the raw diversity of each taxon in each time interval}
\CommentTok{#' @param t3stat the 3 timer stat for each diversity record}
\CommentTok{#' @param pub the number of publications associated with each diversity record}
\CommentTok{#' @param taxa the taxon names for each diversity record}
\CommentTok{#' @param tbin the time interval of each diversity record}
\CommentTok{#' @param tbinTime times associated with each `tbin`}
\CommentTok{#' @param minPub minimum number of publications for inclusion in regression analysis}
\CommentTok{#' @param plotit logical, should plot of taxon richness versus number of publications be made}
\CommentTok{#' @return a matrix with rows corresponding to time intervals and columns to the given taxa}
\CommentTok{#' each cell in the matrix represents corrected taxon richness}


\NormalTok{make3TPub <-}\StringTok{ }\ControlFlowTok{function}\NormalTok{(rawDiv,  t3stat,  pub,  taxa,  tbin,  tbinTime,  }
                      \DataTypeTok{minPub =} \DecValTok{10}\NormalTok{,  }\DataTypeTok{plotit =} \OtherTok{FALSE}\NormalTok{) \{}
    \CommentTok{# put data together so can be universally manipulated}
\NormalTok{    x <-}\StringTok{ }\KeywordTok{data.frame}\NormalTok{(}\DataTypeTok{rawDiv =}\NormalTok{ rawDiv, }\DataTypeTok{t3stat =}\NormalTok{ t3stat, }\DataTypeTok{pub =}\NormalTok{ pub, }\DataTypeTok{taxa =}\NormalTok{ taxa, }\DataTypeTok{tbin =}\NormalTok{ tbin)}
\NormalTok{    x}\OperatorTok{$}\NormalTok{tbin <-}\StringTok{ }\KeywordTok{as.character}\NormalTok{(x}\OperatorTok{$}\NormalTok{tbin)}
\NormalTok{    x}\OperatorTok{$}\NormalTok{taxa <-}\StringTok{ }\KeywordTok{as.character}\NormalTok{(x}\OperatorTok{$}\NormalTok{taxa)}
    
\NormalTok{    x <-}\StringTok{ }\NormalTok{x[}\OperatorTok{!}\KeywordTok{is.na}\NormalTok{(t3stat) }\OperatorTok{&}\StringTok{ }\NormalTok{pub }\OperatorTok{>=}\StringTok{  }\NormalTok{minPub, ]}
    
\NormalTok{    tbinTime <-}\StringTok{ }\NormalTok{tbinTime[}\KeywordTok{names}\NormalTok{(tbinTime) }\OperatorTok\StringTok{ }\NormalTok{x}\OperatorTok{$}\NormalTok{tbin]}
    
    \CommentTok{# 3-timer correction}
\NormalTok{    t3cor <-}\StringTok{ }\NormalTok{x}\OperatorTok{$}\NormalTok{rawDiv}\OperatorTok{/}\NormalTok{x}\OperatorTok{$}\NormalTok{t3stat}
    
    \CommentTok{# publication correction}
\NormalTok{    logPub <-}\StringTok{ }\KeywordTok{log}\NormalTok{(x}\OperatorTok{$}\NormalTok{pub)}
\NormalTok{    pubLM <-}\StringTok{ }\KeywordTok{lm}\NormalTok{(}\KeywordTok{log}\NormalTok{(t3cor)}\OperatorTok{~}\NormalTok{logPub)}
\NormalTok{    pbdbPubLM <<-}\StringTok{ }\NormalTok{pubLM }\CommentTok{# save regression to global env}
    
\NormalTok{    pubResid <-}\StringTok{ }\KeywordTok{exp}\NormalTok{(pubLM}\OperatorTok{$}\NormalTok{residuals)}
    
    \CommentTok{# plot so you can verify cuttoff etc.}
    \ControlFlowTok{if}\NormalTok{(plotit) \{}
        \KeywordTok{plot}\NormalTok{(}\KeywordTok{log}\NormalTok{(x}\OperatorTok{$}\NormalTok{pub), }\KeywordTok{log}\NormalTok{(t3cor),  }
             \DataTypeTok{xlab =} \StringTok{'log(Number of publications)'}\NormalTok{, }
             \DataTypeTok{ylab =} \StringTok{'log(3T-corrected number of genera)'}\NormalTok{)}
        \KeywordTok{abline}\NormalTok{(pubLM, }\DataTypeTok{col =} \StringTok{'red'}\NormalTok{)}
\NormalTok{    \}}
    
\NormalTok{    tbinTaxa <-}\StringTok{ }\NormalTok{socorro}\OperatorTok{::}\KeywordTok{tidy2mat}\NormalTok{(x}\OperatorTok{$}\NormalTok{tbin, x}\OperatorTok{$}\NormalTok{taxa, pubResid)}

    \KeywordTok{return}\NormalTok{(tbinTaxa[}\KeywordTok{names}\NormalTok{(}\KeywordTok{sort}\NormalTok{(tbinTime, }\DataTypeTok{decreasing =} \OtherTok{TRUE}\NormalTok{)), ])}
\NormalTok{\}}
\end{Highlighting}
\end{Shaded}

Our publication correction is simple: we take the exponential of the
residual of this relationship: \[
\text{log(3T-corrected richness)} = \beta_0 + \beta_1 \text{log(number of publications)} + \epsilon
\] The exponentiated residual amounts to dividing the three-timer
corrected richness by a publication correction factor:
\(\text{(3T-corrected richness)} / e^{\hat{y}}\) where \(\hat{y}\) is
the predicted trend line from the above relationship.

So we can use this multiplicative publication correction factor in
addition to the similarely multiplicative three-timer correction to
bias-correct individual genus-level occurrences. This is important when
we permute these bias-corrected genera across families to create a null
set of d-statistics for our superstatistical fit.

We can check our correction against other popular methods. In the script
\texttt{analysis/pbdb\_divCurve.R} we specifically compare simple
rarifaction, with the SQS method, with our three-time and publication
bias correction methods. All these various methods have close agreement.
The script \texttt{analysis/pbdb\_divCurve.R} is shown below:

\begin{Shaded}
\begin{Highlighting}[]
\CommentTok{# **a script to compare our 3TPub curve to other estimates of richness through }
\CommentTok{# the Phanerozoic**}

\CommentTok{# package with diversity dynamics subsampling functions}
\KeywordTok{library}\NormalTok{(divDyn)}

\CommentTok{# package for plotting}
\KeywordTok{library}\NormalTok{(socorro)}

\CommentTok{# load and prep data}
\NormalTok{pbdbFamDiv <-}\StringTok{ }\KeywordTok{read.csv}\NormalTok{(}\StringTok{'data/pbdb_3TPub-corrected.csv'}\NormalTok{, }\DataTypeTok{row.names =} \DecValTok{1}\NormalTok{)}
\NormalTok{pbdbDat <-}\StringTok{ }\KeywordTok{read.csv}\NormalTok{(}\StringTok{'data/pbdb_data.csv'}\NormalTok{, }\DataTypeTok{as.is =} \OtherTok{TRUE}\NormalTok{)}
\NormalTok{tbin <-}\StringTok{ }\KeywordTok{read.csv}\NormalTok{(}\StringTok{'data/tbinsMid.csv'}\NormalTok{, }\DataTypeTok{as.is =} \OtherTok{TRUE}\NormalTok{)}
\NormalTok{tbin}\OperatorTok{$}\NormalTok{tbin <-}\StringTok{ }\KeywordTok{factor}\NormalTok{(tbin}\OperatorTok{$}\NormalTok{tbin, }\DataTypeTok{levels =}\NormalTok{ tbin}\OperatorTok{$}\NormalTok{tbin)}
\NormalTok{pbdbDat}\OperatorTok{$}\NormalTok{tbin <-}\StringTok{ }\KeywordTok{factor}\NormalTok{(pbdbDat}\OperatorTok{$}\NormalTok{tbin, }\DataTypeTok{levels =} \KeywordTok{levels}\NormalTok{(tbin}\OperatorTok{$}\NormalTok{tbin))}
\NormalTok{pbdbDat}\OperatorTok{$}\NormalTok{tbinNum <-}\StringTok{ }\KeywordTok{as.integer}\NormalTok{(pbdbDat}\OperatorTok{$}\NormalTok{tbin)}

\NormalTok{pbdbDatUnique <-}\StringTok{ }\NormalTok{pbdbDat[}\OperatorTok{!}\KeywordTok{duplicated}\NormalTok{(}\KeywordTok{paste0}\NormalTok{(pbdbDat}\OperatorTok{$}\NormalTok{collection_no, pbdbDat}\OperatorTok{$}\NormalTok{otu)), ]}

\CommentTok{# subsampled richness}
\NormalTok{pbdbCR <-}\StringTok{ }\KeywordTok{subsample}\NormalTok{(pbdbDatUnique, }\DataTypeTok{bin =} \StringTok{'tbinNum'}\NormalTok{, }\DataTypeTok{tax =} \StringTok{'otu'}\NormalTok{, }\DataTypeTok{iter =} \DecValTok{50}\NormalTok{, }\DataTypeTok{q =} \DecValTok{120}\NormalTok{, }
                    \DataTypeTok{type =} \StringTok{'cr'}\NormalTok{, }\DataTypeTok{unit =} \StringTok{'reference_no'}\NormalTok{)}
\NormalTok{pbdbSQS <-}\StringTok{ }\KeywordTok{subsample}\NormalTok{(pbdbDatUnique, }\DataTypeTok{bin =} \StringTok{'tbinNum'}\NormalTok{, }\DataTypeTok{tax =} \StringTok{'otu'}\NormalTok{, }\DataTypeTok{iter =} \DecValTok{50}\NormalTok{, }\DataTypeTok{q =} \FloatTok{0.75}\NormalTok{, }
                     \DataTypeTok{ref =} \StringTok{'reference_no'}\NormalTok{, }\DataTypeTok{type =} \StringTok{'sqs'}\NormalTok{, }\DataTypeTok{singleton =} \StringTok{'ref'}\NormalTok{)}

\CommentTok{# our new richness estimate}
\NormalTok{pbdbT3Pub <-}\StringTok{ }\KeywordTok{rowSums}\NormalTok{(pbdbFamDiv)}


\CommentTok{# plot fluctuations to see that they're comprable}
\KeywordTok{pdf}\NormalTok{(}\StringTok{'ms/figSupp_divEstComp.pdf'}\NormalTok{, }\DataTypeTok{width =} \DecValTok{8}\NormalTok{, }\DataTypeTok{height =} \DecValTok{4}\NormalTok{)}
\KeywordTok{layout}\NormalTok{(}\KeywordTok{matrix}\NormalTok{(}\DecValTok{1}\OperatorTok{:}\DecValTok{2}\NormalTok{, }\DataTypeTok{nrow =} \DecValTok{1}\NormalTok{))}

\KeywordTok{par}\NormalTok{(}\DataTypeTok{mar =} \KeywordTok{c}\NormalTok{(}\FloatTok{4.5}\NormalTok{, }\FloatTok{2.5}\NormalTok{, }\DecValTok{0}\NormalTok{, }\FloatTok{0.5}\NormalTok{) }\OperatorTok{+}\StringTok{ }\FloatTok{0.5}\NormalTok{, }\DataTypeTok{mgp =} \KeywordTok{c}\NormalTok{(}\DecValTok{2}\NormalTok{, }\FloatTok{0.75}\NormalTok{, }\DecValTok{0}\NormalTok{))}
\KeywordTok{plot}\NormalTok{(}\DecValTok{1}\NormalTok{, }\DataTypeTok{xlim =} \KeywordTok{c}\NormalTok{(}\DecValTok{540}\NormalTok{, }\DecValTok{0}\NormalTok{), }\DataTypeTok{ylim =} \KeywordTok{c}\NormalTok{(}\OperatorTok{-}\DecValTok{400}\NormalTok{, }\DecValTok{400}\NormalTok{), }\DataTypeTok{type =} \StringTok{'n'}\NormalTok{, }\DataTypeTok{xaxt =} \StringTok{'n'}\NormalTok{, }
     \DataTypeTok{xlab =} \StringTok{''}\NormalTok{, }\DataTypeTok{ylab =} \StringTok{'Richness fluctuations'}\NormalTok{, }\DataTypeTok{xaxs =} \StringTok{'i'}\NormalTok{)}
\KeywordTok{paleoAxis}\NormalTok{(}\DecValTok{1}\NormalTok{)}
\KeywordTok{mtext}\NormalTok{(}\StringTok{'Millions of years ago'}\NormalTok{, }\DataTypeTok{side =} \DecValTok{1}\NormalTok{, }\DataTypeTok{line =} \FloatTok{3.5}\NormalTok{)}

\KeywordTok{lines}\NormalTok{(tbin}\OperatorTok{$}\NormalTok{ma_mid[}\OperatorTok{-}\DecValTok{1}\NormalTok{], }\KeywordTok{diff}\NormalTok{(pbdbCR}\OperatorTok{$}\NormalTok{divCSIB), }\DataTypeTok{col =} \StringTok{'black'}\NormalTok{, }\DataTypeTok{lwd =} \DecValTok{2}\NormalTok{)}
\KeywordTok{lines}\NormalTok{(tbin}\OperatorTok{$}\NormalTok{ma_mid[}\OperatorTok{-}\DecValTok{1}\NormalTok{], }\KeywordTok{diff}\NormalTok{(pbdbSQS}\OperatorTok{$}\NormalTok{divCSIB), }\DataTypeTok{col =} \StringTok{'blue'}\NormalTok{, }\DataTypeTok{lwd =} \DecValTok{2}\NormalTok{)}
\KeywordTok{lines}\NormalTok{(tbin}\OperatorTok{$}\NormalTok{ma_mid[}\OperatorTok{-}\KeywordTok{c}\NormalTok{(}\DecValTok{1}\OperatorTok{:}\DecValTok{2}\NormalTok{, }\KeywordTok{nrow}\NormalTok{(tbin))], }\KeywordTok{diff}\NormalTok{(pbdbT3Pub), }\DataTypeTok{col =} \StringTok{'red'}\NormalTok{, }\DataTypeTok{lwd =} \DecValTok{2}\NormalTok{)}


\KeywordTok{par}\NormalTok{(}\DataTypeTok{mar =} \KeywordTok{c}\NormalTok{(}\DecValTok{3}\NormalTok{, }\DecValTok{3}\NormalTok{, }\DecValTok{0}\NormalTok{, }\DecValTok{0}\NormalTok{) }\OperatorTok{+}\StringTok{ }\FloatTok{0.5}\NormalTok{, }\DataTypeTok{mgp =} \KeywordTok{c}\NormalTok{(}\DecValTok{2}\NormalTok{, }\FloatTok{0.75}\NormalTok{, }\DecValTok{0}\NormalTok{))}
\KeywordTok{plot}\NormalTok{(}\KeywordTok{simpECDF}\NormalTok{(}\KeywordTok{c}\NormalTok{(}\DecValTok{1}\NormalTok{, }\KeywordTok{abs}\NormalTok{(}\KeywordTok{diff}\NormalTok{(pbdbT3Pub))), }\DataTypeTok{complement =} \OtherTok{TRUE}\NormalTok{), }\DataTypeTok{col =} \StringTok{'red'}\NormalTok{, }\DataTypeTok{log =} \StringTok{'xy'}\NormalTok{, }
     \DataTypeTok{type =} \StringTok{'l'}\NormalTok{, }\DataTypeTok{lwd =} \DecValTok{2}\NormalTok{, }\DataTypeTok{xlim =} \KeywordTok{c}\NormalTok{(}\DecValTok{1}\NormalTok{, }\DecValTok{500}\NormalTok{),}
     \DataTypeTok{panel.first =}\NormalTok{ \{}
         \KeywordTok{lines}\NormalTok{(}\KeywordTok{simpECDF}\NormalTok{(}\KeywordTok{c}\NormalTok{(}\DecValTok{1}\NormalTok{, }\KeywordTok{abs}\NormalTok{(}\KeywordTok{diff}\NormalTok{(pbdbCR}\OperatorTok{$}\NormalTok{divCSIB))), }\DataTypeTok{complement =} \OtherTok{TRUE}\NormalTok{), }
               \DataTypeTok{col =} \StringTok{'black'}\NormalTok{, }\DataTypeTok{lwd =} \DecValTok{2}\NormalTok{)}
         \KeywordTok{lines}\NormalTok{(}\KeywordTok{simpECDF}\NormalTok{(}\KeywordTok{c}\NormalTok{(}\DecValTok{1}\NormalTok{, }\KeywordTok{abs}\NormalTok{(}\KeywordTok{diff}\NormalTok{(pbdbSQS}\OperatorTok{$}\NormalTok{divCSIB))), }\DataTypeTok{complement =} \OtherTok{TRUE}\NormalTok{), }
               \DataTypeTok{col =} \StringTok{'blue'}\NormalTok{, }\DataTypeTok{lwd =} \DecValTok{2}\NormalTok{)}
\NormalTok{     \}, }
     \DataTypeTok{axes =} \OtherTok{FALSE}\NormalTok{, }\DataTypeTok{frame.plot =} \OtherTok{TRUE}\NormalTok{, }
     \DataTypeTok{xlab =} \StringTok{'|Fluctuations|'}\NormalTok{, }\DataTypeTok{ylab =} \StringTok{'Cumulative density'}\NormalTok{)}
\KeywordTok{logAxis}\NormalTok{(}\DecValTok{1}\OperatorTok{:}\DecValTok{2}\NormalTok{)}


\KeywordTok{legend}\NormalTok{(}\StringTok{'bottomleft'}\NormalTok{, }\DataTypeTok{legend =} \KeywordTok{c}\NormalTok{(}\StringTok{'Rarefaction'}\NormalTok{, }\StringTok{'SQS'}\NormalTok{, }\StringTok{'3 timer pub'}\NormalTok{), }
       \DataTypeTok{lty =} \DecValTok{1}\NormalTok{, }\DataTypeTok{lwd =} \DecValTok{2}\NormalTok{, }\DataTypeTok{col =} \KeywordTok{c}\NormalTok{(}\StringTok{'black'}\NormalTok{, }\StringTok{'blue'}\NormalTok{, }\StringTok{'red'}\NormalTok{), }\DataTypeTok{bty =} \StringTok{'n'}\NormalTok{)}

\KeywordTok{dev.off}\NormalTok{()}
\end{Highlighting}
\end{Shaded}

\section{Super statistical analysis of 3TPub-corrected PBDB
data}\label{super-statistical-analysis-of-3tpub-corrected-pbdb-data}

Once data have been bias-corrected we can complete their
super-statistical analysis. We do that in the script
\texttt{analysis/pbdb\_sstat.R} shown here:

\begin{Shaded}
\begin{Highlighting}[]
\CommentTok{# **script to run super stat analysis on PBDB data and make plots**}

\CommentTok{# source needed functions}
\NormalTok{R.utils}\OperatorTok{::}\KeywordTok{sourceDirectory}\NormalTok{(}\StringTok{'R'}\NormalTok{, }\DataTypeTok{modifiedOnly =} \OtherTok{FALSE}\NormalTok{)}
\KeywordTok{library}\NormalTok{(socorro) }\CommentTok{# for plotting}

\CommentTok{# load and prepare data}
\CommentTok{# ---------------------}

\NormalTok{pbdbFamDiv <-}\StringTok{ }\KeywordTok{read.csv}\NormalTok{(}\StringTok{'data/pbdb_3TPub-corrected.csv'}\NormalTok{, }\DataTypeTok{row.names =} \DecValTok{1}\NormalTok{)}


\CommentTok{# coarsen to higher taxonomic groupings}

\NormalTok{pbdbTax <-}\StringTok{ }\KeywordTok{read.csv}\NormalTok{(}\StringTok{'data/pbdb_taxa.csv'}\NormalTok{, }\DataTypeTok{as.is =} \OtherTok{TRUE}\NormalTok{)}

\CommentTok{#' helper function to coarsen taxonomic resolution of `pbdbFamDiv` object}
\CommentTok{#' @param level a character string specifying the taxonomic level (from order through phylum)}

\NormalTok{coarsenTaxa <-}\StringTok{ }\ControlFlowTok{function}\NormalTok{(level) \{}
\NormalTok{    m <-}\StringTok{ }\KeywordTok{tidy2mat}\NormalTok{(pbdbTax}\OperatorTok{$}\NormalTok{family[}\KeywordTok{match}\NormalTok{(}\KeywordTok{colnames}\NormalTok{(pbdbFamDiv), pbdbTax}\OperatorTok{$}\NormalTok{family)], }
\NormalTok{                  pbdbTax[}\KeywordTok{match}\NormalTok{(}\KeywordTok{colnames}\NormalTok{(pbdbFamDiv), pbdbTax}\OperatorTok{$}\NormalTok{family), level], }
                  \KeywordTok{rep}\NormalTok{(}\DecValTok{1}\NormalTok{, }\KeywordTok{ncol}\NormalTok{(pbdbFamDiv)))}
\NormalTok{    m <-}\StringTok{ }\NormalTok{m[}\KeywordTok{colnames}\NormalTok{(pbdbFamDiv), ]}
    
\NormalTok{    out <-}\StringTok{ }\KeywordTok{as.matrix}\NormalTok{(pbdbFamDiv) }\OperatorTok\StringTok{ }\NormalTok{m}
\NormalTok{    out <-}\StringTok{ }\NormalTok{out[, }\KeywordTok{colnames}\NormalTok{(out) }\OperatorTok{!=}\StringTok{ ''}\NormalTok{]}
    
    \KeywordTok{return}\NormalTok{(out)}
\NormalTok{\}}

\NormalTok{pbdbOrdDiv <-}\StringTok{ }\KeywordTok{coarsenTaxa}\NormalTok{(}\StringTok{'order'}\NormalTok{)}
\NormalTok{pbdbClsDiv <-}\StringTok{ }\KeywordTok{coarsenTaxa}\NormalTok{(}\StringTok{'class'}\NormalTok{)}
\NormalTok{pbdbPhyDiv <-}\StringTok{ }\KeywordTok{coarsenTaxa}\NormalTok{(}\StringTok{'phylum'}\NormalTok{)}


\CommentTok{# tbin midpoints}
\NormalTok{tbinMid <-}\StringTok{ }\KeywordTok{read.csv}\NormalTok{(}\StringTok{'data/tbinsMid.csv'}\NormalTok{, }\DataTypeTok{as.is =} \OtherTok{TRUE}\NormalTok{)}
\NormalTok{tbinNames <-}\StringTok{ }\NormalTok{tbinMid}\OperatorTok{$}\NormalTok{tbin}
\NormalTok{tbinMid <-}\StringTok{ }\KeywordTok{as.numeric}\NormalTok{(tbinMid[, }\DecValTok{2}\NormalTok{])}
\KeywordTok{names}\NormalTok{(tbinMid) <-}\StringTok{ }\NormalTok{tbinNames}

\NormalTok{tbinMid <-}\StringTok{ }\NormalTok{tbinMid[}\KeywordTok{rownames}\NormalTok{(pbdbFamDiv)]}


\CommentTok{# super stat analysis}
\CommentTok{# -------------------}


\CommentTok{# calculate flux for families}
\NormalTok{pbdbFamFlux <-}\StringTok{ }\KeywordTok{calcFlux}\NormalTok{(pbdbFamDiv)}

\CommentTok{# calculate the mean flux}
\NormalTok{famMeans <-}\StringTok{ }\KeywordTok{sapply}\NormalTok{(pbdbFamFlux, mean)}
\KeywordTok{mean}\NormalTok{(famMeans)}
\KeywordTok{sd}\NormalTok{(famMeans)}


\CommentTok{# make sstat object for families}
\NormalTok{sstatPBDBfam3TP <-}\StringTok{ }\KeywordTok{sstatComp}\NormalTok{(pbdbFamFlux, }\DataTypeTok{minN =} \DecValTok{10}\NormalTok{, }\DataTypeTok{plotit =} \OtherTok{FALSE}\NormalTok{)}

\CommentTok{# deltaAIC}
\KeywordTok{logLik}\NormalTok{(sstatPBDBfam3TP) }\OperatorTok{-}\StringTok{ }\KeywordTok{sum}\NormalTok{(}\KeywordTok{dnorm}\NormalTok{(}\KeywordTok{unlist}\NormalTok{(sstatPBDBfam3TP}\OperatorTok{$}\NormalTok{Px.sub), }\DataTypeTok{log =} \OtherTok{TRUE}\NormalTok{))}

\CommentTok{# likelihood CI for family-level sstat analysis}
\NormalTok{sstatPBDBfam3TPCI <-}\StringTok{ }\KeywordTok{bootMLE.sstat}\NormalTok{(sstatPBDBfam3TP, }\DataTypeTok{B =} \DecValTok{1000}\NormalTok{, }\DataTypeTok{useAll =} \OtherTok{FALSE}\NormalTok{)}


\CommentTok{# do the same for higher taxo levels}
\NormalTok{pbdbOrdFlux <-}\StringTok{ }\KeywordTok{calcFlux}\NormalTok{(pbdbOrdDiv)}
\NormalTok{sstatPBDBOrd <-}\StringTok{ }\KeywordTok{sstatComp}\NormalTok{(pbdbOrdFlux, }\DataTypeTok{minN =} \DecValTok{10}\NormalTok{, }\DataTypeTok{plotit =} \OtherTok{FALSE}\NormalTok{)}

\NormalTok{pbdbClsFlux <-}\StringTok{ }\KeywordTok{calcFlux}\NormalTok{(pbdbClsDiv)}
\NormalTok{sstatPBDBCls <-}\StringTok{ }\KeywordTok{sstatComp}\NormalTok{(pbdbClsFlux, }\DataTypeTok{minN =} \DecValTok{10}\NormalTok{, }\DataTypeTok{plotit =} \OtherTok{FALSE}\NormalTok{)}

\NormalTok{pbdbPhyFlux <-}\StringTok{ }\KeywordTok{calcFlux}\NormalTok{(pbdbPhyDiv)}
\NormalTok{sstatPBDBPhy <-}\StringTok{ }\KeywordTok{sstatComp}\NormalTok{(pbdbPhyFlux, }\DataTypeTok{minN =} \DecValTok{10}\NormalTok{, }\DataTypeTok{plotit =} \OtherTok{FALSE}\NormalTok{)}

\CommentTok{# save the sstat analyses for future use}
\KeywordTok{save}\NormalTok{(sstatPBDBfam3TP, sstatPBDBOrd, sstatPBDBCls, sstatPBDBPhy, }
     \DataTypeTok{file =} \StringTok{'data/pbdb_sstat_objects.RData'}\NormalTok{)}


\CommentTok{# plot all sstat analyses}
\KeywordTok{pdf}\NormalTok{(}\StringTok{'ms/fig_Px.pdf'}\NormalTok{, }\DataTypeTok{width =} \FloatTok{4.25} \OperatorTok{*}\StringTok{ }\FloatTok{1.25}\NormalTok{, }\DataTypeTok{height =} \DecValTok{4} \OperatorTok{*}\StringTok{ }\FloatTok{1.25}\NormalTok{)}

\KeywordTok{layout}\NormalTok{(}\KeywordTok{matrix}\NormalTok{(}\DecValTok{1}\OperatorTok{:}\DecValTok{4}\NormalTok{, }\DataTypeTok{nrow =} \DecValTok{2}\NormalTok{, }\DataTypeTok{byrow =} \OtherTok{TRUE}\NormalTok{))}
\KeywordTok{par}\NormalTok{(}\DataTypeTok{oma =} \KeywordTok{c}\NormalTok{(}\DecValTok{3}\NormalTok{, }\DecValTok{3}\NormalTok{, }\DecValTok{0}\NormalTok{, }\DecValTok{0}\NormalTok{) }\OperatorTok{+}\StringTok{ }\FloatTok{0.5}\NormalTok{, }\DataTypeTok{mar =} \KeywordTok{c}\NormalTok{(}\FloatTok{0.1}\NormalTok{, }\FloatTok{0.1}\NormalTok{, }\FloatTok{1.51}\NormalTok{, }\FloatTok{0.1}\NormalTok{), }
    \DataTypeTok{mgp =} \KeywordTok{c}\NormalTok{(}\DecValTok{2}\NormalTok{, }\FloatTok{0.5}\NormalTok{, }\DecValTok{0}\NormalTok{), }\DataTypeTok{cex.lab =} \FloatTok{1.4}\NormalTok{)}

\KeywordTok{plot}\NormalTok{(sstatPBDBfam3TP, }\DataTypeTok{xlim =} \KeywordTok{c}\NormalTok{(}\FloatTok{1e-04}\NormalTok{, }\FloatTok{5e+02}\NormalTok{), }\DataTypeTok{ylim =} \KeywordTok{c}\NormalTok{(}\FloatTok{8e-05}\NormalTok{, }\DecValTok{1}\NormalTok{),}
     \DataTypeTok{xaxt =} \StringTok{'n'}\NormalTok{, }\DataTypeTok{yaxt =} \StringTok{'n'}\NormalTok{,}
     \DataTypeTok{panel.first =} \KeywordTok{quote}\NormalTok{(}\KeywordTok{mlePoly}\NormalTok{(sstatPBDBfam3TPCI}\OperatorTok{$}\NormalTok{sstat, PPx.gam,}
                                 \DataTypeTok{col =} \KeywordTok{hsv}\NormalTok{(}\DataTypeTok{alpha =} \FloatTok{0.25}\NormalTok{), }\DataTypeTok{border =} \OtherTok{NA}\NormalTok{)))}
\KeywordTok{mtext}\NormalTok{(}\StringTok{'Families'}\NormalTok{, }\DataTypeTok{side =} \DecValTok{3}\NormalTok{, }\DataTypeTok{line =} \DecValTok{0}\NormalTok{)}
\KeywordTok{logAxis}\NormalTok{(}\DecValTok{2}\NormalTok{, }\DataTypeTok{expLab =} \OtherTok{TRUE}\NormalTok{)}
\KeywordTok{legend}\NormalTok{(}\StringTok{'topright'}\NormalTok{, }\DataTypeTok{legend =} \StringTok{'A'}\NormalTok{, }\DataTypeTok{bty =} \StringTok{'n'}\NormalTok{, }\DataTypeTok{cex =} \FloatTok{1.4}\NormalTok{)}

\KeywordTok{plot}\NormalTok{(sstatPBDBOrd, }\DataTypeTok{xlim =} \KeywordTok{c}\NormalTok{(}\FloatTok{1e-04}\NormalTok{, }\FloatTok{5e+02}\NormalTok{), }\DataTypeTok{ylim =} \KeywordTok{c}\NormalTok{(}\FloatTok{8e-05}\NormalTok{, }\DecValTok{1}\NormalTok{), }\DataTypeTok{xaxt =} \StringTok{'n'}\NormalTok{, }\DataTypeTok{yaxt =} \StringTok{'n'}\NormalTok{, }
     \DataTypeTok{addLegend =} \OtherTok{FALSE}\NormalTok{)}
\KeywordTok{mtext}\NormalTok{(}\StringTok{'Orders'}\NormalTok{, }\DataTypeTok{side =} \DecValTok{3}\NormalTok{, }\DataTypeTok{line =} \DecValTok{0}\NormalTok{)}
\KeywordTok{legend}\NormalTok{(}\StringTok{'topright'}\NormalTok{, }\DataTypeTok{legend =} \StringTok{'B'}\NormalTok{, }\DataTypeTok{bty =} \StringTok{'n'}\NormalTok{, }\DataTypeTok{cex =} \FloatTok{1.4}\NormalTok{)}

\KeywordTok{plot}\NormalTok{(sstatPBDBCls, }\DataTypeTok{xlim =} \KeywordTok{c}\NormalTok{(}\FloatTok{1e-04}\NormalTok{, }\FloatTok{5e+02}\NormalTok{), }\DataTypeTok{ylim =} \KeywordTok{c}\NormalTok{(}\FloatTok{8e-05}\NormalTok{, }\DecValTok{1}\NormalTok{), }\DataTypeTok{xaxt =} \StringTok{'n'}\NormalTok{, }\DataTypeTok{yaxt =} \StringTok{'n'}\NormalTok{, }
     \DataTypeTok{addLegend =} \OtherTok{FALSE}\NormalTok{)}
\KeywordTok{mtext}\NormalTok{(}\StringTok{'Classes'}\NormalTok{, }\DataTypeTok{side =} \DecValTok{3}\NormalTok{, }\DataTypeTok{line =} \DecValTok{0}\NormalTok{)}
\KeywordTok{logAxis}\NormalTok{(}\DecValTok{1}\OperatorTok{:}\DecValTok{2}\NormalTok{, }\DataTypeTok{expLab =} \OtherTok{TRUE}\NormalTok{)}
\KeywordTok{legend}\NormalTok{(}\StringTok{'topright'}\NormalTok{, }\DataTypeTok{legend =} \StringTok{'C'}\NormalTok{, }\DataTypeTok{bty =} \StringTok{'n'}\NormalTok{, }\DataTypeTok{cex =} \FloatTok{1.4}\NormalTok{)}

\KeywordTok{plot}\NormalTok{(sstatPBDBPhy, }\DataTypeTok{xlim =} \KeywordTok{c}\NormalTok{(}\FloatTok{1e-04}\NormalTok{, }\FloatTok{5e+02}\NormalTok{), }\DataTypeTok{ylim =} \KeywordTok{c}\NormalTok{(}\FloatTok{8e-05}\NormalTok{, }\DecValTok{1}\NormalTok{), }\DataTypeTok{xaxt =} \StringTok{'n'}\NormalTok{, }\DataTypeTok{yaxt =} \StringTok{'n'}\NormalTok{, }
     \DataTypeTok{addLegend =} \OtherTok{FALSE}\NormalTok{)}
\KeywordTok{mtext}\NormalTok{(}\StringTok{'Phyla'}\NormalTok{, }\DataTypeTok{side =} \DecValTok{3}\NormalTok{, }\DataTypeTok{line =} \DecValTok{0}\NormalTok{)}
\KeywordTok{logAxis}\NormalTok{(}\DecValTok{1}\NormalTok{, }\DataTypeTok{expLab =} \OtherTok{TRUE}\NormalTok{)}
\KeywordTok{legend}\NormalTok{(}\StringTok{'topright'}\NormalTok{, }\DataTypeTok{legend =} \StringTok{'D'}\NormalTok{, }\DataTypeTok{bty =} \StringTok{'n'}\NormalTok{, }\DataTypeTok{cex =} \FloatTok{1.4}\NormalTok{)}

\KeywordTok{mtext}\NormalTok{(}\StringTok{'|Fluctuations|'}\NormalTok{, }\DataTypeTok{side =} \DecValTok{1}\NormalTok{, }\DataTypeTok{outer =} \OtherTok{TRUE}\NormalTok{, }\DataTypeTok{line =} \DecValTok{2}\NormalTok{)}
\KeywordTok{mtext}\NormalTok{(}\StringTok{'Cumulative density'}\NormalTok{, }\DataTypeTok{side =} \DecValTok{2}\NormalTok{, }\DataTypeTok{outer =} \OtherTok{TRUE}\NormalTok{, }\DataTypeTok{line =} \DecValTok{2}\NormalTok{)}
\KeywordTok{dev.off}\NormalTok{()}




\CommentTok{# plot p_k(x|b) and f(beta) for families}
\CommentTok{# --------------------------------------}

\CommentTok{# idea for normality test: sample 1 from each order and do ks test on that subsampled set}

\CommentTok{# highlight individual trajectories}
\NormalTok{loFam <-}\StringTok{ 'Tainoceratidae'} \CommentTok{# nautiloid}
\NormalTok{miFam <-}\StringTok{ 'Lophospiridae'} \CommentTok{# sea snails}
\NormalTok{hiFam <-}\StringTok{ 'Spondylidae'} \CommentTok{# bivalve}
\NormalTok{lo <-}\StringTok{ }\NormalTok{pbdbFamDiv[, loFam]}
\NormalTok{mi <-}\StringTok{ }\NormalTok{pbdbFamDiv[, miFam]}
\NormalTok{hi <-}\StringTok{ }\NormalTok{pbdbFamDiv[, hiFam]}
\NormalTok{cols <-}\StringTok{ }\KeywordTok{hsv}\NormalTok{(}\DataTypeTok{h =} \KeywordTok{c}\NormalTok{(}\FloatTok{0.7}\NormalTok{, }\FloatTok{0.45}\NormalTok{, }\FloatTok{0.12}\NormalTok{), }\DataTypeTok{s =} \KeywordTok{c}\NormalTok{(}\FloatTok{0.7}\NormalTok{, }\DecValTok{1}\NormalTok{, }\DecValTok{1}\NormalTok{), }\DataTypeTok{v =} \KeywordTok{c}\NormalTok{(}\FloatTok{0.8}\NormalTok{, }\FloatTok{0.8}\NormalTok{, }\DecValTok{1}\NormalTok{))}
\KeywordTok{names}\NormalTok{(cols) <-}\StringTok{ }\KeywordTok{c}\NormalTok{(}\StringTok{'hi'}\NormalTok{, }\StringTok{'mi'}\NormalTok{, }\StringTok{'lo'}\NormalTok{)}

\CommentTok{# make CDF for all scale family-level fluctuations}
\NormalTok{pAll <-}\StringTok{ }\KeywordTok{lapply}\NormalTok{(sstatPBDBfam3TP}\OperatorTok{$}\NormalTok{raw.pk, }
               \ControlFlowTok{function}\NormalTok{(x) }\KeywordTok{simpECDF}\NormalTok{(}\KeywordTok{scale}\NormalTok{(x)[, }\DecValTok{1}\NormalTok{], }\DataTypeTok{complement =} \OtherTok{TRUE}\NormalTok{))}

\NormalTok{pHighlight <-}\StringTok{ }\NormalTok{pAll[}\KeywordTok{c}\NormalTok{(loFam, miFam, hiFam)]}

\NormalTok{pAll <-}\StringTok{ }\KeywordTok{do.call}\NormalTok{(rbind, pAll)}

\CommentTok{# function to help with individual trajectory plotting}
\NormalTok{trajLines <-}\StringTok{ }\ControlFlowTok{function}\NormalTok{(t, x, ...) \{}
\NormalTok{    x[x }\OperatorTok{==}\StringTok{ }\DecValTok{0}\NormalTok{] <-}\StringTok{ }\OtherTok{NA}
\NormalTok{    alive <-}\StringTok{ }\KeywordTok{range}\NormalTok{(}\KeywordTok{which}\NormalTok{(}\OperatorTok{!}\KeywordTok{is.na}\NormalTok{(x)))}
    
\NormalTok{    x[}\KeywordTok{min}\NormalTok{(alive) }\OperatorTok{-}\StringTok{ }\DecValTok{1}\NormalTok{] <-}\StringTok{ }\DecValTok{0}
\NormalTok{    x[}\KeywordTok{max}\NormalTok{(alive) }\OperatorTok{+}\StringTok{ }\DecValTok{1}\NormalTok{] <-}\StringTok{ }\DecValTok{0}
    
\NormalTok{    t <-}\StringTok{ }\NormalTok{t[}\OperatorTok{!}\KeywordTok{is.na}\NormalTok{(x)]}
\NormalTok{    x <-}\StringTok{ }\NormalTok{x[}\OperatorTok{!}\KeywordTok{is.na}\NormalTok{(x)]}
    
    \KeywordTok{lines}\NormalTok{(t, x, ...)}
\NormalTok{\}}

\CommentTok{# the actual plotting}

\KeywordTok{pdf}\NormalTok{(}\StringTok{'ms/fig_pkx-fbeta.pdf'}\NormalTok{, }\DataTypeTok{width =} \FloatTok{4.25} \OperatorTok{*}\StringTok{ }\FloatTok{1.25}\NormalTok{, }\DataTypeTok{height =} \DecValTok{4} \OperatorTok{*}\StringTok{ }\FloatTok{1.25}\NormalTok{)}

\KeywordTok{layout}\NormalTok{(}\KeywordTok{matrix}\NormalTok{(}\KeywordTok{c}\NormalTok{(}\DecValTok{1}\NormalTok{, }\DecValTok{2}\NormalTok{, }\DecValTok{1}\NormalTok{, }\DecValTok{3}\NormalTok{), }\DataTypeTok{nrow =} \DecValTok{2}\NormalTok{))}

\KeywordTok{par}\NormalTok{(}\DataTypeTok{oma =} \KeywordTok{c}\NormalTok{(}\DecValTok{0}\NormalTok{, }\DecValTok{3}\NormalTok{, }\DecValTok{0}\NormalTok{, }\DecValTok{0}\NormalTok{) }\OperatorTok{+}\StringTok{ }\FloatTok{0.25}\NormalTok{, }\DataTypeTok{mar =} \KeywordTok{c}\NormalTok{(}\DecValTok{4}\NormalTok{, }\DecValTok{0}\NormalTok{, }\DecValTok{0}\NormalTok{, }\DecValTok{0}\NormalTok{) }\OperatorTok{+}\StringTok{ }\FloatTok{0.25}\NormalTok{, }
    \DataTypeTok{mgp =} \KeywordTok{c}\NormalTok{(}\DecValTok{2}\NormalTok{, }\FloatTok{0.5}\NormalTok{, }\DecValTok{0}\NormalTok{), }\DataTypeTok{cex.lab =} \FloatTok{1.4}\NormalTok{)}

\KeywordTok{plot}\NormalTok{(}\DecValTok{1}\NormalTok{, }\DataTypeTok{xlim =} \KeywordTok{c}\NormalTok{(}\DecValTok{540}\NormalTok{, }\DecValTok{0}\NormalTok{), }\DataTypeTok{xaxt =} \StringTok{'n'}\NormalTok{, }\DataTypeTok{xaxs =} \StringTok{'i'}\NormalTok{, }\DataTypeTok{xlab =} \StringTok{''}\NormalTok{, }
     \DataTypeTok{ylim =} \KeywordTok{c}\NormalTok{(}\DecValTok{0}\NormalTok{, }\KeywordTok{max}\NormalTok{(lo, mi, hi, }\DataTypeTok{na.rm =} \OtherTok{TRUE}\NormalTok{)), }\DataTypeTok{type =} \StringTok{'n'}\NormalTok{)}

\KeywordTok{trajLines}\NormalTok{(tbinMid, lo, }\DataTypeTok{col =}\NormalTok{ cols[}\StringTok{'lo'}\NormalTok{], }\DataTypeTok{lwd =} \DecValTok{2}\NormalTok{)}
\KeywordTok{trajLines}\NormalTok{(tbinMid, mi, }\DataTypeTok{col =}\NormalTok{ cols[}\StringTok{'mi'}\NormalTok{], }\DataTypeTok{lwd =} \DecValTok{2}\NormalTok{)}
\KeywordTok{trajLines}\NormalTok{(tbinMid, hi, }\DataTypeTok{col =}\NormalTok{ cols[}\StringTok{'hi'}\NormalTok{], }\DataTypeTok{lwd =} \DecValTok{2}\NormalTok{)}

\KeywordTok{text}\NormalTok{(}\KeywordTok{c}\NormalTok{(}\DecValTok{450}\NormalTok{, }\DecValTok{230}\NormalTok{, }\DecValTok{10}\NormalTok{), }\KeywordTok{c}\NormalTok{(}\DecValTok{4}\NormalTok{, }\FloatTok{5.25}\NormalTok{, }\DecValTok{2}\NormalTok{), }\DataTypeTok{labels =} \KeywordTok{c}\NormalTok{(miFam, loFam, hiFam), }
     \DataTypeTok{col =}\NormalTok{ cols[}\KeywordTok{c}\NormalTok{(}\StringTok{'mi'}\NormalTok{, }\StringTok{'lo'}\NormalTok{, }\StringTok{'hi'}\NormalTok{)], }\DataTypeTok{pos =} \KeywordTok{c}\NormalTok{(}\DecValTok{3}\NormalTok{, }\DecValTok{4}\NormalTok{, }\DecValTok{2}\NormalTok{))}

\KeywordTok{paleoAxis}\NormalTok{(}\DecValTok{1}\NormalTok{)}
\KeywordTok{mtext}\NormalTok{(}\StringTok{'Millions of years ago'}\NormalTok{, }\DataTypeTok{side =} \DecValTok{1}\NormalTok{, }\DataTypeTok{line =} \FloatTok{3.5}\NormalTok{)}
\KeywordTok{mtext}\NormalTok{(}\StringTok{'Standardized richness'}\NormalTok{, }\DataTypeTok{side =} \DecValTok{2}\NormalTok{, }\DataTypeTok{line =} \DecValTok{2}\NormalTok{)}

\KeywordTok{legend}\NormalTok{(}\StringTok{'topright'}\NormalTok{, }\DataTypeTok{legend =} \StringTok{'A'}\NormalTok{, }\DataTypeTok{pch =} \OtherTok{NA}\NormalTok{, }\DataTypeTok{bty =} \StringTok{'n'}\NormalTok{, }\DataTypeTok{cex =} \FloatTok{1.4}\NormalTok{)}

\CommentTok{# scale fluctuations}
\KeywordTok{par}\NormalTok{(}\DataTypeTok{mar =} \KeywordTok{c}\NormalTok{(}\DecValTok{3}\NormalTok{, }\DecValTok{0}\NormalTok{, }\DecValTok{1}\NormalTok{, }\DecValTok{0}\NormalTok{) }\OperatorTok{+}\StringTok{ }\FloatTok{0.25}\NormalTok{)}
\KeywordTok{plot}\NormalTok{(pAll, }\DataTypeTok{xlim =} \KeywordTok{c}\NormalTok{(}\OperatorTok{-}\DecValTok{4}\NormalTok{, }\DecValTok{4}\NormalTok{), }\DataTypeTok{col =} \StringTok{'gray'}\NormalTok{, }\DataTypeTok{ylim =} \KeywordTok{c}\NormalTok{(}\DecValTok{0}\NormalTok{, }\FloatTok{1.025}\NormalTok{),}
     \DataTypeTok{xlab =} \StringTok{'Scaled fluctuations'}\NormalTok{)}
\KeywordTok{mtext}\NormalTok{(}\StringTok{'Cumultive density'}\NormalTok{, }\DataTypeTok{side =} \DecValTok{2}\NormalTok{, }\DataTypeTok{line =} \DecValTok{2}\NormalTok{)}

\ControlFlowTok{for}\NormalTok{(i }\ControlFlowTok{in} \DecValTok{1}\OperatorTok{:}\KeywordTok{length}\NormalTok{(pHighlight)) }\KeywordTok{lines}\NormalTok{(pHighlight[[i]], }\DataTypeTok{col =}\NormalTok{ cols[i], }\DataTypeTok{lwd =} \DecValTok{2}\NormalTok{)}

\KeywordTok{curve}\NormalTok{(}\KeywordTok{pnorm}\NormalTok{(x, }\DataTypeTok{lower.tail =} \OtherTok{FALSE}\NormalTok{), }\DataTypeTok{lwd =} \DecValTok{2}\NormalTok{, }\DataTypeTok{add =} \OtherTok{TRUE}\NormalTok{)}

\KeywordTok{legend}\NormalTok{(}\StringTok{'topright'}\NormalTok{, }\DataTypeTok{legend =} \StringTok{'B'}\NormalTok{, }\DataTypeTok{pch =} \OtherTok{NA}\NormalTok{, }\DataTypeTok{bty =} \StringTok{'n'}\NormalTok{, }\DataTypeTok{cex =} \FloatTok{1.4}\NormalTok{)}

\CommentTok{# CDF of beta}
\NormalTok{betaCDF <-}\StringTok{ }\KeywordTok{simpECDF}\NormalTok{(sstatPBDBfam3TP}\OperatorTok{$}\NormalTok{beta, }\DataTypeTok{complement =} \OtherTok{TRUE}\NormalTok{)}
\KeywordTok{plot}\NormalTok{(betaCDF, }\DataTypeTok{ylim =} \KeywordTok{c}\NormalTok{(}\DecValTok{0}\NormalTok{, }\FloatTok{1.025}\NormalTok{),}
     \DataTypeTok{log =} \StringTok{'x'}\NormalTok{, }\DataTypeTok{xaxt =} \StringTok{'n'}\NormalTok{, }\DataTypeTok{yaxt =} \StringTok{'n'}\NormalTok{,}
     \DataTypeTok{xlab =} \KeywordTok{expression}\NormalTok{(beta), }\DataTypeTok{col =} \StringTok{'gray'}\NormalTok{)}

\NormalTok{theseBeta <-}\StringTok{ }\NormalTok{sstatPBDBfam3TP}\OperatorTok{$}\NormalTok{beta[}\KeywordTok{c}\NormalTok{(loFam, miFam, hiFam)]}
\KeywordTok{points}\NormalTok{(theseBeta, }\KeywordTok{approxfun}\NormalTok{(betaCDF)(theseBeta), }\DataTypeTok{bg =}\NormalTok{ cols, }\DataTypeTok{pch =} \DecValTok{21}\NormalTok{, }\DataTypeTok{cex =} \FloatTok{1.2}\NormalTok{)}

\KeywordTok{logAxis}\NormalTok{(}\DecValTok{1}\NormalTok{, }\DataTypeTok{expLab =} \OtherTok{TRUE}\NormalTok{)}

\KeywordTok{curve}\NormalTok{(}\KeywordTok{pgamma}\NormalTok{(x, sstatPBDBfam3TP}\OperatorTok{$}\NormalTok{gam.par[}\DecValTok{1}\NormalTok{], sstatPBDBfam3TP}\OperatorTok{$}\NormalTok{gam.par[}\DecValTok{2}\NormalTok{], }
             \DataTypeTok{lower.tail =} \OtherTok{FALSE}\NormalTok{), }
      \DataTypeTok{col =} \StringTok{'black'}\NormalTok{, }\DataTypeTok{lwd =} \DecValTok{2}\NormalTok{, }\DataTypeTok{add =} \OtherTok{TRUE}\NormalTok{)}

\KeywordTok{legend}\NormalTok{(}\StringTok{'topright'}\NormalTok{, }\DataTypeTok{legend =} \StringTok{'C'}\NormalTok{, }\DataTypeTok{pch =} \OtherTok{NA}\NormalTok{, }\DataTypeTok{bty =} \StringTok{'n'}\NormalTok{, }\DataTypeTok{cex =} \FloatTok{1.4}\NormalTok{)}

\KeywordTok{dev.off}\NormalTok{()}
\end{Highlighting}
\end{Shaded}

Now we can calculate a measure of goodness of fit with the
Komolgorov-Smirnov test statistics \(D\). That is done in the script
\texttt{analysis/pbdb\_dperm.R}. This script uses a permutational
approach to create a null distribution of test statistics. The goal is
to see if the good fit of super-statistics at the family and order
levels is purely from the number of different groupings at those levels,
regardless of the biology that might be going on to make those levels
actually mechanistically meaningful. So to achieve such a null, we
permute orders across families, calculate the D-statistics of those
permuted groupings, and compare to the real D-statistics from the actual
biological groupings. The script is shown below:

\begin{Shaded}
\begin{Highlighting}[]
\CommentTok{# **script to caculate d stats on sstat objects and null permutations**}

\KeywordTok{library}\NormalTok{(socorro)}
\NormalTok{R.utils}\OperatorTok{::}\KeywordTok{sourceDirectory}\NormalTok{(}\StringTok{'R'}\NormalTok{, }\DataTypeTok{modifiedOnly =} \OtherTok{FALSE}\NormalTok{)}

\CommentTok{# read in data}
\NormalTok{pbdbGenDiv <-}\StringTok{ }\KeywordTok{read.csv}\NormalTok{(}\StringTok{'data/pbdb_3TPub_genera.csv'}\NormalTok{, }\DataTypeTok{as.is =} \OtherTok{TRUE}\NormalTok{)}
\NormalTok{pbdbTax <-}\StringTok{ }\KeywordTok{read.csv}\NormalTok{(}\StringTok{'data/pbdb_taxa.csv'}\NormalTok{, }\DataTypeTok{as.is =} \OtherTok{TRUE}\NormalTok{)}
\CommentTok{# pbdbFamDiv <- read.csv('data/pbdb_3TPub-corrected.csv', row.names = 1)}
\KeywordTok{load}\NormalTok{(}\StringTok{'data/pbdb_sstat_objects.RData'}\NormalTok{)}

\CommentTok{# the indeces of otu names in `pbdbTax` ordered by their occurence in `pbdbGenDiv`}
\CommentTok{# needed to match permuted families to genera}
\NormalTok{genHash <-}\StringTok{ }\KeywordTok{match}\NormalTok{(pbdbGenDiv}\OperatorTok{$}\NormalTok{otu, pbdbTax}\OperatorTok{$}\NormalTok{otu)}


\CommentTok{#' function to calculate KS test d-stat on sstat objects}
\CommentTok{#' @param x an sstat object}

\NormalTok{ks.sstat <-}\StringTok{ }\ControlFlowTok{function}\NormalTok{(x) \{}
\NormalTok{    dat <-}\StringTok{ }\KeywordTok{unlist}\NormalTok{(x}\OperatorTok{$}\NormalTok{Px.sub)}
\NormalTok{    dat <-}\StringTok{ }\KeywordTok{abs}\NormalTok{(dat)}
    
    \CommentTok{# cumulative density function}
\NormalTok{    pfun <-}\StringTok{ }\ControlFlowTok{function}\NormalTok{(X) x}\OperatorTok{$}\KeywordTok{PPx}\NormalTok{(X, }\DataTypeTok{comp =} \OtherTok{TRUE}\NormalTok{)}
    
    \CommentTok{# cumulative prob observed and from theory}
\NormalTok{    n <-}\StringTok{ }\KeywordTok{length}\NormalTok{(dat)}
\NormalTok{    pobs <-}\StringTok{ }\NormalTok{(n}\OperatorTok{:}\DecValTok{1}\NormalTok{) }\OperatorTok{/}\StringTok{ }\NormalTok{n}
\NormalTok{    pthr <-}\StringTok{ }\KeywordTok{pfun}\NormalTok{(}\KeywordTok{sort}\NormalTok{(dat))}
    
    \CommentTok{# the statisic is the difference between obs and thr}
\NormalTok{    out <-}\StringTok{ }\NormalTok{pthr }\OperatorTok{-}\StringTok{ }\NormalTok{pobs}
    
    \KeywordTok{return}\NormalTok{(}\KeywordTok{max}\NormalTok{(out, }\DecValTok{1} \OperatorTok{/}\StringTok{ }\NormalTok{n }\OperatorTok{-}\StringTok{ }\NormalTok{out, }\DataTypeTok{na.rm =} \OtherTok{TRUE}\NormalTok{))}
\NormalTok{\}}


\CommentTok{# sstat on real (non-permuted) data}
\NormalTok{dObsFam <-}\StringTok{ }\KeywordTok{ks.sstat}\NormalTok{(sstatPBDBfam3TP)}
\NormalTok{dObsOrd <-}\StringTok{ }\KeywordTok{ks.sstat}\NormalTok{(sstatPBDBOrd)}
\NormalTok{dObsCls <-}\StringTok{ }\KeywordTok{ks.sstat}\NormalTok{(sstatPBDBCls)}
\NormalTok{dObsPhy <-}\StringTok{ }\KeywordTok{ks.sstat}\NormalTok{(sstatPBDBPhy)}


\CommentTok{# repeatedly permute data and calculate null ks statistics}
\NormalTok{B <-}\StringTok{ }\DecValTok{500}
\NormalTok{dNull <-}\StringTok{ }\NormalTok{parallel}\OperatorTok{::}\KeywordTok{mclapply}\NormalTok{(}\DecValTok{1}\OperatorTok{:}\NormalTok{B, }\DataTypeTok{mc.cores =} \DecValTok{8}\NormalTok{, }\DataTypeTok{FUN =} \ControlFlowTok{function}\NormalTok{(i) \{}
\NormalTok{    newFam <-}\StringTok{ }\KeywordTok{sample}\NormalTok{(pbdbTax}\OperatorTok{$}\NormalTok{family)}
\NormalTok{    newDiv <-}\StringTok{ }\KeywordTok{tidy2mat}\NormalTok{(pbdbGenDiv}\OperatorTok{$}\NormalTok{tbin, newFam[genHash], pbdbGenDiv}\OperatorTok{$}\NormalTok{T3PubDiv)}
\NormalTok{    newFlux <-}\StringTok{ }\KeywordTok{calcFlux}\NormalTok{(newDiv)}
\NormalTok{    newSstat <-}\StringTok{ }\KeywordTok{sstatComp}\NormalTok{(newFlux, }\DataTypeTok{minN =} \DecValTok{10}\NormalTok{, }\DataTypeTok{plotit =} \OtherTok{FALSE}\NormalTok{)}
    
    \KeywordTok{ks.sstat}\NormalTok{(newSstat)}
    \CommentTok{# return(ks.sstat(newSstat))}
\NormalTok{\})}

\NormalTok{dNull <-}\StringTok{ }\KeywordTok{unlist}\NormalTok{(dNull)}


\CommentTok{# save output in case it's ever needed}
\KeywordTok{save}\NormalTok{(dNull, }\DataTypeTok{file =} \StringTok{'data/dnull.RData'}\NormalTok{)}


\CommentTok{# plotting}
\KeywordTok{pdf}\NormalTok{(}\StringTok{'ms/fig_dStat.pdf'}\NormalTok{, }\DataTypeTok{width =} \DecValTok{4}\NormalTok{, }\DataTypeTok{height =} \DecValTok{4}\NormalTok{)}
\CommentTok{# colors for plotting taxa}
\NormalTok{tcols <-}\StringTok{ }\KeywordTok{colorRampPalette}\NormalTok{(}\KeywordTok{hsv}\NormalTok{(}\KeywordTok{c}\NormalTok{(}\FloatTok{0.12}\NormalTok{, }\DecValTok{0}\NormalTok{, }\FloatTok{0.02}\NormalTok{), }\KeywordTok{c}\NormalTok{(}\DecValTok{1}\NormalTok{, }\FloatTok{0.9}\NormalTok{, }\FloatTok{0.7}\NormalTok{), }\KeywordTok{c}\NormalTok{(}\DecValTok{1}\NormalTok{, }\FloatTok{0.8}\NormalTok{, }\FloatTok{0.3}\NormalTok{)))(}\DecValTok{4}\NormalTok{)}

\KeywordTok{par}\NormalTok{(}\DataTypeTok{mar =} \KeywordTok{c}\NormalTok{(}\DecValTok{3}\NormalTok{, }\DecValTok{3}\NormalTok{, }\DecValTok{0}\NormalTok{, }\DecValTok{0}\NormalTok{) }\OperatorTok{+}\StringTok{ }\FloatTok{0.5}\NormalTok{, }\DataTypeTok{mgp =} \KeywordTok{c}\NormalTok{(}\DecValTok{2}\NormalTok{, }\FloatTok{0.75}\NormalTok{, }\DecValTok{0}\NormalTok{))}
\KeywordTok{denFill}\NormalTok{(dNull, }\DataTypeTok{xlim =} \KeywordTok{range}\NormalTok{(dNull, dObsFam, dObsOrd, dObsCls, dObsPhy) }\OperatorTok{*}\StringTok{ }\KeywordTok{c}\NormalTok{(}\FloatTok{0.9}\NormalTok{, }\FloatTok{1.1}\NormalTok{), }
        \DataTypeTok{xlab =} \StringTok{'D-statistic'}\NormalTok{, }\DataTypeTok{main =} \StringTok{''}\NormalTok{)}

\KeywordTok{abline}\NormalTok{(}\DataTypeTok{v =}\NormalTok{ dObsFam, }\DataTypeTok{lwd =} \DecValTok{2}\NormalTok{, }\DataTypeTok{col =}\NormalTok{ tcols[}\DecValTok{1}\NormalTok{])}
\KeywordTok{text}\NormalTok{(dObsFam, }\FloatTok{1.25} \OperatorTok{*}\StringTok{ }\KeywordTok{mean}\NormalTok{(}\KeywordTok{par}\NormalTok{(}\StringTok{'usr'}\NormalTok{)[}\DecValTok{3}\OperatorTok{:}\DecValTok{4}\NormalTok{]), }\DataTypeTok{labels =} \StringTok{'Families'}\NormalTok{, }\DataTypeTok{col =}\NormalTok{ tcols[}\DecValTok{1}\NormalTok{],}
     \DataTypeTok{srt =} \DecValTok{90}\NormalTok{, }\DataTypeTok{pos =} \DecValTok{4}\NormalTok{)}

\KeywordTok{abline}\NormalTok{(}\DataTypeTok{v =}\NormalTok{ dObsOrd, }\DataTypeTok{lwd =} \DecValTok{2}\NormalTok{, }\DataTypeTok{col =}\NormalTok{ tcols[}\DecValTok{2}\NormalTok{])}
\KeywordTok{text}\NormalTok{(dObsOrd, }\FloatTok{1.25} \OperatorTok{*}\StringTok{ }\KeywordTok{mean}\NormalTok{(}\KeywordTok{par}\NormalTok{(}\StringTok{'usr'}\NormalTok{)[}\DecValTok{3}\OperatorTok{:}\DecValTok{4}\NormalTok{]), }\DataTypeTok{labels =} \StringTok{'Orders'}\NormalTok{, }\DataTypeTok{col =}\NormalTok{ tcols[}\DecValTok{2}\NormalTok{],}
     \DataTypeTok{srt =} \DecValTok{90}\NormalTok{, }\DataTypeTok{pos =} \DecValTok{4}\NormalTok{)}

\KeywordTok{abline}\NormalTok{(}\DataTypeTok{v =}\NormalTok{ dObsCls, }\DataTypeTok{lwd =} \DecValTok{2}\NormalTok{, }\DataTypeTok{col =}\NormalTok{ tcols[}\DecValTok{3}\NormalTok{])}
\KeywordTok{text}\NormalTok{(dObsCls, }\FloatTok{1.25} \OperatorTok{*}\StringTok{ }\KeywordTok{mean}\NormalTok{(}\KeywordTok{par}\NormalTok{(}\StringTok{'usr'}\NormalTok{)[}\DecValTok{3}\OperatorTok{:}\DecValTok{4}\NormalTok{]), }\DataTypeTok{labels =} \StringTok{'Classes'}\NormalTok{, }\DataTypeTok{col =}\NormalTok{ tcols[}\DecValTok{3}\NormalTok{],}
     \DataTypeTok{srt =} \DecValTok{90}\NormalTok{, }\DataTypeTok{pos =} \DecValTok{4}\NormalTok{)}

\KeywordTok{abline}\NormalTok{(}\DataTypeTok{v =}\NormalTok{ dObsPhy, }\DataTypeTok{lwd =} \DecValTok{2}\NormalTok{, }\DataTypeTok{col =}\NormalTok{ tcols[}\DecValTok{4}\NormalTok{])}
\KeywordTok{text}\NormalTok{(dObsPhy, }\FloatTok{1.25} \OperatorTok{*}\StringTok{ }\KeywordTok{mean}\NormalTok{(}\KeywordTok{par}\NormalTok{(}\StringTok{'usr'}\NormalTok{)[}\DecValTok{3}\OperatorTok{:}\DecValTok{4}\NormalTok{]), }\DataTypeTok{labels =} \StringTok{'Phyla'}\NormalTok{, }\DataTypeTok{col =}\NormalTok{ tcols[}\DecValTok{4}\NormalTok{],}
     \DataTypeTok{srt =} \DecValTok{90}\NormalTok{, }\DataTypeTok{adj =} \KeywordTok{c}\NormalTok{(}\DecValTok{0}\NormalTok{, }\OperatorTok{-}\FloatTok{0.5}\NormalTok{))}

\KeywordTok{dev.off}\NormalTok{()}
\end{Highlighting}
\end{Shaded}

This permutational null test is in part motivated by the correlation
between the genus richness of a family and its \(\beta_k\) value. We
demonstrate this correlation in the script
\texttt{analysis/pbdb\_betaRichness.R} which is reproduced below:

\begin{Shaded}
\begin{Highlighting}[]
\KeywordTok{library}\NormalTok{(socorro)}

\KeywordTok{load}\NormalTok{(}\StringTok{'data/pbdb_sstat_objects.RData'}\NormalTok{)}
\NormalTok{pbdbDat <-}\StringTok{ }\KeywordTok{read.csv}\NormalTok{(}\StringTok{'data/pbdb_data.csv'}\NormalTok{, }\DataTypeTok{as.is =} \OtherTok{TRUE}\NormalTok{)}

\NormalTok{divFamRaw <-}\StringTok{ }\KeywordTok{tapply}\NormalTok{(pbdbDat}\OperatorTok{$}\NormalTok{otu, pbdbDat}\OperatorTok{$}\NormalTok{family, }\ControlFlowTok{function}\NormalTok{(x) }\KeywordTok{length}\NormalTok{(}\KeywordTok{unique}\NormalTok{(x)))}

\KeywordTok{pdf}\NormalTok{(}\StringTok{'ms/figSupp_betaByRich.pdf'}\NormalTok{, }\DataTypeTok{width =} \DecValTok{4}\NormalTok{, }\DataTypeTok{height =} \DecValTok{4}\NormalTok{)}
\KeywordTok{par}\NormalTok{(}\DataTypeTok{mar =} \KeywordTok{c}\NormalTok{(}\DecValTok{3}\NormalTok{, }\DecValTok{3}\NormalTok{, }\DecValTok{0}\NormalTok{, }\DecValTok{0}\NormalTok{) }\OperatorTok{+}\StringTok{ }\FloatTok{0.5}\NormalTok{, }\DataTypeTok{mgp =} \KeywordTok{c}\NormalTok{(}\DecValTok{2}\NormalTok{, }\FloatTok{0.75}\NormalTok{, }\DecValTok{0}\NormalTok{))}
\KeywordTok{plot}\NormalTok{(divFamRaw[}\KeywordTok{names}\NormalTok{(sstatPBDBfam3TP}\OperatorTok{$}\NormalTok{beta)], sstatPBDBfam3TP}\OperatorTok{$}\NormalTok{beta, }\DataTypeTok{log =} \StringTok{'xy'}\NormalTok{, }
     \DataTypeTok{xlab =} \StringTok{'Number of genera'}\NormalTok{, }\DataTypeTok{ylab =} \KeywordTok{expression}\NormalTok{(beta), }
     \DataTypeTok{axes =} \OtherTok{FALSE}\NormalTok{, }\DataTypeTok{frame.plot =} \OtherTok{TRUE}\NormalTok{)}
\KeywordTok{logAxis}\NormalTok{(}\DecValTok{1}\OperatorTok{:}\DecValTok{2}\NormalTok{)}
\KeywordTok{dev.off}\NormalTok{()}
\end{Highlighting}
\end{Shaded}

Now that we are reasonably convinced that these superstatistical
findings are not just an artifact of taxonomy or clade size, we can
explore why we see deviations from super statistics with increasing
taxonomic level. We first explore how well the Gaussian
\(p_k(x | \beta_k)\) fit at each taxonomic level in the script
\texttt{analysis/pkx\_diffK.R} shown here:

\begin{Shaded}
\begin{Highlighting}[]
\CommentTok{# **script to compare within clade fluctuation distributions at different taxonomic levels**}

\KeywordTok{library}\NormalTok{(socorro)}

\CommentTok{# source needed functions}
\NormalTok{R.utils}\OperatorTok{::}\KeywordTok{sourceDirectory}\NormalTok{(}\StringTok{'R'}\NormalTok{, }\DataTypeTok{modifiedOnly =} \OtherTok{FALSE}\NormalTok{)}

\KeywordTok{load}\NormalTok{(}\StringTok{'data/pbdb_sstat_objects.RData'}\NormalTok{)}


\CommentTok{# for each family, calculate aggregated eCDF and distribution of kurtosis values}
\NormalTok{famECDF <-}\StringTok{ }\KeywordTok{lapply}\NormalTok{(sstatPBDBfam3TP}\OperatorTok{$}\NormalTok{Px.sub, }\ControlFlowTok{function}\NormalTok{(x) \{}
    \KeywordTok{simpECDF}\NormalTok{(}\KeywordTok{scale}\NormalTok{(x)[, }\DecValTok{1}\NormalTok{], }\DataTypeTok{complement =} \OtherTok{TRUE}\NormalTok{)}
\NormalTok{\})}
\NormalTok{famECDF <-}\StringTok{ }\KeywordTok{do.call}\NormalTok{(rbind, famECDF)}
\NormalTok{famKurt <-}\StringTok{ }\KeywordTok{sapply}\NormalTok{(sstatPBDBfam3TP}\OperatorTok{$}\NormalTok{Px.sub, kurt)}

\NormalTok{ordECDF <-}\StringTok{ }\KeywordTok{lapply}\NormalTok{(sstatPBDBOrd}\OperatorTok{$}\NormalTok{Px.sub, }\ControlFlowTok{function}\NormalTok{(x) \{}
    \KeywordTok{simpECDF}\NormalTok{(}\KeywordTok{scale}\NormalTok{(x)[, }\DecValTok{1}\NormalTok{], }\DataTypeTok{complement =} \OtherTok{TRUE}\NormalTok{)}
\NormalTok{\})}
\NormalTok{ordECDF <-}\StringTok{ }\KeywordTok{do.call}\NormalTok{(rbind, ordECDF)}
\NormalTok{ordKurt <-}\StringTok{ }\KeywordTok{sapply}\NormalTok{(sstatPBDBOrd}\OperatorTok{$}\NormalTok{Px.sub, kurt)}

\NormalTok{clsECDF <-}\StringTok{ }\KeywordTok{lapply}\NormalTok{(sstatPBDBCls}\OperatorTok{$}\NormalTok{Px.sub, }\ControlFlowTok{function}\NormalTok{(x) \{}
    \KeywordTok{simpECDF}\NormalTok{(}\KeywordTok{scale}\NormalTok{(x)[, }\DecValTok{1}\NormalTok{], }\DataTypeTok{complement =} \OtherTok{TRUE}\NormalTok{)}
\NormalTok{\})}
\NormalTok{clsECDF <-}\StringTok{ }\KeywordTok{do.call}\NormalTok{(rbind, clsECDF)}
\NormalTok{clsKurt <-}\StringTok{ }\KeywordTok{sapply}\NormalTok{(sstatPBDBCls}\OperatorTok{$}\NormalTok{Px.sub, kurt)}

\NormalTok{phyECDF <-}\StringTok{ }\KeywordTok{lapply}\NormalTok{(sstatPBDBPhy}\OperatorTok{$}\NormalTok{Px.sub, }\ControlFlowTok{function}\NormalTok{(x) \{}
    \KeywordTok{simpECDF}\NormalTok{(}\KeywordTok{scale}\NormalTok{(x)[, }\DecValTok{1}\NormalTok{], }\DataTypeTok{complement =} \OtherTok{TRUE}\NormalTok{)}
\NormalTok{\})}
\NormalTok{phyECDF <-}\StringTok{ }\KeywordTok{do.call}\NormalTok{(rbind, phyECDF)}
\NormalTok{phyKurt <-}\StringTok{ }\KeywordTok{sapply}\NormalTok{(sstatPBDBPhy}\OperatorTok{$}\NormalTok{Px.sub, kurt)}


\CommentTok{#' @description function to plot theoretical and observed percentiles}
\CommentTok{#' @param x aggregated eCDF}
\CommentTok{#' @param ... additional plotting parameters}

\NormalTok{ppECDF <-}\StringTok{ }\ControlFlowTok{function}\NormalTok{(x, ...) \{}
\NormalTok{    alpha <-}\StringTok{ }\FloatTok{0.75} \OperatorTok{/}\StringTok{ }\NormalTok{(}\DecValTok{1} \OperatorTok{+}\StringTok{ }\KeywordTok{exp}\NormalTok{(}\FloatTok{0.0003} \OperatorTok{*}\StringTok{ }\NormalTok{(}\KeywordTok{nrow}\NormalTok{(x) }\OperatorTok{-}\StringTok{ }\DecValTok{300}\NormalTok{))) }\CommentTok{# nicely scale transparency}
    \KeywordTok{plot}\NormalTok{(}\KeywordTok{pnorm}\NormalTok{(x[, }\DecValTok{1}\NormalTok{], }\DataTypeTok{lower.tail =} \OtherTok{FALSE}\NormalTok{), x[, }\DecValTok{2}\NormalTok{], }\DataTypeTok{pch =} \DecValTok{16}\NormalTok{,}
         \DataTypeTok{col =} \KeywordTok{gray}\NormalTok{(}\DecValTok{0}\NormalTok{, }\DataTypeTok{alpha =}\NormalTok{ alpha), }\DataTypeTok{xlim =} \DecValTok{0}\OperatorTok{:}\DecValTok{1}\NormalTok{, }\DataTypeTok{ylim =} \DecValTok{0}\OperatorTok{:}\DecValTok{1}\NormalTok{, ...)}
    
    \KeywordTok{abline}\NormalTok{(}\DecValTok{0}\NormalTok{, }\DecValTok{1}\NormalTok{, }\DataTypeTok{col =} \StringTok{'red'}\NormalTok{)}
\NormalTok{\}}


\CommentTok{#' @description function to plot summary of kurtosis values distribution}
\CommentTok{#' @param x kurtosis values}
\CommentTok{#' @param ... additional plotting parameters}

\NormalTok{kurtInset <-}\StringTok{ }\ControlFlowTok{function}\NormalTok{(x, ...) \{}
\NormalTok{    allMean <-}\StringTok{ }\KeywordTok{c}\NormalTok{(}\KeywordTok{mean}\NormalTok{(famKurt), }\KeywordTok{mean}\NormalTok{(ordKurt), }\KeywordTok{mean}\NormalTok{(clsKurt), }\KeywordTok{mean}\NormalTok{(phyKurt))}
\NormalTok{    allSD <-}\StringTok{ }\KeywordTok{c}\NormalTok{(}\KeywordTok{sd}\NormalTok{(famKurt), }\KeywordTok{sd}\NormalTok{(ordKurt), }\KeywordTok{sd}\NormalTok{(clsKurt), }\KeywordTok{sd}\NormalTok{(phyKurt))}
    
    \KeywordTok{plot}\NormalTok{(}\DecValTok{1}\NormalTok{, }\KeywordTok{mean}\NormalTok{(x), }\DataTypeTok{pch =} \DecValTok{16}\NormalTok{, }
         \DataTypeTok{ylim =} \KeywordTok{range}\NormalTok{(allMean }\OperatorTok{-}\StringTok{ }\NormalTok{allSD, allMean }\OperatorTok{+}\StringTok{ }\NormalTok{allSD) }\OperatorTok{*}\StringTok{ }\KeywordTok{c}\NormalTok{(}\FloatTok{1.5}\NormalTok{, }\FloatTok{1.25}\NormalTok{), }
         \DataTypeTok{xaxt =} \StringTok{'n'}\NormalTok{, }\DataTypeTok{frame.plot =} \OtherTok{FALSE}\NormalTok{, }\DataTypeTok{yaxs =} \StringTok{'i'}\NormalTok{, }
\NormalTok{         ...)}
    \KeywordTok{segments}\NormalTok{(}\DataTypeTok{x0 =} \DecValTok{1}\NormalTok{, }\DataTypeTok{y0 =} \KeywordTok{mean}\NormalTok{(x) }\OperatorTok{-}\StringTok{ }\KeywordTok{sd}\NormalTok{(x), }\DataTypeTok{y1 =} \KeywordTok{mean}\NormalTok{(x) }\OperatorTok{+}\StringTok{ }\KeywordTok{sd}\NormalTok{(x))}
\NormalTok{\}}




\CommentTok{# plot it}
\KeywordTok{pdf}\NormalTok{(}\StringTok{'ms/figSupp_pkx_allTaxa.pdf'}\NormalTok{, }\DataTypeTok{width =} \DecValTok{9}\NormalTok{, }\DataTypeTok{height =} \DecValTok{3}\NormalTok{)}

\KeywordTok{split.screen}\NormalTok{(}\KeywordTok{c}\NormalTok{(}\DecValTok{1}\NormalTok{, }\DecValTok{4}\NormalTok{))}

\CommentTok{# first plots of the ECDF's}
\KeywordTok{screen}\NormalTok{(}\DecValTok{1}\NormalTok{, }\DataTypeTok{new =} \OtherTok{FALSE}\NormalTok{)}
\KeywordTok{par}\NormalTok{(}\DataTypeTok{mar =} \KeywordTok{c}\NormalTok{(}\FloatTok{0.3}\NormalTok{, }\FloatTok{0.3}\NormalTok{, }\FloatTok{1.5}\NormalTok{, }\FloatTok{0.3}\NormalTok{), }\DataTypeTok{oma =} \KeywordTok{c}\NormalTok{(}\FloatTok{2.5}\NormalTok{, }\FloatTok{2.5}\NormalTok{, }\DecValTok{0}\NormalTok{, }\DecValTok{0}\NormalTok{), }
    \DataTypeTok{mgp =} \KeywordTok{c}\NormalTok{(}\DecValTok{2}\NormalTok{, }\FloatTok{0.5}\NormalTok{, }\DecValTok{0}\NormalTok{))}
\KeywordTok{ppECDF}\NormalTok{(famECDF)}
\KeywordTok{mtext}\NormalTok{(}\StringTok{'Families'}\NormalTok{, }\DataTypeTok{side =} \DecValTok{3}\NormalTok{, }\DataTypeTok{line =} \FloatTok{0.5}\NormalTok{)}

\KeywordTok{screen}\NormalTok{(}\DecValTok{2}\NormalTok{, }\DataTypeTok{new =} \OtherTok{FALSE}\NormalTok{)}
\KeywordTok{par}\NormalTok{(}\DataTypeTok{mar =} \KeywordTok{c}\NormalTok{(}\FloatTok{0.3}\NormalTok{, }\FloatTok{0.3}\NormalTok{, }\FloatTok{1.5}\NormalTok{, }\FloatTok{0.3}\NormalTok{), }\DataTypeTok{oma =} \KeywordTok{c}\NormalTok{(}\FloatTok{2.5}\NormalTok{, }\FloatTok{2.5}\NormalTok{, }\DecValTok{0}\NormalTok{, }\DecValTok{0}\NormalTok{), }
    \DataTypeTok{mgp =} \KeywordTok{c}\NormalTok{(}\DecValTok{2}\NormalTok{, }\FloatTok{0.5}\NormalTok{, }\DecValTok{0}\NormalTok{))}
\KeywordTok{ppECDF}\NormalTok{(ordECDF, }\DataTypeTok{yaxt =} \StringTok{'n'}\NormalTok{)}
\KeywordTok{mtext}\NormalTok{(}\StringTok{'Orders'}\NormalTok{, }\DataTypeTok{side =} \DecValTok{3}\NormalTok{, }\DataTypeTok{line =} \FloatTok{0.5}\NormalTok{)}

\KeywordTok{screen}\NormalTok{(}\DecValTok{3}\NormalTok{, }\DataTypeTok{new =} \OtherTok{FALSE}\NormalTok{)}
\KeywordTok{par}\NormalTok{(}\DataTypeTok{mar =} \KeywordTok{c}\NormalTok{(}\FloatTok{0.3}\NormalTok{, }\FloatTok{0.3}\NormalTok{, }\FloatTok{1.5}\NormalTok{, }\FloatTok{0.3}\NormalTok{), }\DataTypeTok{oma =} \KeywordTok{c}\NormalTok{(}\FloatTok{2.5}\NormalTok{, }\FloatTok{2.5}\NormalTok{, }\DecValTok{0}\NormalTok{, }\DecValTok{0}\NormalTok{), }
    \DataTypeTok{mgp =} \KeywordTok{c}\NormalTok{(}\DecValTok{2}\NormalTok{, }\FloatTok{0.5}\NormalTok{, }\DecValTok{0}\NormalTok{))}
\KeywordTok{ppECDF}\NormalTok{(clsECDF, }\DataTypeTok{yaxt =} \StringTok{'n'}\NormalTok{)}
\KeywordTok{mtext}\NormalTok{(}\StringTok{'Classes'}\NormalTok{, }\DataTypeTok{side =} \DecValTok{3}\NormalTok{, }\DataTypeTok{line =} \FloatTok{0.5}\NormalTok{)}

\KeywordTok{screen}\NormalTok{(}\DecValTok{4}\NormalTok{, }\DataTypeTok{new =} \OtherTok{FALSE}\NormalTok{)}
\KeywordTok{par}\NormalTok{(}\DataTypeTok{mar =} \KeywordTok{c}\NormalTok{(}\FloatTok{0.3}\NormalTok{, }\FloatTok{0.3}\NormalTok{, }\FloatTok{1.5}\NormalTok{, }\FloatTok{0.3}\NormalTok{), }\DataTypeTok{oma =} \KeywordTok{c}\NormalTok{(}\FloatTok{2.5}\NormalTok{, }\FloatTok{2.5}\NormalTok{, }\DecValTok{0}\NormalTok{, }\DecValTok{0}\NormalTok{), }
    \DataTypeTok{mgp =} \KeywordTok{c}\NormalTok{(}\DecValTok{2}\NormalTok{, }\FloatTok{0.5}\NormalTok{, }\DecValTok{0}\NormalTok{))}
\KeywordTok{ppECDF}\NormalTok{(phyECDF, }\DataTypeTok{yaxt =} \StringTok{'n'}\NormalTok{)}
\KeywordTok{mtext}\NormalTok{(}\StringTok{'Phyla'}\NormalTok{, }\DataTypeTok{side =} \DecValTok{3}\NormalTok{, }\DataTypeTok{line =} \FloatTok{0.5}\NormalTok{)}


\KeywordTok{mtext}\NormalTok{(}\StringTok{'N(0, 1) percentiles'}\NormalTok{, }\DataTypeTok{side =} \DecValTok{1}\NormalTok{, }\DataTypeTok{outer =} \OtherTok{TRUE}\NormalTok{, }\DataTypeTok{line =} \FloatTok{1.25}\NormalTok{)}
\KeywordTok{mtext}\NormalTok{(}\StringTok{'Observed percentiles'}\NormalTok{, }\DataTypeTok{side =} \DecValTok{2}\NormalTok{, }\DataTypeTok{outer =} \OtherTok{TRUE}\NormalTok{, }\DataTypeTok{line =} \FloatTok{1.25}\NormalTok{)}

\KeywordTok{close.screen}\NormalTok{(}\DataTypeTok{all.screens =} \OtherTok{TRUE}\NormalTok{)}


\CommentTok{# now inset plots of kurtosis}

\NormalTok{start <-}\StringTok{ }\DecValTok{1}\OperatorTok{/}\DecValTok{4} \OperatorTok{+}\StringTok{ }\FloatTok{0.01}
\NormalTok{swidth <-}\StringTok{ }\DecValTok{1}\OperatorTok{/}\DecValTok{32}
\NormalTok{increment <-}\StringTok{ }\DecValTok{1}\OperatorTok{/}\DecValTok{4} \OperatorTok{-}\StringTok{ }\DecValTok{1}\OperatorTok{/}\DecValTok{64}
\NormalTok{s <-}\StringTok{ }\KeywordTok{split.screen}\NormalTok{(}\KeywordTok{matrix}\NormalTok{(}\KeywordTok{c}\NormalTok{(start }\OperatorTok{+}\StringTok{ }\DecValTok{0} \OperatorTok{*}\StringTok{ }\NormalTok{increment, start }\OperatorTok{+}\StringTok{ }\DecValTok{0} \OperatorTok{*}\StringTok{ }\NormalTok{increment }\OperatorTok{+}\StringTok{ }\NormalTok{swidth, }\FloatTok{0.25}\NormalTok{, }\FloatTok{0.6}\NormalTok{, }
\NormalTok{                           start }\OperatorTok{+}\StringTok{ }\DecValTok{1} \OperatorTok{*}\StringTok{ }\NormalTok{increment, start }\OperatorTok{+}\StringTok{ }\DecValTok{1} \OperatorTok{*}\StringTok{ }\NormalTok{increment }\OperatorTok{+}\StringTok{ }\NormalTok{swidth, }\FloatTok{0.25}\NormalTok{, }\FloatTok{0.6}\NormalTok{, }
\NormalTok{                           start }\OperatorTok{+}\StringTok{ }\DecValTok{2} \OperatorTok{*}\StringTok{ }\NormalTok{increment, start }\OperatorTok{+}\StringTok{ }\DecValTok{2} \OperatorTok{*}\StringTok{ }\NormalTok{increment }\OperatorTok{+}\StringTok{ }\NormalTok{swidth, }\FloatTok{0.25}\NormalTok{, }\FloatTok{0.6}\NormalTok{, }
\NormalTok{                           start }\OperatorTok{+}\StringTok{ }\DecValTok{3} \OperatorTok{*}\StringTok{ }\NormalTok{increment, start }\OperatorTok{+}\StringTok{ }\DecValTok{3} \OperatorTok{*}\StringTok{ }\NormalTok{increment }\OperatorTok{+}\StringTok{ }\NormalTok{swidth, }\FloatTok{0.25}\NormalTok{, }\FloatTok{0.6}\NormalTok{),}
                         \DataTypeTok{ncol =} \DecValTok{4}\NormalTok{, }\DataTypeTok{byrow =} \OtherTok{TRUE}\NormalTok{), }\DataTypeTok{erase =} \OtherTok{FALSE}\NormalTok{)}

\ControlFlowTok{for}\NormalTok{(i }\ControlFlowTok{in} \DecValTok{1}\OperatorTok{:}\DecValTok{4}\NormalTok{) \{}
    \KeywordTok{screen}\NormalTok{(s[i], }\DataTypeTok{new =} \OtherTok{FALSE}\NormalTok{)}
    \KeywordTok{par}\NormalTok{(}\DataTypeTok{mar =} \KeywordTok{rep}\NormalTok{(}\DecValTok{0}\NormalTok{, }\DecValTok{4}\NormalTok{), }\DataTypeTok{mgp =} \KeywordTok{c}\NormalTok{(}\DecValTok{1}\NormalTok{, }\FloatTok{0.25}\NormalTok{, }\DecValTok{0}\NormalTok{))}
    \KeywordTok{kurtInset}\NormalTok{(}\ControlFlowTok{switch}\NormalTok{(i, }
                     \StringTok{`}\DataTypeTok{1}\StringTok{`}\NormalTok{ =}\StringTok{ }\NormalTok{famKurt, }
                     \StringTok{`}\DataTypeTok{2}\StringTok{`}\NormalTok{ =}\StringTok{ }\NormalTok{ordKurt, }
                     \StringTok{`}\DataTypeTok{3}\StringTok{`}\NormalTok{ =}\StringTok{ }\NormalTok{clsKurt, }
                     \StringTok{`}\DataTypeTok{4}\StringTok{`}\NormalTok{ =}\StringTok{ }\NormalTok{phyKurt), }
              \DataTypeTok{tcl =} \OperatorTok{-}\FloatTok{0.1}\NormalTok{)}
    \KeywordTok{mtext}\NormalTok{(}\StringTok{'Kurtosis'}\NormalTok{, }\DataTypeTok{side =} \DecValTok{2}\NormalTok{, }\DataTypeTok{line =} \FloatTok{1.25}\NormalTok{)}
\NormalTok{\}}


\KeywordTok{close.screen}\NormalTok{(}\DataTypeTok{all.screens =} \OtherTok{TRUE}\NormalTok{)}

\KeywordTok{dev.off}\NormalTok{()}
\end{Highlighting}
\end{Shaded}

We can also explore how well the \(f(\beta_k)\) fit at different
taxonomic levels in the script \texttt{analysis/pbeta\_diffK.R}
reproduced below:

\begin{Shaded}
\begin{Highlighting}[]
\CommentTok{# **script to compare within clade volatility distributions at different taxonomic levels**}

\KeywordTok{library}\NormalTok{(socorro)}

\CommentTok{# source needed functions}
\NormalTok{R.utils}\OperatorTok{::}\KeywordTok{sourceDirectory}\NormalTok{(}\StringTok{'R'}\NormalTok{, }\DataTypeTok{modifiedOnly =} \OtherTok{FALSE}\NormalTok{)}

\KeywordTok{load}\NormalTok{(}\StringTok{'data/pbdb_sstat_objects.RData'}\NormalTok{)}


\CommentTok{#' @description function to plot f(beta) distribution}
\CommentTok{#' @param obj the sstat object}
\CommentTok{#' @param thrCol color for plotting of theoretical curve}

\NormalTok{fbetaPlot <-}\StringTok{ }\ControlFlowTok{function}\NormalTok{(obj, }\DataTypeTok{thrCol =} \StringTok{'red'}\NormalTok{, ...) \{}
\NormalTok{    betaCDF <-}\StringTok{ }\KeywordTok{simpECDF}\NormalTok{(obj}\OperatorTok{$}\NormalTok{beta, }\DataTypeTok{complement =} \OtherTok{TRUE}\NormalTok{)}
    \KeywordTok{plot}\NormalTok{(betaCDF, }\DataTypeTok{ylim =} \KeywordTok{c}\NormalTok{(}\DecValTok{0}\NormalTok{, }\FloatTok{1.025}\NormalTok{),}
         \DataTypeTok{log =} \StringTok{'x'}\NormalTok{, }\DataTypeTok{xaxt =} \StringTok{'n'}\NormalTok{, }\DataTypeTok{yaxt =} \StringTok{'n'}\NormalTok{,}
         \DataTypeTok{xlab =} \KeywordTok{expression}\NormalTok{(beta), ...)}
    
    \KeywordTok{logAxis}\NormalTok{(}\DecValTok{1}\NormalTok{, }\DataTypeTok{expLab =} \OtherTok{TRUE}\NormalTok{)}
    
    \KeywordTok{curve}\NormalTok{(}\KeywordTok{pgamma}\NormalTok{(x, obj}\OperatorTok{$}\NormalTok{gam.par[}\DecValTok{1}\NormalTok{], obj}\OperatorTok{$}\NormalTok{gam.par[}\DecValTok{2}\NormalTok{], }
                 \DataTypeTok{lower.tail =} \OtherTok{FALSE}\NormalTok{), }
          \DataTypeTok{col =}\NormalTok{ thrCol, }\DataTypeTok{lwd =} \DecValTok{2}\NormalTok{, }\DataTypeTok{add =} \OtherTok{TRUE}\NormalTok{)}
    
\NormalTok{\}}


\CommentTok{# the plotting}
\KeywordTok{pdf}\NormalTok{(}\StringTok{'ms/figSupp_fbeta_allTaxa.pdf'}\NormalTok{, }\DataTypeTok{width =} \DecValTok{9}\NormalTok{, }\DataTypeTok{height =} \DecValTok{3}\NormalTok{)}

\KeywordTok{split.screen}\NormalTok{(}\KeywordTok{c}\NormalTok{(}\DecValTok{1}\NormalTok{, }\DecValTok{4}\NormalTok{))}

\KeywordTok{screen}\NormalTok{(}\DecValTok{1}\NormalTok{, }\DataTypeTok{new =} \OtherTok{FALSE}\NormalTok{)}
\KeywordTok{par}\NormalTok{(}\DataTypeTok{mar =} \KeywordTok{c}\NormalTok{(}\FloatTok{0.3}\NormalTok{, }\FloatTok{0.3}\NormalTok{, }\FloatTok{1.5}\NormalTok{, }\FloatTok{0.3}\NormalTok{), }\DataTypeTok{oma =} \KeywordTok{c}\NormalTok{(}\FloatTok{2.5}\NormalTok{, }\FloatTok{2.5}\NormalTok{, }\DecValTok{0}\NormalTok{, }\DecValTok{0}\NormalTok{), }
    \DataTypeTok{mgp =} \KeywordTok{c}\NormalTok{(}\DecValTok{2}\NormalTok{, }\FloatTok{0.5}\NormalTok{, }\DecValTok{0}\NormalTok{))}
\KeywordTok{fbetaPlot}\NormalTok{(sstatPBDBfam3TP)}
\KeywordTok{axis}\NormalTok{(}\DecValTok{2}\NormalTok{)}
\KeywordTok{mtext}\NormalTok{(}\StringTok{'Families'}\NormalTok{, }\DataTypeTok{side =} \DecValTok{3}\NormalTok{, }\DataTypeTok{line =} \FloatTok{0.5}\NormalTok{)}

\KeywordTok{screen}\NormalTok{(}\DecValTok{2}\NormalTok{, }\DataTypeTok{new =} \OtherTok{FALSE}\NormalTok{)}
\KeywordTok{par}\NormalTok{(}\DataTypeTok{mar =} \KeywordTok{c}\NormalTok{(}\FloatTok{0.3}\NormalTok{, }\FloatTok{0.3}\NormalTok{, }\FloatTok{1.5}\NormalTok{, }\FloatTok{0.3}\NormalTok{), }\DataTypeTok{oma =} \KeywordTok{c}\NormalTok{(}\FloatTok{2.5}\NormalTok{, }\FloatTok{2.5}\NormalTok{, }\DecValTok{0}\NormalTok{, }\DecValTok{0}\NormalTok{), }
    \DataTypeTok{mgp =} \KeywordTok{c}\NormalTok{(}\DecValTok{2}\NormalTok{, }\FloatTok{0.5}\NormalTok{, }\DecValTok{0}\NormalTok{))}
\KeywordTok{fbetaPlot}\NormalTok{(sstatPBDBOrd)}
\KeywordTok{mtext}\NormalTok{(}\StringTok{'Oders'}\NormalTok{, }\DataTypeTok{side =} \DecValTok{3}\NormalTok{, }\DataTypeTok{line =} \FloatTok{0.5}\NormalTok{)}

\KeywordTok{screen}\NormalTok{(}\DecValTok{3}\NormalTok{, }\DataTypeTok{new =} \OtherTok{FALSE}\NormalTok{)}
\KeywordTok{par}\NormalTok{(}\DataTypeTok{mar =} \KeywordTok{c}\NormalTok{(}\FloatTok{0.3}\NormalTok{, }\FloatTok{0.3}\NormalTok{, }\FloatTok{1.5}\NormalTok{, }\FloatTok{0.3}\NormalTok{), }\DataTypeTok{oma =} \KeywordTok{c}\NormalTok{(}\FloatTok{2.5}\NormalTok{, }\FloatTok{2.5}\NormalTok{, }\DecValTok{0}\NormalTok{, }\DecValTok{0}\NormalTok{), }
    \DataTypeTok{mgp =} \KeywordTok{c}\NormalTok{(}\DecValTok{2}\NormalTok{, }\FloatTok{0.5}\NormalTok{, }\DecValTok{0}\NormalTok{))}
\KeywordTok{fbetaPlot}\NormalTok{(sstatPBDBCls)}
\KeywordTok{mtext}\NormalTok{(}\StringTok{'Classes'}\NormalTok{, }\DataTypeTok{side =} \DecValTok{3}\NormalTok{, }\DataTypeTok{line =} \FloatTok{0.5}\NormalTok{)}

\KeywordTok{screen}\NormalTok{(}\DecValTok{4}\NormalTok{, }\DataTypeTok{new =} \OtherTok{FALSE}\NormalTok{)}
\KeywordTok{par}\NormalTok{(}\DataTypeTok{mar =} \KeywordTok{c}\NormalTok{(}\FloatTok{0.3}\NormalTok{, }\FloatTok{0.3}\NormalTok{, }\FloatTok{1.5}\NormalTok{, }\FloatTok{0.3}\NormalTok{), }\DataTypeTok{oma =} \KeywordTok{c}\NormalTok{(}\FloatTok{2.5}\NormalTok{, }\FloatTok{2.5}\NormalTok{, }\DecValTok{0}\NormalTok{, }\DecValTok{0}\NormalTok{), }
    \DataTypeTok{mgp =} \KeywordTok{c}\NormalTok{(}\DecValTok{2}\NormalTok{, }\FloatTok{0.5}\NormalTok{, }\DecValTok{0}\NormalTok{))}
\KeywordTok{fbetaPlot}\NormalTok{(sstatPBDBPhy)}
\KeywordTok{mtext}\NormalTok{(}\StringTok{'Phyla'}\NormalTok{, }\DataTypeTok{side =} \DecValTok{3}\NormalTok{, }\DataTypeTok{line =} \FloatTok{0.5}\NormalTok{)}

\KeywordTok{mtext}\NormalTok{(}\KeywordTok{expression}\NormalTok{(beta), }\DataTypeTok{side =} \DecValTok{1}\NormalTok{, }\DataTypeTok{outer =} \OtherTok{TRUE}\NormalTok{, }\DataTypeTok{line =} \FloatTok{1.25}\NormalTok{)}
\KeywordTok{mtext}\NormalTok{(}\StringTok{'Cumulative density'}\NormalTok{, }\DataTypeTok{side =} \DecValTok{2}\NormalTok{, }\DataTypeTok{outer =} \OtherTok{TRUE}\NormalTok{, }\DataTypeTok{line =} \FloatTok{1.25}\NormalTok{)}

\KeywordTok{close.screen}\NormalTok{(}\DataTypeTok{all.screens =} \OtherTok{TRUE}\NormalTok{)}
\KeywordTok{dev.off}\NormalTok{()}
\end{Highlighting}
\end{Shaded}

Part of our argument about the failure of superstatistics at higher
taxonomic levels is that these higher taxa aggregate increasingly
disperate subtaxa. To investigate this idea we look at the number of
guilds represented by the average taxon at each taxonomic level in the
script \texttt{analysis/pbdb\_ecoEvoSpace.R} shown here:

\begin{Shaded}
\begin{Highlighting}[]
\CommentTok{# **a script to evaluate hour occupancy of eco-evolutionary space changes }
\CommentTok{# across taxonomy**}

\KeywordTok{library}\NormalTok{(socorro)}

\NormalTok{pbdbDat <-}\StringTok{ }\KeywordTok{read.csv}\NormalTok{(}\StringTok{'data/pbdb_data.csv'}\NormalTok{, }\DataTypeTok{as.is =} \OtherTok{TRUE}\NormalTok{)}

\CommentTok{# extract only the eco/evo/life history data and remove duplicates}
\CommentTok{# `taxon_environment`, `reproduction`, `ontogeny`}
\NormalTok{eeSpaceDat <-}\StringTok{ }\NormalTok{pbdbDat[, }\KeywordTok{c}\NormalTok{(}\StringTok{'phylum'}\NormalTok{, }\StringTok{'class'}\NormalTok{, }\StringTok{'order'}\NormalTok{, }\StringTok{'family'}\NormalTok{, }\StringTok{'otu'}\NormalTok{, }
                          \StringTok{'taxon_environment'}\NormalTok{, }\StringTok{'motility'}\NormalTok{, }\StringTok{'life_habit'}\NormalTok{,}
                          \StringTok{'vision'}\NormalTok{, }\StringTok{'diet'}\NormalTok{, }\StringTok{'reproduction'}\NormalTok{, }\StringTok{'ontogeny'}\NormalTok{)]}
\NormalTok{eeSpaceDat <-}\StringTok{ }\NormalTok{eeSpaceDat[}\OperatorTok{!}\KeywordTok{duplicated}\NormalTok{(eeSpaceDat), ]}

\CommentTok{# remove entries that are all missing}
\NormalTok{eeSpaceDat <-}\StringTok{ }\NormalTok{eeSpaceDat[}\KeywordTok{rowSums}\NormalTok{(eeSpaceDat[, }\OperatorTok{-}\NormalTok{(}\DecValTok{1}\OperatorTok{:}\DecValTok{5}\NormalTok{)] }\OperatorTok{!=}\StringTok{ ''}\NormalTok{) }\OperatorTok{!=}\StringTok{ }\DecValTok{0}\NormalTok{, ]}


\CommentTok{#' function to determine how many eco-evo hypercubes are occupied by each taxonomic level}
\CommentTok{#' @param tax the taxonomic unit to consider}
\CommentTok{#' @param eeDat a data.frame containing eco-evo data}
\NormalTok{eeOcc <-}\StringTok{ }\ControlFlowTok{function}\NormalTok{(tax, eeDat) \{}
    \KeywordTok{sapply}\NormalTok{(}\KeywordTok{split}\NormalTok{(eeDat[tax }\OperatorTok{!=}\StringTok{ ''}\NormalTok{, ], tax[tax }\OperatorTok{!=}\StringTok{ ''}\NormalTok{]), }
           \ControlFlowTok{function}\NormalTok{(x) }\KeywordTok{sum}\NormalTok{(}\OperatorTok{!}\KeywordTok{duplicated}\NormalTok{(x)))}
\NormalTok{\}}

\CommentTok{#' bootstraps ee space occupancy}
\CommentTok{#' @param x the vector of niche occupancies}
\CommentTok{#' @param B number of bootrap replicates}
\CommentTok{#' @param fun the function to apply to each replicate}
\NormalTok{eeOccBoot <-}\StringTok{ }\ControlFlowTok{function}\NormalTok{(x, B, fun) \{}
    \KeywordTok{replicate}\NormalTok{(B, }\KeywordTok{fun}\NormalTok{(}\KeywordTok{sample}\NormalTok{(x, }\KeywordTok{length}\NormalTok{(x), }\DataTypeTok{replace =} \OtherTok{TRUE}\NormalTok{)))}
\NormalTok{\}}


\CommentTok{# calculate eco-evolutionary space occupancy of each taxonomic level}
\NormalTok{famEE <-}\StringTok{ }\KeywordTok{eeOccBoot}\NormalTok{(}\KeywordTok{eeOcc}\NormalTok{(eeSpaceDat}\OperatorTok{$}\NormalTok{family, eeSpaceDat[, }\OperatorTok{-}\NormalTok{(}\DecValTok{1}\OperatorTok{:}\DecValTok{5}\NormalTok{)]), }\DecValTok{500}\NormalTok{, mean)}
\NormalTok{ordEE <-}\StringTok{ }\KeywordTok{eeOccBoot}\NormalTok{(}\KeywordTok{eeOcc}\NormalTok{(eeSpaceDat}\OperatorTok{$}\NormalTok{order, eeSpaceDat[, }\OperatorTok{-}\NormalTok{(}\DecValTok{1}\OperatorTok{:}\DecValTok{5}\NormalTok{)]), }\DecValTok{500}\NormalTok{, mean)}
\NormalTok{clsEE <-}\StringTok{ }\KeywordTok{eeOccBoot}\NormalTok{(}\KeywordTok{eeOcc}\NormalTok{(eeSpaceDat}\OperatorTok{$}\NormalTok{class, eeSpaceDat[, }\OperatorTok{-}\NormalTok{(}\DecValTok{1}\OperatorTok{:}\DecValTok{5}\NormalTok{)]), }\DecValTok{500}\NormalTok{, mean)}
\NormalTok{phyEE <-}\StringTok{ }\KeywordTok{eeOccBoot}\NormalTok{(}\KeywordTok{eeOcc}\NormalTok{(eeSpaceDat}\OperatorTok{$}\NormalTok{phylum, eeSpaceDat[, }\OperatorTok{-}\NormalTok{(}\DecValTok{1}\OperatorTok{:}\DecValTok{5}\NormalTok{)]), }\DecValTok{500}\NormalTok{, mean)}


\CommentTok{# plotting}
\KeywordTok{pdf}\NormalTok{(}\StringTok{'ms/figSupp_eeSpaceOcc.pdf'}\NormalTok{, }\DataTypeTok{width =} \DecValTok{4}\NormalTok{, }\DataTypeTok{height =} \FloatTok{3.5}\NormalTok{)}

\KeywordTok{par}\NormalTok{(}\DataTypeTok{mar =} \KeywordTok{c}\NormalTok{(}\DecValTok{3}\NormalTok{, }\DecValTok{3}\NormalTok{, }\DecValTok{0}\NormalTok{, }\DecValTok{0}\NormalTok{) }\OperatorTok{+}\StringTok{ }\FloatTok{0.5}\NormalTok{, }\DataTypeTok{mgp =} \KeywordTok{c}\NormalTok{(}\DecValTok{2}\NormalTok{, }\FloatTok{0.75}\NormalTok{, }\DecValTok{0}\NormalTok{))}
\KeywordTok{plot}\NormalTok{(}\DecValTok{1}\OperatorTok{:}\DecValTok{4}\NormalTok{, }\DataTypeTok{ylim =} \KeywordTok{c}\NormalTok{(}\DecValTok{1}\NormalTok{, }\DecValTok{100}\NormalTok{), }\DataTypeTok{type =} \StringTok{'n'}\NormalTok{, }\DataTypeTok{log =} \StringTok{'y'}\NormalTok{, }\DataTypeTok{yaxt =} \StringTok{'n'}\NormalTok{, }\DataTypeTok{xaxt =} \StringTok{'n'}\NormalTok{,}
     \DataTypeTok{xlab =} \StringTok{'Taxonomic level'}\NormalTok{, }\DataTypeTok{ylab =} \StringTok{'Number of Bambach guilds'}\NormalTok{)}
\KeywordTok{axis}\NormalTok{(}\DecValTok{1}\NormalTok{, }\DataTypeTok{at =} \DecValTok{1}\OperatorTok{:}\DecValTok{4}\NormalTok{, }\DataTypeTok{labels =} \KeywordTok{c}\NormalTok{(}\StringTok{'Families'}\NormalTok{, }\StringTok{'Orders'}\NormalTok{, }\StringTok{'Classes'}\NormalTok{, }\StringTok{'Phyla'}\NormalTok{))}
\KeywordTok{logAxis}\NormalTok{(}\DecValTok{2}\NormalTok{)}
\KeywordTok{segments}\NormalTok{(}\DataTypeTok{x0 =} \DecValTok{1}\OperatorTok{:}\DecValTok{4}\NormalTok{, }\DataTypeTok{y0 =} \KeywordTok{c}\NormalTok{(}\KeywordTok{min}\NormalTok{(famEE), }\KeywordTok{min}\NormalTok{(ordEE), }\KeywordTok{min}\NormalTok{(clsEE), }\KeywordTok{min}\NormalTok{(phyEE)), }
         \DataTypeTok{y1 =} \KeywordTok{c}\NormalTok{(}\KeywordTok{max}\NormalTok{(famEE), }\KeywordTok{max}\NormalTok{(ordEE), }\KeywordTok{max}\NormalTok{(clsEE), }\KeywordTok{max}\NormalTok{(phyEE)), }\DataTypeTok{lwd =} \DecValTok{2}\NormalTok{)}
\KeywordTok{points}\NormalTok{(}\DecValTok{1}\OperatorTok{:}\DecValTok{4}\NormalTok{, }\KeywordTok{c}\NormalTok{(}\KeywordTok{mean}\NormalTok{(famEE), }\KeywordTok{mean}\NormalTok{(ordEE), }\KeywordTok{mean}\NormalTok{(clsEE), }\KeywordTok{mean}\NormalTok{(phyEE)), }\DataTypeTok{pch =} \DecValTok{16}\NormalTok{, }\DataTypeTok{cex =} \FloatTok{1.2}\NormalTok{)}

\KeywordTok{dev.off}\NormalTok{()}
\end{Highlighting}
\end{Shaded}

\section{Helper functions}\label{helper-functions}

All the above analyses make use of helpful functions in the \texttt{R}
directory. We reproduce those functions below:

\begin{Shaded}
\begin{Highlighting}[]
\CommentTok{#' helper function to calculate corrected flux}
\CommentTok{#' @param x the matrix of corrected diversities over which to calculate fluxes}

\NormalTok{calcFlux <-}\StringTok{ }\ControlFlowTok{function}\NormalTok{(x) \{}
    \KeywordTok{apply}\NormalTok{(x, }\DecValTok{2}\NormalTok{, }\ControlFlowTok{function}\NormalTok{(X) \{}
\NormalTok{        flux <-}\StringTok{ }\KeywordTok{diff}\NormalTok{(}\KeywordTok{c}\NormalTok{(}\DecValTok{0}\NormalTok{, X))}
        \KeywordTok{return}\NormalTok{(flux[flux }\OperatorTok{!=}\StringTok{ }\DecValTok{0}\NormalTok{])}
\NormalTok{    \})}
\NormalTok{\}}
\end{Highlighting}
\end{Shaded}

\begin{Shaded}
\begin{Highlighting}[]
\NormalTok{gammaLS <-}\StringTok{ }\ControlFlowTok{function}\NormalTok{(data,}\DataTypeTok{comp=}\OtherTok{FALSE}\NormalTok{) \{}
\NormalTok{    par.init <-}\StringTok{ }\KeywordTok{c}\NormalTok{(}\KeywordTok{mean}\NormalTok{(data)}\OperatorTok{^}\DecValTok{2}\NormalTok{,}\KeywordTok{mean}\NormalTok{(data))}\OperatorTok{/}\KeywordTok{var}\NormalTok{(data)}
    \KeywordTok{optim}\NormalTok{(par.init,gamma.ss,}\DataTypeTok{data=}\NormalTok{data,}\DataTypeTok{comp=}\NormalTok{comp)}
\NormalTok{\}}

\NormalTok{gamma.ss <-}\StringTok{ }\ControlFlowTok{function}\NormalTok{(pars,data,comp) \{}
\NormalTok{    shape <-}\StringTok{ }\NormalTok{pars[}\DecValTok{1}\NormalTok{]}
\NormalTok{    rate <-}\StringTok{ }\NormalTok{pars[}\DecValTok{2}\NormalTok{]}
\NormalTok{    tabz <-}\StringTok{ }\KeywordTok{table}\NormalTok{(data)}
\NormalTok{    xval <-}\StringTok{ }\KeywordTok{as.numeric}\NormalTok{(}\KeywordTok{names}\NormalTok{(tabz))}
\NormalTok{    yval <-}\StringTok{ }\KeywordTok{cumsum}\NormalTok{(}\KeywordTok{as.numeric}\NormalTok{(tabz))}\OperatorTok{/}\KeywordTok{sum}\NormalTok{(tabz)}
    \ControlFlowTok{if}\NormalTok{(comp) \{}
\NormalTok{        yval <-}\StringTok{ }\DecValTok{1} \OperatorTok{-}\StringTok{ }\NormalTok{yval}
\NormalTok{        yval <-}\StringTok{ }\KeywordTok{c}\NormalTok{(}\DecValTok{1}\NormalTok{,yval[}\OperatorTok{-}\KeywordTok{length}\NormalTok{(yval)])}
\NormalTok{        lower <-}\StringTok{ }\OtherTok{FALSE}
\NormalTok{    \} }\ControlFlowTok{else}\NormalTok{ \{}
\NormalTok{        lower <-}\StringTok{ }\OtherTok{TRUE}
\NormalTok{    \}}
\NormalTok{    difz <-}\StringTok{ }\KeywordTok{pgamma}\NormalTok{(xval,}\DataTypeTok{shape=}\NormalTok{shape,}\DataTypeTok{rate=}\NormalTok{rate,}\DataTypeTok{lower.tail=}\NormalTok{lower) }\OperatorTok{-}\StringTok{ }\NormalTok{yval}
    \KeywordTok{sum}\NormalTok{(difz}\OperatorTok{^}\DecValTok{2}\NormalTok{)}
\NormalTok{\}}
\end{Highlighting}
\end{Shaded}

\begin{Shaded}
\begin{Highlighting}[]
\CommentTok{#' @description function to produce a matrix of time by taxa with cells of corrected diversity}
\CommentTok{#' @param rawDiv the raw diversity of each taxon in each time interval}
\CommentTok{#' @param t3stat the 3 timer stat for each diversity record}
\CommentTok{#' @param pub the number of publications associated with each diversity record}
\CommentTok{#' @param taxa the taxon names for each diversity record}
\CommentTok{#' @param tbin the time interval of each diversity record}
\CommentTok{#' @param tbinTime times associated with each `tbin`}
\CommentTok{#' @param minPub minimum number of publications for inclusion in regression analysis}
\CommentTok{#' @param plotit logical, should plot of taxon richness versus number of publications be made}
\CommentTok{#' @return a matrix with rows corresponding to time intervals and columns to the given taxa}
\CommentTok{#' each cell in the matrix represents corrected taxon richness}


\NormalTok{make3TPub <-}\StringTok{ }\ControlFlowTok{function}\NormalTok{(rawDiv,  t3stat,  pub,  taxa,  tbin,  tbinTime,  }
                      \DataTypeTok{minPub =} \DecValTok{10}\NormalTok{,  }\DataTypeTok{plotit =} \OtherTok{FALSE}\NormalTok{) \{}
    \CommentTok{# put data together so can be universally manipulated}
\NormalTok{    x <-}\StringTok{ }\KeywordTok{data.frame}\NormalTok{(}\DataTypeTok{rawDiv =}\NormalTok{ rawDiv, }\DataTypeTok{t3stat =}\NormalTok{ t3stat, }\DataTypeTok{pub =}\NormalTok{ pub, }\DataTypeTok{taxa =}\NormalTok{ taxa, }\DataTypeTok{tbin =}\NormalTok{ tbin)}
\NormalTok{    x}\OperatorTok{$}\NormalTok{tbin <-}\StringTok{ }\KeywordTok{as.character}\NormalTok{(x}\OperatorTok{$}\NormalTok{tbin)}
\NormalTok{    x}\OperatorTok{$}\NormalTok{taxa <-}\StringTok{ }\KeywordTok{as.character}\NormalTok{(x}\OperatorTok{$}\NormalTok{taxa)}
    
\NormalTok{    x <-}\StringTok{ }\NormalTok{x[}\OperatorTok{!}\KeywordTok{is.na}\NormalTok{(t3stat) }\OperatorTok{&}\StringTok{ }\NormalTok{pub }\OperatorTok{>=}\StringTok{  }\NormalTok{minPub, ]}
    
\NormalTok{    tbinTime <-}\StringTok{ }\NormalTok{tbinTime[}\KeywordTok{names}\NormalTok{(tbinTime) }\OperatorTok\StringTok{ }\NormalTok{x}\OperatorTok{$}\NormalTok{tbin]}
    
    \CommentTok{# 3-timer correction}
\NormalTok{    t3cor <-}\StringTok{ }\NormalTok{x}\OperatorTok{$}\NormalTok{rawDiv}\OperatorTok{/}\NormalTok{x}\OperatorTok{$}\NormalTok{t3stat}
    
    \CommentTok{# publication correction}
\NormalTok{    logPub <-}\StringTok{ }\KeywordTok{log}\NormalTok{(x}\OperatorTok{$}\NormalTok{pub)}
\NormalTok{    pubLM <-}\StringTok{ }\KeywordTok{lm}\NormalTok{(}\KeywordTok{log}\NormalTok{(t3cor)}\OperatorTok{~}\NormalTok{logPub)}
\NormalTok{    pbdbPubLM <<-}\StringTok{ }\NormalTok{pubLM }\CommentTok{# save regression to global env}
    
\NormalTok{    pubResid <-}\StringTok{ }\KeywordTok{exp}\NormalTok{(pubLM}\OperatorTok{$}\NormalTok{residuals)}
    
    \CommentTok{# plot so you can verify cuttoff etc.}
    \ControlFlowTok{if}\NormalTok{(plotit) \{}
        \KeywordTok{plot}\NormalTok{(}\KeywordTok{log}\NormalTok{(x}\OperatorTok{$}\NormalTok{pub), }\KeywordTok{log}\NormalTok{(t3cor),  }
             \DataTypeTok{xlab =} \StringTok{'log(Number of publications)'}\NormalTok{, }
             \DataTypeTok{ylab =} \StringTok{'log(3T-corrected number of genera)'}\NormalTok{)}
        \KeywordTok{abline}\NormalTok{(pubLM, }\DataTypeTok{col =} \StringTok{'red'}\NormalTok{)}
\NormalTok{    \}}
    
\NormalTok{    tbinTaxa <-}\StringTok{ }\NormalTok{socorro}\OperatorTok{::}\KeywordTok{tidy2mat}\NormalTok{(x}\OperatorTok{$}\NormalTok{tbin, x}\OperatorTok{$}\NormalTok{taxa, pubResid)}

    \KeywordTok{return}\NormalTok{(tbinTaxa[}\KeywordTok{names}\NormalTok{(}\KeywordTok{sort}\NormalTok{(tbinTime, }\DataTypeTok{decreasing =} \OtherTok{TRUE}\NormalTok{)), ])}
\NormalTok{\}}
\end{Highlighting}
\end{Shaded}

\begin{Shaded}
\begin{Highlighting}[]
\NormalTok{normLS <-}\StringTok{ }\ControlFlowTok{function}\NormalTok{(data,}\DataTypeTok{comp=}\OtherTok{FALSE}\NormalTok{) \{}
\NormalTok{    par.init <-}\StringTok{ }\KeywordTok{c}\NormalTok{(}\KeywordTok{mean}\NormalTok{(data),}\KeywordTok{sd}\NormalTok{(data))}
    \KeywordTok{optim}\NormalTok{(par.init,norm.ss,}\DataTypeTok{data=}\NormalTok{data,}\DataTypeTok{comp=}\NormalTok{comp)}
\NormalTok{\}}

\NormalTok{norm.ss <-}\StringTok{ }\ControlFlowTok{function}\NormalTok{(pars,data,comp) \{}
\NormalTok{    mean <-}\StringTok{ }\NormalTok{pars[}\DecValTok{1}\NormalTok{]}
\NormalTok{    sd <-}\StringTok{ }\NormalTok{pars[}\DecValTok{2}\NormalTok{]}
\NormalTok{    tabz <-}\StringTok{ }\KeywordTok{table}\NormalTok{(data)}
\NormalTok{    xval <-}\StringTok{ }\KeywordTok{as.numeric}\NormalTok{(}\KeywordTok{names}\NormalTok{(tabz))}
\NormalTok{    yval <-}\StringTok{ }\KeywordTok{cumsum}\NormalTok{(}\KeywordTok{as.numeric}\NormalTok{(tabz))}\OperatorTok{/}\KeywordTok{sum}\NormalTok{(tabz)}
    \ControlFlowTok{if}\NormalTok{(comp) \{}
\NormalTok{        yval <-}\StringTok{ }\DecValTok{1} \OperatorTok{-}\StringTok{ }\NormalTok{yval}
\NormalTok{        yval <-}\StringTok{ }\KeywordTok{c}\NormalTok{(}\DecValTok{1}\NormalTok{,yval[}\OperatorTok{-}\KeywordTok{length}\NormalTok{(yval)])}
\NormalTok{        lower <-}\StringTok{ }\OtherTok{FALSE}
\NormalTok{    \} }\ControlFlowTok{else}\NormalTok{ \{}
\NormalTok{        lower <-}\StringTok{ }\OtherTok{TRUE}
\NormalTok{    \}}
\NormalTok{    difz <-}\StringTok{ }\KeywordTok{pnorm}\NormalTok{(xval,}\DataTypeTok{mean=}\NormalTok{mean,}\DataTypeTok{sd=}\NormalTok{sd,}\DataTypeTok{lower.tail=}\NormalTok{lower) }\OperatorTok{-}\StringTok{ }\NormalTok{yval}
    \KeywordTok{sum}\NormalTok{(difz}\OperatorTok{^}\DecValTok{2}\NormalTok{)}
\NormalTok{\}}
\end{Highlighting}
\end{Shaded}

\begin{Shaded}
\begin{Highlighting}[]
\CommentTok{# pdf for P(x) with f(beat) ~ Gamma}
\CommentTok{#' @param x diversity fluctuation value}
\CommentTok{#' @param shape the shape parameter of the gamma distribution}
\CommentTok{#' @param rate the rate parameter  of the gamma distribution}

\NormalTok{Px.gam <-}\StringTok{ }\NormalTok{PxGam <-}\StringTok{ }\ControlFlowTok{function}\NormalTok{(x, shape, rate) \{}
\NormalTok{    scale <-}\StringTok{ }\DecValTok{1} \OperatorTok{/}\StringTok{ }\NormalTok{rate}
\NormalTok{    n <-}\StringTok{ }\DecValTok{2} \OperatorTok{*}\StringTok{ }\NormalTok{shape}
\NormalTok{    b0 <-}\StringTok{ }\NormalTok{scale }\OperatorTok{*}\StringTok{ }\NormalTok{shape}
    
\NormalTok{    t1 <-}\StringTok{ }\KeywordTok{gamma}\NormalTok{((n}\OperatorTok{+}\DecValTok{1}\NormalTok{) }\OperatorTok{/}\StringTok{ }\DecValTok{2}\NormalTok{) }\OperatorTok{/}\StringTok{ }\KeywordTok{gamma}\NormalTok{(n }\OperatorTok{/}\StringTok{ }\DecValTok{2}\NormalTok{)}
\NormalTok{    t2 <-}\StringTok{ }\KeywordTok{sqrt}\NormalTok{(b0 }\OperatorTok{/}\StringTok{ }\NormalTok{(pi }\OperatorTok{*}\StringTok{ }\NormalTok{n))}
\NormalTok{    t3 <-}\StringTok{ }\NormalTok{(}\DecValTok{1} \OperatorTok{+}\StringTok{ }\NormalTok{(b0 }\OperatorTok{*}\StringTok{ }\NormalTok{x}\OperatorTok{^}\DecValTok{2}\NormalTok{) }\OperatorTok{/}\StringTok{ }\NormalTok{n)}\OperatorTok{^-}\NormalTok{((n }\OperatorTok{+}\StringTok{ }\DecValTok{1}\NormalTok{) }\OperatorTok{/}\StringTok{ }\DecValTok{2}\NormalTok{)}
    
\NormalTok{    t1 }\OperatorTok{*}\StringTok{ }\NormalTok{t2 }\OperatorTok{*}\StringTok{ }\NormalTok{t3}
\NormalTok{\}}

\CommentTok{# cdf for P(x) with f(beat) ~ Gamma}
\CommentTok{#' @param x diversity fluctuation value}
\CommentTok{#' @param shape the shape parameter of the gamma distribution}
\CommentTok{#' @param rate the rate parameter  of the gamma distribution}
\CommentTok{#' @param comp logical, whether to compute the complement or not (`comp = TRUE` is }
\CommentTok{#' equivilant to `lower.tail = FALSE` for typical `p` functions [e.g. `pnorm`])}

\NormalTok{PPx.gam <-}\StringTok{ }\NormalTok{PPxGam <-}\StringTok{ }\ControlFlowTok{function}\NormalTok{(x, shape, rate, }\DataTypeTok{comp=}\OtherTok{TRUE}\NormalTok{) \{}
    \ControlFlowTok{if}\NormalTok{(}\KeywordTok{length}\NormalTok{(x) }\OperatorTok{==}\StringTok{ }\DecValTok{1}\NormalTok{) \{}
\NormalTok{        intgral <-}\StringTok{ }\KeywordTok{integrate}\NormalTok{(PxGam, }\DecValTok{0}\NormalTok{, x, }\DataTypeTok{shape =}\NormalTok{ shape, }\DataTypeTok{rate =}\NormalTok{ rate)}
        \ControlFlowTok{if}\NormalTok{(intgral}\OperatorTok{$}\NormalTok{message }\OperatorTok{!=}\StringTok{ 'OK'}\NormalTok{) }\KeywordTok{print}\NormalTok{(intrgral}\OperatorTok{$}\NormalTok{message)}
\NormalTok{        val <-}\StringTok{ }\NormalTok{intgral}\OperatorTok{$}\NormalTok{value}
        
        \ControlFlowTok{if}\NormalTok{(comp) \{}
            \KeywordTok{return}\NormalTok{(}\DecValTok{1} \OperatorTok{-}\StringTok{ }\DecValTok{2} \OperatorTok{*}\StringTok{ }\NormalTok{val)}
\NormalTok{        \} }\ControlFlowTok{else}\NormalTok{ \{}
            \KeywordTok{return}\NormalTok{(}\DecValTok{2} \OperatorTok{*}\StringTok{ }\NormalTok{val)}
\NormalTok{        \}}
        
\NormalTok{    \} }\ControlFlowTok{else}\NormalTok{ \{}
        \CommentTok{# recursive handeling for multiple `x` values}
        \KeywordTok{return}\NormalTok{(}\KeywordTok{sapply}\NormalTok{(x, }\ControlFlowTok{function}\NormalTok{(X) }\KeywordTok{PPxGam}\NormalTok{(X, shape, rate, comp)))}
\NormalTok{    \}}
\NormalTok{\}}
\end{Highlighting}
\end{Shaded}

\begin{Shaded}
\begin{Highlighting}[]
\CommentTok{#' @description gives the log likelihood function under sstat model}
\CommentTok{#' @param par the parameter values}
\CommentTok{#' @param dat the data}

\NormalTok{sstatLL <-}\StringTok{ }\ControlFlowTok{function}\NormalTok{(par, dat) \{}
    \OperatorTok{-}\KeywordTok{sum}\NormalTok{(}\KeywordTok{log}\NormalTok{(}\KeywordTok{Px.gam}\NormalTok{(dat,par[}\DecValTok{1}\NormalTok{], par[}\DecValTok{2}\NormalTok{])))}
\NormalTok{\}}


\CommentTok{#' @description finds the maximum likelihood estimate of the superstats model}
\CommentTok{#' @param dat the data to fit}

\NormalTok{sstatMLE <-}\StringTok{ }\ControlFlowTok{function}\NormalTok{(dat) \{}
    \KeywordTok{optim}\NormalTok{(}\KeywordTok{c}\NormalTok{(}\FloatTok{0.55}\NormalTok{, }\FloatTok{0.17}\NormalTok{), sstatLL, }\DataTypeTok{method =} \StringTok{'BFGS'}\NormalTok{, }\DataTypeTok{hessian =} \OtherTok{TRUE}\NormalTok{, }
          \DataTypeTok{dat =}\NormalTok{ dat)}
\NormalTok{\}}


\CommentTok{#' @description bootstrap likelihood for super stats model}
\CommentTok{#' @param x the `sstat` object}
\CommentTok{#' @param B the number of boostrap replicates}
\CommentTok{#' @param useAll logical, whether all orders, or only those with the minimum number of }
\CommentTok{#' occurences as specified}
\CommentTok{#' in `make3TPub` argument `minPub` should be used}

\NormalTok{bootMLE.sstat <-}\StringTok{ }\ControlFlowTok{function}\NormalTok{(x, }\DataTypeTok{B =} \DecValTok{1000}\NormalTok{, }\DataTypeTok{useAll =} \OtherTok{FALSE}\NormalTok{) \{}
    \ControlFlowTok{if}\NormalTok{(useAll) \{}
\NormalTok{        theseDat <-}\StringTok{ }\NormalTok{x}\OperatorTok{$}\NormalTok{Px.raw}
\NormalTok{    \} }\ControlFlowTok{else}\NormalTok{ \{}
\NormalTok{        theseDat <-}\StringTok{ }\NormalTok{x}\OperatorTok{$}\NormalTok{Px.sub}
\NormalTok{    \}}
    
\NormalTok{    boots <-}\StringTok{ }\KeywordTok{replicate}\NormalTok{(B, \{}
\NormalTok{        subDat <-}\StringTok{ }\KeywordTok{sapply}\NormalTok{(theseDat, sample, }\DataTypeTok{size =} \DecValTok{1}\NormalTok{)}
        
\NormalTok{        thisMLE <-}\StringTok{ }\KeywordTok{try}\NormalTok{(}\KeywordTok{sstatMLE}\NormalTok{(subDat), }\DataTypeTok{silent =} \OtherTok{TRUE}\NormalTok{)}
        
        \ControlFlowTok{if}\NormalTok{(}\KeywordTok{class}\NormalTok{(thisMLE) }\OperatorTok{!=}\StringTok{  'try-error'}\NormalTok{) \{}
            \ControlFlowTok{if}\NormalTok{(thisMLE}\OperatorTok{$}\NormalTok{convergence }\OperatorTok{!=}\StringTok{  }\DecValTok{0}\NormalTok{) \{}
\NormalTok{                out <-}\StringTok{ }\KeywordTok{rep}\NormalTok{(}\OtherTok{NA}\NormalTok{, }\DecValTok{2}\NormalTok{)}
\NormalTok{            \} }\ControlFlowTok{else}\NormalTok{ \{}
\NormalTok{                out <-}\StringTok{ }\NormalTok{thisMLE}\OperatorTok{$}\NormalTok{par}
\NormalTok{            \}}
\NormalTok{        \} }\ControlFlowTok{else}\NormalTok{ \{}
\NormalTok{            out <-}\StringTok{ }\KeywordTok{rep}\NormalTok{(}\OtherTok{NA}\NormalTok{, }\DecValTok{2}\NormalTok{)}
\NormalTok{        \}}
\NormalTok{        out}
\NormalTok{    \})}
    
\NormalTok{    sstatOut <-}\StringTok{ }\KeywordTok{rbind}\NormalTok{(}\KeywordTok{quantile}\NormalTok{(boots[}\DecValTok{1}\NormalTok{, ], }\KeywordTok{c}\NormalTok{(}\FloatTok{0.025}\NormalTok{, }\FloatTok{0.975}\NormalTok{), }\DataTypeTok{na.rm =} \OtherTok{TRUE}\NormalTok{), }
                      \KeywordTok{quantile}\NormalTok{(boots[}\DecValTok{2}\NormalTok{, ], }\KeywordTok{c}\NormalTok{(}\FloatTok{0.025}\NormalTok{, }\FloatTok{0.975}\NormalTok{), }\DataTypeTok{na.rm =} \OtherTok{TRUE}\NormalTok{))}
    \KeywordTok{rownames}\NormalTok{(sstatOut) <-}\StringTok{ }\KeywordTok{c}\NormalTok{(}\StringTok{'shape'}\NormalTok{, }\StringTok{'rate'}\NormalTok{)}
    
    \KeywordTok{return}\NormalTok{(}\KeywordTok{list}\NormalTok{(}\DataTypeTok{sstat =}\NormalTok{ sstatOut))}
\NormalTok{\}}


\CommentTok{#' @description logLik for sstat class}
\CommentTok{#' @param x the `sstat` object}
\CommentTok{#' @param fitted logical, was the model fitted by max likelihood or computed from first }
\CommentTok{#' principles}
\CommentTok{#' @param useAll logical, should all data be used, or only those taxa that have greater }
\CommentTok{#' than `minN` occurrences}
\CommentTok{#' as specified in `sstatComp`}

\NormalTok{logLik.sstat <-}\StringTok{ }\ControlFlowTok{function}\NormalTok{(x, }\DataTypeTok{fitted =} \OtherTok{TRUE}\NormalTok{, }\DataTypeTok{useAll =} \OtherTok{FALSE}\NormalTok{) \{}
    \ControlFlowTok{if}\NormalTok{(useAll) \{}
\NormalTok{        theseDat <-}\StringTok{ }\KeywordTok{unlist}\NormalTok{(x}\OperatorTok{$}\NormalTok{Px.raw)}
\NormalTok{    \} }\ControlFlowTok{else}\NormalTok{ \{}
\NormalTok{        theseDat <-}\StringTok{ }\KeywordTok{unlist}\NormalTok{(x}\OperatorTok{$}\NormalTok{Px.sub)}
\NormalTok{    \}}
    
\NormalTok{    lik <-}\StringTok{ }\KeywordTok{sum}\NormalTok{(}\KeywordTok{log}\NormalTok{(x}\OperatorTok{$}\KeywordTok{Px}\NormalTok{(theseDat)))}
    
    \ControlFlowTok{if}\NormalTok{(fitted) \{}
        \KeywordTok{attr}\NormalTok{(lik, }\StringTok{'df'}\NormalTok{) <-}\StringTok{ }\DecValTok{2}
\NormalTok{    \} }\ControlFlowTok{else}\NormalTok{ \{}
        \KeywordTok{attr}\NormalTok{(lik, }\StringTok{'df'}\NormalTok{) <-}\StringTok{ }\DecValTok{0}
\NormalTok{    \}}
    
    \KeywordTok{class}\NormalTok{(lik) <-}\StringTok{ 'logLik'}
    
    \KeywordTok{return}\NormalTok{(lik)}
\NormalTok{\}}
\end{Highlighting}
\end{Shaded}

\begin{Shaded}
\begin{Highlighting}[]
\CommentTok{#' @description plot method for sstat class}
\CommentTok{#' @param x the `sstat` object}
\CommentTok{#' @param sstatCol color for super stats fit}
\CommentTok{#' @param normCol color for Gaussian fit}
\CommentTok{#' @param showNorm logical, should Gaussian fit be shown}
\CommentTok{#' @param addLegend logical, should legend be added}
\CommentTok{#' @param ... other parameters passed to `plot.default`}

\NormalTok{plot.sstat <-}\StringTok{ }\ControlFlowTok{function}\NormalTok{(x, }\DataTypeTok{sstatCol =} \StringTok{'red'}\NormalTok{, }\DataTypeTok{normCol =} \StringTok{'blue'}\NormalTok{, }
                       \DataTypeTok{showNorm =} \OtherTok{TRUE}\NormalTok{, }\DataTypeTok{addLegend =} \OtherTok{TRUE}\NormalTok{, ...) \{}
\NormalTok{    thisECDF <-}\StringTok{ }\NormalTok{socorro}\OperatorTok{::}\KeywordTok{simpECDF}\NormalTok{(}\KeywordTok{abs}\NormalTok{(}\KeywordTok{unlist}\NormalTok{(x}\OperatorTok{$}\NormalTok{Px.sub)), }\DataTypeTok{complement =} \OtherTok{TRUE}\NormalTok{)}
    
    \CommentTok{# helper function to deal with optional axis arguments}
\NormalTok{    .axissetup <-}\StringTok{ }\ControlFlowTok{function}\NormalTok{(side) \{}
        \ControlFlowTok{if}\NormalTok{(}\KeywordTok{sprintf}\NormalTok{(}\StringTok{'%saxt'}\NormalTok{, side) }\OperatorTok\StringTok{ }\KeywordTok{names}\NormalTok{(pargs)) \{}
            \ControlFlowTok{if}\NormalTok{(pargs[[}\KeywordTok{sprintf}\NormalTok{(}\StringTok{'%saxt'}\NormalTok{, side)]] }\OperatorTok{==}\StringTok{ 'n'}\NormalTok{) \{}
                \KeywordTok{assign}\NormalTok{(}\KeywordTok{sprintf}\NormalTok{(}\StringTok{'%saxfun'}\NormalTok{, side), }\ControlFlowTok{function}\NormalTok{(...) \{\}, }\DataTypeTok{pos =} \DecValTok{1}\NormalTok{)}
\NormalTok{            \} }\ControlFlowTok{else}\NormalTok{ \{}
                \ControlFlowTok{if}\NormalTok{(side }\OperatorTok\StringTok{ }\NormalTok{pargs}\OperatorTok{$}\NormalTok{log) \{}
                    \KeywordTok{assign}\NormalTok{(}\KeywordTok{sprintf}\NormalTok{(}\StringTok{'%saxfun'}\NormalTok{, side), socorro}\OperatorTok{::}\NormalTok{logAxis, }\DataTypeTok{pos =} \DecValTok{1}\NormalTok{)}
\NormalTok{                \} }\ControlFlowTok{else}\NormalTok{ \{}
                    \KeywordTok{assign}\NormalTok{(}\KeywordTok{sprintf}\NormalTok{(}\StringTok{'%saxfun'}\NormalTok{, side), axis, }\DataTypeTok{pos =} \DecValTok{1}\NormalTok{)}
\NormalTok{                \}}
\NormalTok{            \}}
\NormalTok{        \} }\ControlFlowTok{else}\NormalTok{ \{}
            \ControlFlowTok{if}\NormalTok{(side }\OperatorTok\StringTok{ }\NormalTok{pargs}\OperatorTok{$}\NormalTok{log) \{}
                \KeywordTok{assign}\NormalTok{(}\KeywordTok{sprintf}\NormalTok{(}\StringTok{'%saxfun'}\NormalTok{, side), socorro}\OperatorTok{::}\NormalTok{logAxis, }\DataTypeTok{pos =} \DecValTok{1}\NormalTok{)}
\NormalTok{            \} }\ControlFlowTok{else}\NormalTok{ \{}
                \KeywordTok{assign}\NormalTok{(}\KeywordTok{sprintf}\NormalTok{(}\StringTok{'%saxfun'}\NormalTok{, side), axis, }\DataTypeTok{pos =} \DecValTok{1}\NormalTok{)}
\NormalTok{            \}}
\NormalTok{        \}}
\NormalTok{    \}}
    
\NormalTok{    pargs <-}\StringTok{ }\KeywordTok{list}\NormalTok{(...)}
    \ControlFlowTok{if}\NormalTok{(}\OperatorTok{!}\NormalTok{(}\StringTok{'log'} \OperatorTok\StringTok{ }\KeywordTok{names}\NormalTok{(pargs))) pargs}\OperatorTok{$}\NormalTok{log <-}\StringTok{ 'xy'}
    \ControlFlowTok{if}\NormalTok{(}\OperatorTok{!}\NormalTok{(}\StringTok{'xlab'} \OperatorTok\StringTok{ }\KeywordTok{names}\NormalTok{(pargs))) pargs}\OperatorTok{$}\NormalTok{xlab <-}\StringTok{ '|Fluctuations|'}
    \ControlFlowTok{if}\NormalTok{(}\OperatorTok{!}\NormalTok{(}\StringTok{'ylab'} \OperatorTok\StringTok{ }\KeywordTok{names}\NormalTok{(pargs))) pargs}\OperatorTok{$}\NormalTok{ylab <-}\StringTok{ 'Cumulative density'}
    \KeywordTok{.axissetup}\NormalTok{(}\StringTok{'x'}\NormalTok{)}
    \KeywordTok{.axissetup}\NormalTok{(}\StringTok{'y'}\NormalTok{)}
    
\NormalTok{    pargs}\OperatorTok{$}\NormalTok{xaxt <-}\StringTok{ 'n'}
\NormalTok{    pargs}\OperatorTok{$}\NormalTok{yaxt <-}\StringTok{ 'n'}
    
    \KeywordTok{do.call}\NormalTok{(plot, }\KeywordTok{c}\NormalTok{(}\KeywordTok{list}\NormalTok{(}\DataTypeTok{x =}\NormalTok{ thisECDF), pargs))}
    \KeywordTok{xaxfun}\NormalTok{(}\DecValTok{1}\NormalTok{)}
    \KeywordTok{yaxfun}\NormalTok{(}\DecValTok{2}\NormalTok{)}
    
\NormalTok{    PPx <-}\StringTok{ }\NormalTok{x}\OperatorTok{$}\NormalTok{PPx}
    \KeywordTok{curve}\NormalTok{(}\KeywordTok{PPx}\NormalTok{(x, }\DataTypeTok{comp =} \OtherTok{TRUE}\NormalTok{), }\DataTypeTok{col =}\NormalTok{ sstatCol, }\DataTypeTok{lwd =} \DecValTok{2}\NormalTok{, }\DataTypeTok{add =} \OtherTok{TRUE}\NormalTok{)}
    
    \ControlFlowTok{if}\NormalTok{(showNorm) \{}
\NormalTok{        thisSD <-}\StringTok{ }\KeywordTok{sd}\NormalTok{(}\KeywordTok{unlist}\NormalTok{(x}\OperatorTok{$}\NormalTok{Px.sub))}
        \KeywordTok{curve}\NormalTok{(}\DecValTok{2}\OperatorTok{*}\KeywordTok{pnorm}\NormalTok{(x, }\DecValTok{0}\NormalTok{, thisSD, }\DataTypeTok{lower.tail =} \OtherTok{FALSE}\NormalTok{), }\DataTypeTok{col =}\NormalTok{ normCol, }\DataTypeTok{lwd =} \DecValTok{2}\NormalTok{, }\DataTypeTok{add =} \OtherTok{TRUE}\NormalTok{)}
\NormalTok{    \}}
    
    \ControlFlowTok{if}\NormalTok{(addLegend) \{}
\NormalTok{        leg <-}\StringTok{ }\KeywordTok{c}\NormalTok{(}\StringTok{'Observed'}\NormalTok{, }\StringTok{'Superstatistics'}\NormalTok{)}
\NormalTok{        col <-}\StringTok{ }\KeywordTok{c}\NormalTok{(}\KeywordTok{par}\NormalTok{(}\StringTok{'fg'}\NormalTok{), sstatCol)}
\NormalTok{        pch <-}\StringTok{ }\KeywordTok{c}\NormalTok{(}\KeywordTok{ifelse}\NormalTok{(}\StringTok{'pch'} \OperatorTok\StringTok{ }\KeywordTok{names}\NormalTok{(}\KeywordTok{list}\NormalTok{(...)), }\KeywordTok{list}\NormalTok{(...)}\OperatorTok{$}\NormalTok{pch, }\DecValTok{1}\NormalTok{), }\OtherTok{NA}\NormalTok{)}
\NormalTok{        pt.lwd <-}\StringTok{ }\KeywordTok{c}\NormalTok{(}\DecValTok{1}\NormalTok{, }\OtherTok{NA}\NormalTok{)}
\NormalTok{        pt.cex <-}\StringTok{ }\KeywordTok{c}\NormalTok{(}\DecValTok{1}\NormalTok{, }\OtherTok{NA}\NormalTok{)}
\NormalTok{        lwd <-}\StringTok{ }\KeywordTok{c}\NormalTok{(}\OtherTok{NA}\NormalTok{, }\DecValTok{2}\NormalTok{)}
        
        \ControlFlowTok{if}\NormalTok{(}\StringTok{'panel.first'} \OperatorTok\StringTok{ }\KeywordTok{names}\NormalTok{(}\KeywordTok{list}\NormalTok{(...))) \{}
\NormalTok{            leg <-}\StringTok{ }\KeywordTok{c}\NormalTok{(leg, }\StringTok{'Superstatistics CI'}\NormalTok{)}
\NormalTok{            col <-}\StringTok{ }\KeywordTok{c}\NormalTok{(col, socorro}\OperatorTok{::}\KeywordTok{colAlpha}\NormalTok{(sstatCol, }\FloatTok{0.25}\NormalTok{))}
\NormalTok{            pt.lwd <-}\StringTok{ }\KeywordTok{c}\NormalTok{(pt.lwd, }\DecValTok{1}\NormalTok{)}
\NormalTok{            pt.cex <-}\StringTok{ }\KeywordTok{c}\NormalTok{(pt.cex, }\DecValTok{2}\NormalTok{)}
\NormalTok{            lwd <-}\StringTok{ }\KeywordTok{c}\NormalTok{(lwd, }\OtherTok{NA}\NormalTok{)}
\NormalTok{            pch <-}\StringTok{ }\KeywordTok{c}\NormalTok{(pch, }\DecValTok{15}\NormalTok{)}
\NormalTok{        \}}
        
        \ControlFlowTok{if}\NormalTok{(showNorm) \{}
\NormalTok{            leg <-}\StringTok{ }\KeywordTok{c}\NormalTok{(leg, }\StringTok{'Gaussian'}\NormalTok{)}
\NormalTok{            col <-}\StringTok{ }\KeywordTok{c}\NormalTok{(col, normCol)}
\NormalTok{            pt.lwd <-}\StringTok{ }\KeywordTok{c}\NormalTok{(pt.lwd, }\OtherTok{NA}\NormalTok{)}
\NormalTok{            pt.cex <-}\StringTok{ }\KeywordTok{c}\NormalTok{(pt.cex, }\OtherTok{NA}\NormalTok{)}
\NormalTok{            lwd <-}\StringTok{ }\KeywordTok{c}\NormalTok{(lwd, }\DecValTok{2}\NormalTok{)}
\NormalTok{            pch <-}\StringTok{ }\KeywordTok{c}\NormalTok{(pch, }\OtherTok{NA}\NormalTok{)}
\NormalTok{        \}}
        
\NormalTok{        extracex <-}\StringTok{ }\KeywordTok{ifelse}\NormalTok{(}\StringTok{'cex'} \OperatorTok\StringTok{ }\KeywordTok{names}\NormalTok{(}\KeywordTok{list}\NormalTok{(...)), }\KeywordTok{list}\NormalTok{(...)}\OperatorTok{$}\NormalTok{cex, }\DecValTok{1}\NormalTok{)}
        \KeywordTok{legend}\NormalTok{(}\StringTok{'bottomleft'}\NormalTok{, }\DataTypeTok{legend =}\NormalTok{ leg, }\DataTypeTok{col =}\NormalTok{ col, }\DataTypeTok{pch =}\NormalTok{ pch, }\DataTypeTok{pt.lwd =}\NormalTok{ pt.lwd, }
               \DataTypeTok{pt.cex =}\NormalTok{ pt.cex}\OperatorTok{*}\NormalTok{extracex, }\DataTypeTok{lwd =}\NormalTok{ lwd, }\DataTypeTok{bty =} \StringTok{'n'}\NormalTok{)}
\NormalTok{    \}}
\NormalTok{\}}


\CommentTok{#' @description function to add confidence interval polygon from ML analysis}
\CommentTok{#' @param ci the matrix of CI intervals for the parameter values returned by `bootMLE.sstat`}
\CommentTok{#' @param fun the CDF function to plug the parameter values into}
\CommentTok{#' @param ... further arguments passed to `polygon` (e.g. `col`, `boarder`, etc.)}

\NormalTok{mlePoly <-}\StringTok{ }\ControlFlowTok{function}\NormalTok{(ci, fun, ...) \{}
\NormalTok{    n <-}\StringTok{ }\DecValTok{50}
\NormalTok{    x <-}\StringTok{ }\KeywordTok{seq}\NormalTok{(}\KeywordTok{par}\NormalTok{(}\StringTok{'usr'}\NormalTok{)[}\DecValTok{1}\NormalTok{], }\KeywordTok{par}\NormalTok{(}\StringTok{'usr'}\NormalTok{)[}\DecValTok{2}\NormalTok{], }\DataTypeTok{length =}\NormalTok{ n)}
\NormalTok{    x <-}\StringTok{ }\KeywordTok{c}\NormalTok{(x, }\KeywordTok{rev}\NormalTok{(x))}
    
    \ControlFlowTok{if}\NormalTok{(}\KeywordTok{par}\NormalTok{(}\StringTok{'xlog'}\NormalTok{)) x <-}\StringTok{ }\DecValTok{10}\OperatorTok{^}\NormalTok{x}
    
\NormalTok{    y <-}\StringTok{ }\KeywordTok{c}\NormalTok{(}\KeywordTok{fun}\NormalTok{(x[}\DecValTok{1}\OperatorTok{:}\NormalTok{n], ci[}\DecValTok{1}\NormalTok{, }\DecValTok{1}\NormalTok{], ci[}\DecValTok{2}\NormalTok{, }\DecValTok{2}\NormalTok{]), }\KeywordTok{fun}\NormalTok{(x[(}\DecValTok{1}\OperatorTok{:}\NormalTok{n) }\OperatorTok{+}\StringTok{ }\NormalTok{n], ci[}\DecValTok{1}\NormalTok{, }\DecValTok{2}\NormalTok{], ci[}\DecValTok{2}\NormalTok{, }\DecValTok{1}\NormalTok{]))}
    
    \KeywordTok{polygon}\NormalTok{(}\DataTypeTok{x =}\NormalTok{ x, }\DataTypeTok{y =}\NormalTok{ y, ...)}
\NormalTok{\}}
\end{Highlighting}
\end{Shaded}

\begin{Shaded}
\begin{Highlighting}[]
\NormalTok{sstatComp <-}\StringTok{ }\ControlFlowTok{function}\NormalTok{(grp.data,}\DataTypeTok{minN=}\DecValTok{15}\NormalTok{,}\DataTypeTok{xlab=}\StringTok{"Absolute Fluctuation"}\NormalTok{,}
                      \DataTypeTok{ylab=}\StringTok{"Cumulative Density"}\NormalTok{,}\DataTypeTok{leg=}\OtherTok{TRUE}\NormalTok{,}\DataTypeTok{plotit=}\OtherTok{TRUE}\NormalTok{) \{}
\NormalTok{    these2use <-}\StringTok{ }\KeywordTok{sapply}\NormalTok{(grp.data,length) }\OperatorTok{>=}\StringTok{ }\NormalTok{minN}
\NormalTok{    p2use <-}\StringTok{ }\NormalTok{grp.data[these2use]}
    
    \KeywordTok{cat}\NormalTok{(}\StringTok{"computing Gaussian fit for p_k(x|sigma) }\CharTok{\textbackslash{}n}\StringTok{"}\NormalTok{)}
\NormalTok{    pk.par <-}\StringTok{ }\KeywordTok{sapply}\NormalTok{(p2use,}\ControlFlowTok{function}\NormalTok{(x) }\KeywordTok{unlist}\NormalTok{(}\KeywordTok{normLS}\NormalTok{(x)[}\KeywordTok{c}\NormalTok{(}\StringTok{"par"}\NormalTok{,}\StringTok{"value"}\NormalTok{)]))}
\NormalTok{    pk.par <-}\StringTok{ }\KeywordTok{t}\NormalTok{(pk.par)}
    \KeywordTok{colnames}\NormalTok{(pk.par) <-}\StringTok{ }\KeywordTok{c}\NormalTok{(}\StringTok{"mu"}\NormalTok{,}\StringTok{"sig"}\NormalTok{,}\StringTok{"ss"}\NormalTok{)}
    
    \KeywordTok{cat}\NormalTok{(}\StringTok{"re-centering }\CharTok{\textbackslash{}n}\StringTok{"}\NormalTok{)}
    \ControlFlowTok{for}\NormalTok{(i }\ControlFlowTok{in} \DecValTok{1}\OperatorTok{:}\KeywordTok{length}\NormalTok{(p2use)) \{}
\NormalTok{        p2use[[i]] <-}\StringTok{ }\NormalTok{p2use[[i]] }\OperatorTok{-}\StringTok{ }\NormalTok{pk.par[i,}\StringTok{"mu"}\NormalTok{]}
\NormalTok{    \}}
    
    \KeywordTok{cat}\NormalTok{(}\StringTok{"computing f(beta) }\CharTok{\textbackslash{}n}\StringTok{"}\NormalTok{)}
\NormalTok{    f.beta.par <-}\StringTok{ }\KeywordTok{gammaLS}\NormalTok{(}\DecValTok{1}\OperatorTok{/}\NormalTok{(pk.par[,}\StringTok{"sig"}\NormalTok{])}\OperatorTok{^}\DecValTok{2}\NormalTok{)}\OperatorTok{$}\NormalTok{par}
\NormalTok{    fuent.par <-}\StringTok{ }\KeywordTok{c}\NormalTok{(}\DataTypeTok{n=}\DecValTok{2}\OperatorTok{*}\NormalTok{f.beta.par[}\DecValTok{1}\NormalTok{],}\DataTypeTok{b0=}\NormalTok{f.beta.par[}\DecValTok{1}\NormalTok{]}\OperatorTok{*}\NormalTok{f.beta.par[}\DecValTok{2}\NormalTok{])}
    
    \KeywordTok{cat}\NormalTok{(}\StringTok{"computing P(x) }\CharTok{\textbackslash{}n}\StringTok{"}\NormalTok{)}
\NormalTok{    this.Px <-}\StringTok{ }\ControlFlowTok{function}\NormalTok{(x) }\KeywordTok{Px.gam}\NormalTok{(x,f.beta.par[}\DecValTok{1}\NormalTok{],f.beta.par[}\DecValTok{2}\NormalTok{])}
\NormalTok{    this.PPx <-}\StringTok{ }\ControlFlowTok{function}\NormalTok{(x,}\DataTypeTok{comp=}\OtherTok{TRUE}\NormalTok{) }\KeywordTok{PPxGam}\NormalTok{(x,f.beta.par[}\DecValTok{1}\NormalTok{],f.beta.par[}\DecValTok{2}\NormalTok{],comp)}
    
\NormalTok{    out <-}\StringTok{ }\KeywordTok{list}\NormalTok{(}\DataTypeTok{gam.par=}\NormalTok{f.beta.par,}\DataTypeTok{sspar=}\NormalTok{fuent.par,}\DataTypeTok{beta=}\DecValTok{1}\OperatorTok{/}\NormalTok{(pk.par[,}\StringTok{"sig"}\NormalTok{])}\OperatorTok{^}\DecValTok{2}\NormalTok{,}
                \DataTypeTok{sumSq=}\NormalTok{pk.par[,}\StringTok{"ss"}\NormalTok{],}\DataTypeTok{minN=}\NormalTok{minN,}\DataTypeTok{raw.pk=}\NormalTok{p2use,}
                \DataTypeTok{Px.raw=}\NormalTok{grp.data,}\DataTypeTok{Px.sub=}\NormalTok{p2use,}\DataTypeTok{incld=}\NormalTok{these2use,}\DataTypeTok{Px=}\NormalTok{this.Px,}\DataTypeTok{PPx=}\NormalTok{this.PPx)}
    
    \KeywordTok{class}\NormalTok{(out) <-}\StringTok{ "sstat"}
    
    \ControlFlowTok{if}\NormalTok{(plotit) }\KeywordTok{plot}\NormalTok{(out)}
    
    \KeywordTok{return}\NormalTok{(out)}
\NormalTok{\}}
\end{Highlighting}
\end{Shaded}


\end{document}
